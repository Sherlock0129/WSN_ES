% IEEE conference paper skeleton for S2 manuscript (separate from main.tex)
\documentclass[conference]{IEEEtran}
\IEEEoverridecommandlockouts

\usepackage{cite}
\usepackage{amsmath,amssymb,amsfonts}
\usepackage{algorithmic}
\usepackage{graphicx}
\usepackage{textcomp}
\usepackage{xcolor}
\usepackage{hyperref}
\usepackage{CJKutf8}

\def\BibTeX{{\rm B\kern-.05em{\sc i\kern-.025em b}\kern-.08em
    T\kern-.1667em\lower.7ex\hbox{E}\kern-.125emX}}

% Temporary switch: use inline bibliography to avoid BibTeX errors during skeleton phase.
% Set to \draftbibfalse later to switch back to BibTeX.
\newif\ifdraftbib
\draftbibtrue
\ifdraftbib
\makeatletter
\AtBeginDocument{%
  \immediate\write\@auxout{\string\bibstyle{IEEEtran}}%
  \immediate\write\@auxout{\string\bibdata{references}}%
  \immediate\write\@auxout{\string\citation{placeholder2025}}%
}
\makeatother
\fi

\begin{document}

\begin{CJK}{UTF8}{gbsn}

% 中文题名
\title{融合AOEI驱动与数字孪生技术的能量共享经济机制研究}

\author{\IEEEauthorblockN{Author Name}
\IEEEauthorblockA{\textit{Department/School} \\
\textit{University Name}\\
City, Country \\
email@university.edu}
}

\maketitle

\begin{abstract}
% Placeholder: to be written in Phase 2 after consensus on scope and claims.
\end{abstract}

\begin{IEEEkeywords}
无线传感器网络,能量共享,AOEI,数字孪生,经济机制
\end{IEEEkeywords}

\section{Introduction}
% Phase 2: Problem background, economic-first perspective, motivation to expand Pareto frontier, contributions.

\section{Related Work}
% Phase 2: Categorize prior work (efficiency/variance-focused vs. mechanism interpretability), identify gaps.

\section{Modeling}
\subsection{Node and Scenario Modeling}
% Phase 2: Node energy, harvesting, communication, mobility; physical center; topology assumptions.
\subsection{Economic Interpretation and AOEI}
% Phase 2: AOEI as price signal; InfoNode as market account; transparency via opportunistic reporting.
\subsection{Objectives and Constraints}
% Phase 2: Efficiency, balance, fairness, freshness; no code-level details.

\section{Problems and Mechanism Design}
% Phase 3: Map economic problems to technical levers; design of AOEI triggers, InfoNode/Digital Twin, routing policy, adaptive waiting/length; avoid code specifics.

\section{Experiments}
\subsection{Setup}
% Phase 4: Scenarios (sizes, area, models), toggles.
\subsection{Metrics}
% Phase 4: Lifetime, CV, efficiency, death rate, freshness.
\subsection{Baselines}
% Phase 4: No-sharing, Lyapunov, DurationAware, DQN, DDPG.
\subsection{Ablation Studies}
% Phase 4: Remove AOEI, dedup/waiting, fixed duration.
\subsection{Visualization}
% Phase 4: Topology heatmap, paths, time series, dynamic K, Pareto frontier.

\section{Conclusion}
% Phase 5: Findings, implications, limitations.
\subsection{Future Work}
% Phase 5: Multi-agent, mobility, harvesting integration, transferability.

\ifdraftbib
\begin{thebibliography}{1}
\bibitem{placeholder2025}
T.~B.~Decided, ``Placeholder Reference for Compilation,'' 2025.
\end{thebibliography}
\else
\nocite{placeholder2025}
\bibliographystyle{IEEEtran}
\bibliography{references}
\fi

\end{CJK}
\end{document}


