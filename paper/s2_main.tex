% IEEE conference paper skeleton for S2 manuscript (separate from main.tex)
\documentclass[conference]{IEEEtran}
\IEEEoverridecommandlockouts

\usepackage{cite}
\usepackage{amsmath,amssymb,amsfonts}
\usepackage{algorithmic}
\usepackage{graphicx}
\usepackage{textcomp}
\usepackage{xcolor}
\usepackage{hyperref}
\usepackage{CJKutf8}

\def\BibTeX{{\rm B\kern-.05em{\sc i\kern-.025em b}\kern-.08em
    T\kern-.1667em\lower.7ex\hbox{E}\kern-.125emX}}

% Temporary switch: use inline bibliography to avoid BibTeX errors during skeleton phase.
% Set to \draftbibfalse later to switch back to BibTeX.
\newif\ifdraftbib
\draftbibtrue
\ifdraftbib
\makeatletter
\AtBeginDocument{%
  \immediate\write\@auxout{\string\bibstyle{IEEEtran}}%
  \immediate\write\@auxout{\string\bibdata{references}}%
  \immediate\write\@auxout{\string\citation{placeholder2025}}%
}
\makeatother
\fi

\begin{document}

\begin{CJK}{UTF8}{gbsn}

% 中文题名
\title{融合AOEI驱动与数字孪生技术的能量共享经济机制研究}

\author{\IEEEauthorblockN{Author Name}
\IEEEauthorblockA{\textit{Department/School} \\
\textit{University Name}\\
City, Country \\
email@university.edu}
}

\maketitle

\begin{abstract}
% Placeholder: to be written in Phase 2 after consensus on scope and claims.
\end{abstract}

\begin{IEEEkeywords}
无线传感器网络,能量共享,AOEI,数字孪生,经济机制
\end{IEEEkeywords}

\section{Introduction}
无线传感器网络(WSN)广泛部署于环境监测、工业物联网与智慧城市等场景,但长期运行受制于节点能量约束与信息时效性的双重瓶颈。传统以工程优化为中心的做法(如仅最小化能量方差或传输损耗)在固定约束下往往只能在既有的静态帕累托边界上做内部权衡,难以通过机制性改造拓展系统可达的效率—公平解集。本文转向以经济学为先导的设计路径:将信息的新鲜度与价值显式纳入资源配置,利用“能量信息年龄(Age of Energy Information, AOEI)”作为价格信号驱动能量共享时机与强度,通过数字孪生化的 InfoNode 提升市场透明度与状态可得性,并辅以传输路由与时长的自适应调控,从而在不改变物理资源总量的前提下,实现从静态到动态帕累托边界的外推。

具体而言,我们建立一套“AOEI 作为价格信号、InfoNode 作为信息孪生账户、机会主义信息上报作为交易规则”的能量共享经济机制:
(1)AOEI 驱动的触发机制在信息价值高、紧急性强时优先引发传能行为;
(2)InfoNode 以数字孪生方式维护节点多层状态(当前/历史/预测),配合信息去重与动态等待,保障市场的高可得性与低开销透明度;
(3)能量传输专用路由(如 EETOR)在效率阈值与多跳损耗间进行抑制性选择,减少低效路径;
(4)在帕累托视角下给出机制性证据,表明系统解集由静态边界外移至动态边界,兼顾效率与公平。

本文的主要贡献包括:
\begin{itemize}
    \item 提出以 AOEI 为价格信号的能量共享触发机制,将信息新鲜度—价值—紧急性纳入系统性决策;
    \item 构建 InfoNode 数字孪生,以信息去重与动态等待实现高可得性、低冗余的状态透明;
    \item 结合能量传输专用路由的效率阈值与多跳抑制策略,降低无效损耗并保护脆弱节点;
    \item 从帕累托边界视角分析机制带来的动态外移与效率—公平权衡改善。
\end{itemize}

\section{Related Work}
相关研究可概括为四类:(i)面向效率/方差的能量共享优化,典型目标是降低能量不均衡或传输损耗;(ii)以 Lyapunov/凸优化为代表的均衡方法,提供收敛性与稳定性保证;(iii)基于深度强化学习(如 DQN/DDPG)的自适应调度,在复杂状态下学习近似最优策略;(iv)围绕市场机制/信息价值/AoI/数字孪生的资源分配思想,强调机制解释性与状态可得性。

上述路径在不同侧面取得进展,但仍存在三点不足。其一,多数方法将“信息新鲜度—价值—紧急性”作为外生变量处理,触发传能与配给强度缺乏可解释的价格信号;其二,系统状态的可得性与透明度不足,导致策略依赖静态或周期性上报,时效性与开销难以兼顾;其三,路由常以通用通信准则设计,未充分体现能量传输的效率阈值与多跳抑制规律。本文从经济学机制切入,以 AOEI 价格信号与 InfoNode 数字孪生为核心,联合专用路由与自适应等待/时长,构造“可解释—可获取—低开销”的一体化机制,并在帕累托视角下讨论动态边界外移的系统性收益。为保证篇幅重点,本文仅择要对标“无共享、Lyapunov、DurationAware、DQN、DDPG”五类基线,不展开冗长综述。

\section{Modeling}
\subsection{Node and Scenario Modeling}
本节给出实体与场景建模(不涉及代码细节)。

节点建模:网络包含普通传感器节点与物理中心节点(ID=0,作为信息汇聚,不参与传能)。普通节点具备能量存储、采集与消耗三类过程:感知/计算/通信消耗,及(可选的)太阳能采集补给。节点可为静止或移动,其位置决定可行邻居与路径损耗。每个节点在信息空间上对应一个 InfoNode(见下一小节),用于管理多层状态。

场景建模:考虑多种拓扑与分布(均匀、随机、含能量空洞等),规模取典型数量级(如 10–100 节点)。环境层面,太阳能采集随日间时刻呈周期性变化,体现“白天富余、夜间紧缺”的时变供给;链路层面,能量传输效率随距离衰减,存在显著多跳损耗累积;管理层面,可配置最大跳数与效率阈值,避免极低效路径。上述设定为后续机制与实验提供统一语义空间。
\subsection{Economic Interpretation and AOEI}
经济学解释以“价格信号—账户—交易规则”三元组展开。首先,AOEI 作为价格信号,将信息新鲜度与价值显性化:当某节点的关键信息陈旧且影响系统目标(寿命/公平/效率)时,AOEI 升高,相当于“愿付更高价格”以换取及时能量支持,触发优先传能。其次,InfoNode 是节点在信息市场中的“数字孪生账户”,维护多层状态(当前、历史、预测/不确定度),并提供统一查询接口。再次,机会主义信息上报在执行能量传输时“搭载”路径所经节点信息,辅以信息去重与动态等待:信息量小则延迟上报以聚合,信息量大则缩短等待以保新鲜;在不增加通信开销的前提下维持较高透明度与可得性。

在此机制下,价格信号决定“何时/对谁/以多大强度”传能,账户保障“看得见的状态”,交易规则降低“获取状态的边际成本”。结合能量传输专用路由的效率阈值与多跳抑制,可以在局部理性决策下达成全局可解释的资源再分配。
\subsection{Objectives and Constraints}
系统目标与约束包括:
\begin{itemize}
    \item 网络寿命最大化(首个节点死亡时间);
    \item 能量均衡与公平(如方差/变异系数 CV 降低、低能节点受保护);
    \item 能量效率提升(有效接收能量占比提高、无效损耗降低);
    \item 信息新鲜度提升(关键状态的时效性保障);
    \item 约束:效率阈值、最大跳数、预算与调度频率等系统性限制。
\end{itemize}

从帕累托视角,静态边界对应在给定约束下的最优效率—公平权衡;通过 AOEI 价格信号、数字孪生与交易规则的机制性改造,可获得对可行集的外推,表现为在同等资源条件下实现更优的目标组合(动态边界外移)。本文在实验部分给出相应证据与对标结果。

\section{Problems and Mechanism Design}
% Phase 3: Map economic problems to technical levers; design of AOEI triggers, InfoNode/Digital Twin, routing policy, adaptive waiting/length; avoid code specifics.

\section{Experiments}
\subsection{Setup}
% Phase 4: Scenarios (sizes, area, models), toggles.
\subsection{Metrics}
% Phase 4: Lifetime, CV, efficiency, death rate, freshness.
\subsection{Baselines}
% Phase 4: No-sharing, Lyapunov, DurationAware, DQN, DDPG.
\subsection{Ablation Studies}
% Phase 4: Remove AOEI, dedup/waiting, fixed duration.
\subsection{Visualization}
% Phase 4: Topology heatmap, paths, time series, dynamic K, Pareto frontier.

\section{Conclusion}
% Phase 5: Findings, implications, limitations.
\subsection{Future Work}
% Phase 5: Multi-agent, mobility, harvesting integration, transferability.

\ifdraftbib
\begin{thebibliography}{1}
\bibitem{placeholder2025}
T.~B.~Decided, ``Placeholder Reference for Compilation,'' 2025.
\end{thebibliography}
\else
\nocite{placeholder2025}
\bibliographystyle{IEEEtran}
\bibliography{references}
\fi

\end{CJK}
\end{document}


