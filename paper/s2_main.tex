% IEEE conference paper skeleton for S2 manuscript (separate from main.tex)
\documentclass[conference]{IEEEtran}
\IEEEoverridecommandlockouts

\usepackage{cite}
\usepackage{amsmath,amssymb,amsfonts}
\usepackage{algorithmic}
\usepackage{graphicx}
\usepackage{textcomp}
\usepackage{xcolor}
\usepackage{hyperref}
\usepackage{CJKutf8}

\def\BibTeX{{\rm B\kern-.05em{\sc i\kern-.025em b}\kern-.08em
    T\kern-.1667em\lower.7ex\hbox{E}\kern-.125emX}}

% Temporary switch: use inline bibliography to avoid BibTeX errors during skeleton phase.
% Set to \draftbibfalse later to switch back to BibTeX.
\newif\ifdraftbib
\draftbibtrue
\ifdraftbib
\makeatletter
\AtBeginDocument{%
  \immediate\write\@auxout{\string\bibstyle{IEEEtran}}%
  \immediate\write\@auxout{\string\bibdata{references}}%
  \immediate\write\@auxout{\string\citation{placeholder2025}}%
}
\makeatother
\fi

\begin{document}

\begin{CJK}{UTF8}{gbsn}

% 中文题名
\title{融合AOEI驱动与数字孪生技术的能量共享经济机制研究}

\author{\IEEEauthorblockN{Author Name}
\IEEEauthorblockA{\textit{Department/School} \\
\textit{University Name}\\
City, Country \\
email@university.edu}
}

\maketitle

\begin{abstract}
% Placeholder: to be written in Phase 2 after consensus on scope and claims.
\end{abstract}

\begin{IEEEkeywords}
无线传感器网络,能量共享,AOEI,数字孪生,经济机制
\end{IEEEkeywords}

\section{Introduction}
无线传感器网络(Wireless Sensor Networks, WSN)作为典型的边缘感知基础设施,已广泛服务于环境监测、智慧城市、工业物联网与农业生产等关键场景\cite{surveyWSN}。然则,节点普遍受限于电池容量与能量采集不稳定性,系统长期运行面临两类根本性挑战:(i)能量约束与空间—时间不均衡导致网络寿命缩短与功能退化;(ii)信息时效性不足使得调度、路由与资源配置决策滞后,诱发系统性低效\cite{energyHarvestSurvey,aoiSurvey}。既有方法大多在工程优化框架内对能量方差、传输损耗或吞吐延迟进行单目标或多目标权衡\cite{energyBalSurvey,lyapunovEnergy},即便可以在给定约束下达成较优折中,其本质仍受限于既定的静态帕累托边界,难以通过制度性与机制性设计实现可行解空间的外延扩张。

本文采取“经济学先导—技术落地”的研究范式,核心主张是:将信息新鲜度与价值显式内生化到能量共享决策之中,以“能量信息年龄(Age of Energy Information, AOEI)”充当价格信号,刻画“何时/对谁/以何强度”触发能量传输的优先级;同时以数字孪生化的信息账户(InfoNode)提升市场透明度与可观测性\cite{digitalTwinSurvey},借助机会主义上报、信息去重与动态等待实现低开销的状态更新;并辅以面向能量传输的专用路由策略与传输时长的自适应调节\cite{eetor}。该机制意在在不改变物理资源总量的前提下,通过价格信号与制度设计的耦合\cite{marketMechanismWSN},使系统从“静态可达的折中集合”推移到“动态可达的扩展边界”,在效率与公平之间获得整体改进。

为支撑上述主张,本文构建一套“价格信号—数字孪生—交易规则—路径治理”的一体化机制:
(1)AOEI 作为内生化的价格信号,与信息的新鲜度、情境价值与紧急性同频变化,驱动能量共享触发的时机与强度;
(2)InfoNode 作为节点在信息市场中的数字孪生账户,维护多层状态(当前/历史/预测),并提供统一查询接口以提升市场透明度和可得性;
(3)机会主义信息上报结合信息去重与动态等待,实现“传能即上报、低冗余、保新鲜”的状态更新范式;
(4)能量传输专用路由(如 EETOR)以效率阈值与多跳抑制为原则,减少低效路径并保护脆弱节点;
(5)从帕累托边界视角评估机制外推效应,给出效率—公平权衡改善的证据与讨论。

本文的主要贡献如下:
\begin{itemize}
    \item 提出以 AOEI 为核心价格信号的能量共享触发机制,将信息新鲜度—价值—紧急性从外生变量转化为内生决策因子,统一于资源配置过程;
    \item 设计 InfoNode 数字孪生与机会主义信息上报的组合机制,配置信息去重与动态等待,实现高可得性、低通信开销与较强时效性的“透明市场”;
    \item 融合能量传输专用路由的效率阈值与多跳抑制策略,在全局可解释的框架下减少低效能量路径并提升系统鲁棒性;
    \item 基于帕累托边界的分析视角,论证机制性改造对可行解空间的外推效应,并从效率与公平两维度展示动态边界外移。
\end{itemize}

\section{Related Work}
现有研究可大致分为四条主线。其一,面向效率或方差的能量共享优化方法,典型目标在于降低能量不均衡、减少传输损耗或提高能量利用率\cite{energyBalSurvey};这类方法通常建立在确定性或近似静态的约束集合之上,强调工程可实现性,但往往将信息时效性与价值视为外部条件。其二,以 Lyapunov 或凸优化为代表的均衡框架,能够在理论上给出收敛性与稳定性保证\cite{lyapunovEnergy},具有较强的分析可解释性;然而,这类方法的触发与强度决策依赖预先设定的权衡参数,对于非平稳环境与异质场景的自适应能力受限。其三,基于深度强化学习(如 DQN、DDPG)的自适应调度在高维与非线性场景中显示出策略学习优势\cite{drlSurvey},能够端到端地近似最优策略;但其政策可解释性相对不足,且对状态可得性与信息时效性较为敏感\cite{aoiSurvey}。其四,围绕市场机制、信息价值、AoI/AoEI 与数字孪生的资源分配思想,提出以价格信号与制度设计提升系统可达解集的设想\cite{marketMechanismWSN,digitalTwinSurvey},强调“机制—行为—结果”的因果链条与可解释性。

综合观之,尚存三方面关键缺口:(i)“信息新鲜度—价值—紧急性”缺乏统一的价格化建模,触发传能的时机与强度难以与系统目标同构;(ii)状态可得性与透明度不足,常见的静态或周期上报机制在时效性与通信开销之间难以兼顾;(iii)路径治理多沿用通用通信路由原则,未充分体现能量传输的效率阈值与多跳抑制规律,进而导致全局能量效率受损。与此相对,本文以 AOEI 为价格信号、以 InfoNode 数字孪生为可得性基座,辅以机会主义上报与专用路由,面向“低开销—高时效—强可解释”的系统目标构造一体化机制。为凸显机制贡献并控制篇幅,本文仅与“无共享、Lyapunov、DurationAware、DQN、DDPG”五类基线进行对标,不展开冗长综述。

\section{Modeling}
\subsection{Node and Scenario Modeling}
本章建立不依赖具体实现细节的抽象化模型。

\textbf{节点建模:}网络由普通传感器节点与物理中心节点组成。物理中心节点(ID=0)不参与能量传输,其职责在于信息汇聚与全网状态维护。普通节点具备能量存储、采集与消耗过程:能量消耗来源于感知、计算与通信;能量采集(若启用)主要来自太阳能。节点可为静止或移动实体,其空间位置决定可行邻居、链路代价与潜在路径集合。每个物理节点在信息空间一一映射为 InfoNode,用于维护与暴露该节点的多层状态。

\textbf{场景建模:}从网络结构、环境供给与链路特性三个层面描述:
\begin{itemize}
    \item \emph{网络结构:}考虑均匀、随机与含能量空洞的多种空间分布,网络规模典型取值为 10–100 节点;可选启用节点移动以反映动态拓扑。
    \item \emph{环境供给:}太阳能采集呈显著的日内周期性变化,白天富余、夜间紧缺;可通过时变参数刻画非平稳能量供给。
    \item \emph{链路特性:}能量传输效率随距离衰减,多跳累积损耗显著;系统设置最大跳数与效率阈值以抑制极低效路径,保障全局能量效率与公平。
\end{itemize}
该三层刻画共同定义了机制运行的语义空间与外生扰动来源,为后续决策与评估提供统一背景。

\subsection{Economic Interpretation and AOEI}
经济学解释围绕“价格信号—数字账户—交易规则”的结构展开。首先,\emph{AOEI 作为价格信号}:当某一节点承担关键任务而其信息状态陈旧(新鲜度不足)且影响系统目标(寿命、效率与公平)时,应通过提高价格信号反映“更强支付意愿”,优先获得能量支持;AOEI 的时变性使其能够对不同时段、不同节点的重要性差异进行精细表达。其次,\emph{InfoNode 作为数字账户}:其维护当前、历史与预测层次的信息状态,并统一对外提供可查询的“透明市场视图”,弱化了自治决策之间的信息不对称。再次,\emph{交易规则}以机会主义信息上报为核心,结合信息去重与动态等待:在传能路径上“搭载”状态更新,在信息量较小时延迟聚合、在信息量较大时提前上报,以兼顾通信开销与信息新鲜度。

上述三者相互耦合:价格信号决定分配优先级与强度,数字账户保障状态可得,交易规则降低获取状态的边际成本;辅以面向能量传输的专用路由原则(效率阈值与多跳抑制),可在局部理性前提下导出全局可解释的资源再配置结果。

\subsection{Objectives and Constraints}
系统目标与约束定义如下:
\begin{itemize}
    \item \emph{寿命目标:}最大化首个节点死亡时间,体现系统持续服务能力;
    \item \emph{均衡与公平:}降低能量方差或变异系数(CV),对低能节点实施优先保护;
    \item \emph{效率目标:}提升有效接收能量占比,减少路径与链路的无效损耗;
    \item \emph{时效目标:}提升关键状态的新鲜度,降低决策滞后带来的系统性低效;
    \item \emph{约束条件:}效率阈值、最大跳数、预算与调度频率等策略层与系统层限制。
\end{itemize}
从帕累托视角看,静态边界对应于在既定约束集合下的最优效率—公平权衡;通过将 AOEI 价格信号与数字账户、交易规则制度化内生引入,可对可行解空间实现外推,即在相同资源与约束条件下获得更优的目标组合(动态边界外移)。本文在实验部分将以多指标对标与消融试验展示该外推效应。

\section{Problems and Mechanism Design}
本章按照“经济学问题(E)—技术问题(T)—机制设计(M)”的行文逻辑展开。我们首先明确目标向量(效率、均衡/公平、时效、寿命),随后围绕六类核心经济学问题依次建立技术映射与机制规则,强调价格信号内生化、状态可得性与透明度、路径外部性治理、强度(时长/额度)匹配、公平保护与非平稳鲁棒。该结构旨在表明:本章所有技术设计均服务于所识别的经济学问题,且由此带来的制度性改造可推动系统的动态帕累托边界外移\cite{paretoFrontier}。

\subsection{问题—映射—目标框架}
设时刻 \(t\) 的状态包含节点能量、拓扑与链路特性、环境供给与信息状态等。记 AOEI 为 \(A_i(t)\),InfoNode 状态集为 \(S_i(t)\),路径效率为 \(\eta(\cdot)\),传输时长为 \(\tau\)。我们的目标向量包括:(i)效率——提升有效接收能量占比、降低无效损耗;(ii)均衡/公平——降低方差或变异系数(CV),对低能分位或关键任务节点提供保护;(iii)时效——提升关键状态的新鲜度;(iv)寿命——延长首个节点死亡时间。映射原则为:将信息新鲜度—价值—紧急性价格化(内生化)为触发与排序依据;以数字孪生账户提高状态可得性与透明度\cite{digitalTwinSurvey};通过路径治理抑制负外部性(低效多跳);在预算与阈值约束下进行强度匹配与公平保护;对非平稳供给实施鲁棒调节\cite{marketMechanismWSN}。

\subsection{E1:信息价值定价缺失 \texorpdfstring{$\rightarrow$}{→} T1:触发/排序/预算清算 \texorpdfstring{$\rightarrow$}{→} M1:AOEI定价与触发}
\textbf{经济学问题(E1)}:缺乏可解释的价格信号会导致传能时机与对象选择失准,出现“该救的不救、该等的不等”的错配。

\textbf{技术问题(T1)}:如何将信息新鲜度、任务价值与紧急性内生化为触发与排序的价格化规则,并与预算一致?

\textbf{机制(M1)}:以 AOEI 为核心价格信号,构造价格函数 \(p_i(t)=f(A_i(t),\ \text{稀缺度},\ \text{能量占比})\)。采用分位数阈值或“预算下的前-K清算”以适配不同负载情形;当 \(A_i(t)\) 超过紧急阈值或节点能量低于安全线时,启用硬抢占优先。该设计满足单调性(\(A_i\uparrow \Rightarrow p_i\uparrow\))、预算一致性与弱势保护可加权性。预期影响体现在触发精确度、过/欠供比例与效率—公平改进幅度的提升上\cite{aoiSurvey,energyBalSurvey}。

\subsection{E2:市场不透明/状态不可得 \texorpdfstring{$\rightarrow$}{→} T2:低开销高时效的状态获取 \texorpdfstring{$\rightarrow$}{→} M2:InfoNode + 机会主义上报 + 去重/动态等待}
\textbf{经济学问题(E2)}:信息不对称与市场不透明会诱发错误定价与逆向选择,削弱机制的有效性。

\textbf{技术问题(T2)}:如何在近零额外通信开销下保障状态的可得性与新鲜度?

\textbf{机制(M2)}:以 InfoNode 为数字孪生账户,维护当前/历史/预测(含不确定度)的多层状态,并提供统一查询接口以提升透明度\cite{digitalTwinSurvey}。在执行传能时沿路径“搭载”上报,结合基于源节点集合的去重策略与动态等待上限 \(T_{\max}/(1+\text{info\_volume}/\text{scale})\),实现“低开销—高新鲜”的信息治理\cite{opportunisticInfo}。该设计一方面提升触发与路径选择的正确性,另一方面以显式规则平衡通信开销与时效性。

\subsection{E3:路径外部性与低效多跳 \texorpdfstring{$\rightarrow$}{→} T3:能量传输的路径治理 \texorpdfstring{$\rightarrow$}{→} M3:EETOR(效率阈值+最大跳数+保护策略)}
\textbf{经济学问题(E3)}:低效路径带来系统性负外部性,消耗大量资源并可能对弱势节点造成二次伤害。

\textbf{技术问题(T3)}:如何在能量传输特性下重写路由准则,避免“看似可达、实则巨亏”的多跳路径?

\textbf{机制(M3)}:采用面向能量传输的专用路由(EETOR)\cite{eetor}。以 \(\eta(d)=\eta_0/d^\gamma\) 作为效率模型,并在路径层面设定效率阈值与最大跳数;以“有效接收—损耗”为评分,联合“太阳能节点优先、低能节点保护”的制度性偏好,抑制低效多跳。该机制用于治理路径外部性,提升全局能量效率并降低弱势节点的受损风险。

\subsection{E4:配给强度与时长错配 \texorpdfstring{$\rightarrow$}{→} T4:强度/时长自适应 \texorpdfstring{$\rightarrow$}{→} M4:自适应时长/额度决策}
\textbf{经济学问题(E4)}:在异质节点与时变供给下,固定强度/时长导致边际效用错配。

\textbf{技术问题(T4)}:如何结合 AOEI、路径效率与预算约束联合决定传输时长/额度?

\textbf{机制(M4)}:提供两套接口以兼容不同学习/优化框架:离散时长(适配 DQN)与连续时长(适配 DDPG)\cite{drlSurvey,lyapunovEnergy}。时长/额度评分与 M1(价格信号)与 M3(路径治理)耦合,使边际收益在预算内最大化,并对非平稳供给具备自适应能力。评价指标包括单位能量收益、过/欠供比例与训练前后对比。

\subsection{E5:公平性与弱势保护 \texorpdfstring{$\rightarrow$}{→} T5:公平约束内生化 \texorpdfstring{$\rightarrow$}{→} M5:弱势权重与保护规则}
\textbf{经济学问题(E5)}:单纯的“效率优先”易造成结构性不公平,弱势节点被持续边缘化。

\textbf{技术问题(T5)}:如何在不显著牺牲效率的前提下实现内生公平保护?

\textbf{机制(M5)}:在价格与选择评分中引入弱势权重(低能量分位、关键任务节点),并对过度集中化的路径选择加入惩罚项,以提升服务的覆盖均衡性\cite{fairnessAlloc}。性质层面,可给出受保护概率/最低服务频度的下界与参数依赖。指标包括低分位能量轨迹、CV/最小能量与死亡节点数。

\subsection{E6:非平稳供给与鲁棒性 \texorpdfstring{$\rightarrow$}{→} T6:阈值/预算的自适应与前瞻 \texorpdfstring{$\rightarrow$}{→} M6:分位数阈值、动态预算与预测项}
\textbf{经济学问题(E6)}:日内供给与需求非平稳,静态参数配置易失效,并可能诱发策略振荡。

\textbf{技术问题(T6)}:如何让阈值、预算与优先级随时态调节并具备前瞻性?

\textbf{机制(M6)}:采用分位数阈值以顺应分布漂移,设置随时段或负载水平自适应的动态预算,并允许 InfoNode 的预测项参与优先级评估,提升稳健性与稳定性,抑制过度敏感与振荡\cite{aoiSurvey,energyHarvestSurvey}。评价维度包括昼/夜阶段性表现、振荡幅度与重配频率。

\subsection{小结与命题(性质与预期影响)}
为突出“机制—性质—影响”的因果链条,我们以命题形式陈述关键性质(不在此展开证明):
\begin{itemize}
    \item \textbf{命题1(单调性与预算一致性)}:在阈值与预算固定时,若 \(A_i(t)\) 单调上升,则对应节点的被服务概率不下降;在预算清算规则下,价格排序与资源分配相容。
    \item \textbf{命题2(外部性抑制)}:在效率阈值与最大跳数约束下,低效多跳路径的占比下降到给定上界,系统的无效损耗期望减少。
    \item \textbf{命题3(公平保护下界)}:在弱势权重与保护规则启用时,低能分位或关键任务节点的最低服务频度存在参数化下界。
    \item \textbf{命题4(边界外移的充分条件草案)}:当透明度增益(来自 InfoNode 与机会主义上报)与路径抑制(来自效率阈值与最大跳数)同时成立时,目标向量在效率—公平两维度上相对给定基线存在严格优势,从而对应动态帕累托边界外移\cite{paretoFrontier}.
\end{itemize}
上述性质为后续实验设计与对比评估提供理论支撑,并指导参数选择与消融试验的组织方式。

\section{Experiments}
\subsection{Setup}
% Phase 4: Scenarios (sizes, area, models), toggles.
\subsection{Metrics}
% Phase 4: Lifetime, CV, efficiency, death rate, freshness.
\subsection{Baselines}
% Phase 4: No-sharing, Lyapunov, DurationAware, DQN, DDPG.
\subsection{Ablation Studies}
% Phase 4: Remove AOEI, dedup/waiting, fixed duration.
\subsection{Visualization}
% Phase 4: Topology heatmap, paths, time series, dynamic K, Pareto frontier.

\section{Conclusion}
% Phase 5: Findings, implications, limitations.
\subsection{Future Work}
% Phase 5: Multi-agent, mobility, harvesting integration, transferability.

\ifdraftbib
\begin{thebibliography}{10}
\bibitem{placeholder2025}
T.~B.~Decided, ``Placeholder Reference for Compilation,'' 2025.
\bibitem{surveyWSN}
To~Be~Updated, ``A Survey on Wireless Sensor Networks,'' 2015--2025, placeholder entry.
\bibitem{energyHarvestSurvey}
To~Be~Updated, ``Energy Harvesting in Wireless Sensor Networks: A Survey,'' 2015--2025, placeholder entry.
\bibitem{energyBalSurvey}
To~Be~Updated, ``Energy Balancing and Sharing Strategies in WSN: A Review,'' 2015--2025, placeholder entry.
\bibitem{lyapunovEnergy}
To~Be~Updated, ``Lyapunov Optimization for Energy Management in Networks,'' 2010--2025, placeholder entry.
\bibitem{drlSurvey}
To~Be~Updated, ``Deep Reinforcement Learning for Networked Systems: A Survey,'' 2016--2025, placeholder entry.
\bibitem{aoiSurvey}
To~Be~Updated, ``Age of Information: A Survey and Applications,'' 2017--2025, placeholder entry.
\bibitem{digitalTwinSurvey}
To~Be~Updated, ``Digital Twin for CPS/IoT: Concepts and Applications,'' 2018--2025, placeholder entry.
\bibitem{marketMechanismWSN}
To~Be~Updated, ``Market-based Resource Allocation in Wireless Networks,'' 2008--2025, placeholder entry.
\bibitem{eetor}
To~Be~Updated, ``Energy-Efficient Transfer Opportunistic Routing (EETOR),'' 2019--2025, placeholder entry.
\bibitem{opportunisticInfo}
To~Be~Updated, ``Opportunistic Information Reporting in Networked Systems,'' 2012--2025, placeholder entry.
\bibitem{fairnessAlloc}
To~Be~Updated, ``Fairness-aware Resource Allocation in Wireless Networks,'' 2005--2025, placeholder entry.
\bibitem{paretoFrontier}
To~Be~Updated, ``Pareto Optimality and Frontier Expansion in Networked Systems,'' 2000--2025, placeholder entry.
\end{thebibliography}
\else
\nocite{placeholder2025}
\bibliographystyle{IEEEtran}
\bibliography{references}
\fi

\end{CJK}
\end{document}


