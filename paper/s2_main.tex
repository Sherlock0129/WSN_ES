% IEEE conference paper skeleton for S2 manuscript (separate from main.tex)
\documentclass[conference]{IEEEtran}
\IEEEoverridecommandlockouts

\usepackage{cite}
\usepackage{amsmath,amssymb,amsfonts}
\usepackage{algorithmic}
\usepackage{graphicx}
\usepackage{textcomp}
\usepackage{xcolor}
\usepackage{hyperref}
\usepackage{CJKutf8}

\def\BibTeX{{\rm B\kern-.05em{\sc i\kern-.025em b}\kern-.08em
    T\kern-.1667em\lower.7ex\hbox{E}\kern-.125emX}}

% Temporary switch: use inline bibliography to avoid BibTeX errors during skeleton phase.
% Set to \draftbibfalse later to switch back to BibTeX.
\newif\ifdraftbib
\draftbibtrue
\ifdraftbib
\makeatletter
\AtBeginDocument{%
  \immediate\write\@auxout{\string\bibstyle{IEEEtran}}%
  \immediate\write\@auxout{\string\bibdata{references}}%
  \immediate\write\@auxout{\string\citation{placeholder2025}}%
}
\makeatother
\fi

\begin{document}

\begin{CJK}{UTF8}{gbsn}

% 中文题名
\title{融合AOEI驱动与数字孪生技术的能量共享经济机制研究}

\author{\IEEEauthorblockN{Author Name}
\IEEEauthorblockA{\textit{Department/School} \\
\textit{University Name}\\
City, Country \\
email@university.edu}
}

\maketitle
\begin{abstract}
本文面向无线传感器网络(WSN)能量受限与信息时效性不足的双重挑战,提出一套以经济学机制为先导、以工程技术为落地的一体化能量共享框架。核心思想是将“能量信息年龄(Age of Energy Information, AOEI)”内生化为价格信号,驱动“何时/对谁/以何强度”触发传能;并以数字孪生化的 InfoNode 提升状态可得性与市场透明度,配合机会主义信息上报、去重与动态等待实现低开销的高新鲜度信息治理;同时采用面向能量传输的专用路由 EETOR(效率阈值 + 最大跳数 + 弱势保护)抑制低效多跳,再结合传输时长/额度的自适应决策与预算清算、分位数阈值与动态预算等非平稳鲁棒策略。该机制在不改变物理资源总量的前提下,通过价格化触发、状态可得性与路径外部性治理,协同实现效率、均衡/公平、时效与寿命四维度的联合提升,并在理论与实证上呈现“动态帕累托边界外移”。在规则网格、均匀随机与含能量空洞三类拓扑、\(N\in\{15,30,60,100\}\) 的设置下,本文与“无共享、Lyapunov、DurationAware、DQN、DDPG”等基线对比并开展系统化消融。结果显示:首个节点死亡时间、能量效率、CV 以及弱势服务覆盖均获得统计显著改进,低效多跳占比明显下降,信息新鲜度与昼/夜稳定性同步提升。上述证据验证了“经济学机制—技术实现”的可解释设计在 WSN 能量共享中的有效性与普适价值。
\end{abstract}

\begin{IEEEkeywords}
无线传感器网络,能量共享,AOEI,数字孪生,经济机制
\end{IEEEkeywords}

\section{Introduction}
无线传感器网络(Wireless Sensor Networks, WSN)作为典型的边缘感知基础设施,已广泛服务于环境监测、智慧城市、工业物联网与农业生产等关键场景\cite{surveyWSN}。然则,节点普遍受限于电池容量与能量采集不稳定性,系统长期运行面临两类根本性挑战:(i)能量约束与空间—时间不均衡导致网络寿命缩短与功能退化;(ii)信息时效性不足使得调度、路由与资源配置决策滞后,诱发系统性低效\cite{energyHarvestSurvey,aoiSurvey}。既有方法大多在工程优化框架内对能量方差、传输损耗或吞吐延迟进行单目标或多目标权衡\cite{energyBalSurvey,lyapunovEnergy},即便可以在给定约束下达成较优折中,其本质仍受限于既定的静态帕累托边界,难以通过制度性与机制性设计实现可行解空间的外延扩张。

本文采取“经济学先导—技术落地”的研究范式,核心主张是:将信息新鲜度与价值显式内生化到能量共享决策之中,以“能量信息年龄(Age of Energy Information, AOEI)”充当价格信号,刻画“何时/对谁/以何强度”触发能量传输的优先级;同时以数字孪生化的信息账户(InfoNode)提升市场透明度与可观测性\cite{digitalTwinSurvey},借助机会主义上报、信息去重与动态等待实现低开销的状态更新;并辅以面向能量传输的专用路由策略与传输时长的自适应调节\cite{eetor}。该机制意在在不改变物理资源总量的前提下,通过价格信号与制度设计的耦合\cite{marketMechanismWSN},使系统从“静态可达的折中集合”推移到“动态可达的扩展边界”,在效率与公平之间获得整体改进。

为支撑上述主张,本文构建一套“价格信号—数字孪生—交易规则—路径治理”的一体化机制:
(1)AOEI 作为内生化的价格信号,与信息的新鲜度、情境价值与紧急性同频变化,驱动能量共享触发的时机与强度;
(2)InfoNode 作为节点在信息市场中的数字孪生账户,维护多层状态(当前/历史/预测),并提供统一查询接口以提升市场透明度和可得性;
(3)机会主义信息上报结合信息去重与动态等待,实现“传能即上报、低冗余、保新鲜”的状态更新范式;
(4)能量传输专用路由(如 EETOR)以效率阈值与多跳抑制为原则,减少低效路径并保护脆弱节点;
(5)从帕累托边界视角评估机制外推效应,给出效率—公平权衡改善的证据与讨论。

本文的主要贡献如下:
\begin{itemize}
    \item 提出以 AOEI 为核心价格信号的能量共享触发机制,将信息新鲜度—价值—紧急性从外生变量转化为内生决策因子,统一于资源配置过程;
    \item 设计 InfoNode 数字孪生与机会主义信息上报的组合机制,配置信息去重与动态等待,实现高可得性、低通信开销与较强时效性的“透明市场”;
    \item 融合能量传输专用路由的效率阈值与多跳抑制策略,在全局可解释的框架下减少低效能量路径并提升系统鲁棒性;
    \item 基于帕累托边界的分析视角,论证机制性改造对可行解空间的外推效应,并从效率与公平两维度展示动态边界外移。
\end{itemize}

\section{Related Work}
现有研究可大致分为四条主线。其一,面向效率或方差的能量共享优化方法,典型目标在于降低能量不均衡、减少传输损耗或提高能量利用率\cite{energyBalSurvey};这类方法通常建立在确定性或近似静态的约束集合之上,强调工程可实现性,但往往将信息时效性与价值视为外部条件。其二,以 Lyapunov 或凸优化为代表的均衡框架,能够在理论上给出收敛性与稳定性保证\cite{lyapunovEnergy},具有较强的分析可解释性;然而,这类方法的触发与强度决策依赖预先设定的权衡参数,对于非平稳环境与异质场景的自适应能力受限。其三,基于深度强化学习(如 DQN、DDPG)的自适应调度在高维与非线性场景中显示出策略学习优势\cite{drlSurvey},能够端到端地近似最优策略;但其政策可解释性相对不足,且对状态可得性与信息时效性较为敏感\cite{aoiSurvey}。其四,围绕市场机制、信息价值、AoI/AoEI 与数字孪生的资源分配思想,提出以价格信号与制度设计提升系统可达解集的设想\cite{marketMechanismWSN,digitalTwinSurvey}。总体而言,前述多数工作仍停留于算法或度量层面:AoI/AoEI 常作为性能指标而非内生价格信号\cite{aoiSurvey};数字孪生多用于监测/仿真而未与资源分配闭环\cite{digitalTwinSurvey};路由多遵循通用通信准则而缺少针对能量传输效率阈值与多跳外部性的制度化治理。与之区别,本文强调“机制—行为—结果”的可解释链条。

综合观之,尚存三方面关键缺口:(i)“信息新鲜度—价值—紧急性”缺乏统一的价格化建模,触发传能的时机与强度难以与系统目标同构(机制层缺位);(ii)状态可得性与透明度不足,常见的静态或周期上报在时效性与通信开销之间难以兼顾(账户与交易规则未成体系);(iii)路径治理多沿用通用通信路由原则,未充分体现能量传输的效率阈值与多跳外部性(缺少外部性治理)。对此,本文在统一框架下将 AOEI→价格信号、InfoNode→数字账户、机会主义上报→交易规则、EETOR→路径治理、弱势权重/动态预算→公平与鲁棒性,组织为一套“低开销—高时效—强可解释”的一体化机制,并以“帕累托边界外移”作为统一评估主张。为凸显机制贡献并控制篇幅,本文仅与“无共享、Lyapunov、DurationAware、DQN、DDPG”五类基线进行对标,不展开冗长综述。



\section{Modeling}
\subsection{Node and Scenario Modeling}
本章建立不依赖具体实现细节的抽象化模型。

\textbf{节点建模:}网络由普通传感器节点与物理中心节点组成。物理中心节点(ID=0)不参与能量传输,其职责在于信息汇聚与全网状态维护。普通节点具备能量存储、采集与消耗过程:能量消耗来源于感知、计算与通信;能量采集(若启用)主要来自太阳能。节点可为静止或移动实体,其空间位置决定可行邻居、链路代价与潜在路径集合。每个物理节点在信息空间一一映射为 InfoNode,用于维护与暴露该节点的多层状态。

\textbf{场景建模:}从网络结构、环境供给与链路特性三个层面描述:
\begin{itemize}
    \item \emph{网络结构:}考虑均匀、随机与含能量空洞的多种空间分布,网络规模典型取值为 10–100 节点;可选启用节点移动以反映动态拓扑。
    \item \emph{环境供给:}太阳能采集呈显著的日内周期性变化,白天富余、夜间紧缺;可通过时变参数刻画非平稳能量供给。
    \item \emph{链路特性:}能量传输效率随距离衰减,多跳累积损耗显著;系统设置最大跳数与效率阈值以抑制极低效路径,保障全局能量效率与公平。
\end{itemize}
该三层刻画共同定义了机制运行的语义空间与外生扰动来源,为后续决策与评估提供统一背景。

\subsection{Economic Interpretation and AOEI}
经济学解释围绕“价格信号—数字账户—交易规则”的结构展开。首先,\emph{AOEI 作为价格信号}:当某一节点承担关键任务而其信息状态陈旧(新鲜度不足)且影响系统目标(寿命、效率与公平)时,应通过提高价格信号反映“更强支付意愿”,优先获得能量支持;AOEI 的时变性使其能够对不同时段、不同节点的重要性差异进行精细表达。其次,\emph{InfoNode 作为数字账户}:其维护当前、历史与预测层次的信息状态,并统一对外提供可查询的“透明市场视图”,弱化了自治决策之间的信息不对称。再次,\emph{交易规则}以机会主义信息上报为核心,结合信息去重与动态等待:在传能路径上“搭载”状态更新,在信息量较小时延迟聚合、在信息量较大时提前上报,以兼顾通信开销与信息新鲜度。

上述三者相互耦合:价格信号决定分配优先级与强度,数字账户保障状态可得,交易规则降低获取状态的边际成本;辅以面向能量传输的专用路由原则(效率阈值与多跳抑制),可在局部理性前提下导出全局可解释的资源再配置结果。

\subsection{Objectives and Constraints}
系统目标与约束定义如下:
\begin{itemize}
    \item \emph{寿命目标:}最大化首个节点死亡时间,体现系统持续服务能力;
    \item \emph{均衡与公平:}降低能量方差或变异系数(CV),对低能节点实施优先保护;
    \item \emph{效率目标:}提升有效接收能量占比,减少路径与链路的无效损耗;
    \item \emph{时效目标:}提升关键状态的新鲜度,降低决策滞后带来的系统性低效;
    \item \emph{约束条件:}效率阈值、最大跳数、预算与调度频率等策略层与系统层限制。
\end{itemize}
从帕累托视角看,静态边界对应于在既定约束集合下的最优效率—公平权衡;通过将 AOEI 价格信号与数字账户、交易规则制度化内生引入,可对可行解空间实现外推,即在相同资源与约束条件下获得更优的目标组合(动态边界外移)。本文在实验部分将以多指标对标与消融试验展示该外推效应。

\section{Problems and Mechanism Design}
本章按照“经济学问题(E)—技术问题(T)—机制设计(M)”的行文逻辑展开。我们首先明确目标向量(效率、均衡/公平、时效、寿命),随后围绕六类核心经济学问题依次建立技术映射与机制规则,强调价格信号内生化、状态可得性与透明度、路径外部性治理、强度(时长/额度)匹配、公平保护与非平稳鲁棒。该结构旨在表明:本章所有技术设计均服务于所识别的经济学问题,且由此带来的制度性改造可推动系统的动态帕累托边界外移\cite{paretoFrontier}。

\subsection{问题—映射—目标框架}
设时刻 \(t\) 的状态包含节点能量、拓扑与链路特性、环境供给与信息状态等。记 AOEI 为 \(A_i(t)\),InfoNode 状态集为 \(S_i(t)\),路径效率为 \(\eta(\cdot)\),传输时长为 \(\tau\)。我们的目标向量包括:(i)效率——提升有效接收能量占比、降低无效损耗;(ii)均衡/公平——降低方差或变异系数(CV),对低能分位或关键任务节点提供保护;(iii)时效——提升关键状态的新鲜度;(iv)寿命——延长首个节点死亡时间。映射原则为:将信息新鲜度—价值—紧急性价格化(内生化)为触发与排序依据;以数字孪生账户提高状态可得性与透明度\cite{digitalTwinSurvey};通过路径治理抑制负外部性(低效多跳);在预算与阈值约束下进行强度匹配与公平保护;对非平稳供给实施鲁棒调节\cite{marketMechanismWSN}。

\subsection{E1:信息价值定价缺失 \texorpdfstring{$\rightarrow$}{→} T1:触发/排序/预算清算 \texorpdfstring{$\rightarrow$}{→} M1:AOEI定价与触发}
\textbf{经济学问题(E1)}:缺乏可解释的价格信号会导致传能时机与对象选择失准,出现“该救的不救、该等的不等”的错配。

\textbf{技术问题(T1)}:如何将信息新鲜度、任务价值与紧急性内生化为触发与排序的价格化规则,并与预算一致?

\textbf{机制(M1)}:以 AOEI 为核心价格信号,构造价格函数 \(p_i(t)=f(A_i(t),\ \text{稀缺度},\ \text{能量占比})\)。采用分位数阈值或“预算下的前-K清算”以适配不同负载情形;当 \(A_i(t)\) 超过紧急阈值或节点能量低于安全线时,启用硬抢占优先。该设计满足单调性(\(A_i\uparrow \Rightarrow p_i\uparrow\))、预算一致性与弱势保护可加权性。预期影响体现在触发精确度、过/欠供比例与效率—公平改进幅度的提升上\cite{aoiSurvey,energyBalSurvey}。

\subsection{E2:市场不透明/状态不可得 \texorpdfstring{$\rightarrow$}{→} T2:低开销高时效的状态获取 \texorpdfstring{$\rightarrow$}{→} M2:InfoNode + 机会主义上报 + 去重/动态等待}
\textbf{经济学问题(E2)}:信息不对称与市场不透明会诱发错误定价与逆向选择,削弱机制的有效性。

\textbf{技术问题(T2)}:如何在近零额外通信开销下保障状态的可得性与新鲜度?

\textbf{机制(M2)}:以 InfoNode 为数字孪生账户,维护当前/历史/预测(含不确定度)的多层状态,并提供统一查询接口以提升透明度\cite{digitalTwinSurvey}。在执行传能时沿路径“搭载”上报,结合基于源节点集合的去重策略与动态等待上限 \(T_{\max}/(1+\text{info\_volume}/\text{scale})\),实现“低开销—高新鲜”的信息治理\cite{opportunisticInfo}。该设计一方面提升触发与路径选择的正确性,另一方面以显式规则平衡通信开销与时效性。

\subsection{E3:路径外部性与低效多跳 \texorpdfstring{$\rightarrow$}{→} T3:能量传输的路径治理 \texorpdfstring{$\rightarrow$}{→} M3:EETOR(效率阈值+最大跳数+保护策略)}
\textbf{经济学问题(E3)}:低效路径带来系统性负外部性,消耗大量资源并可能对弱势节点造成二次伤害。

\textbf{技术问题(T3)}:如何在能量传输特性下重写路由准则,避免“看似可达、实则巨亏”的多跳路径?

\textbf{机制(M3)}:采用面向能量传输的专用路由(EETOR)\cite{eetor}。以 \(\eta(d)=\eta_0/d^\gamma\) 作为效率模型,并在路径层面设定效率阈值与最大跳数;以“有效接收—损耗”为评分,联合“太阳能节点优先、低能节点保护”的制度性偏好,抑制低效多跳。该机制用于治理路径外部性,提升全局能量效率并降低弱势节点的受损风险。

\subsection{E4:配给强度与时长错配 \texorpdfstring{$\rightarrow$}{→} T4:强度/时长自适应 \texorpdfstring{$\rightarrow$}{→} M4:自适应时长/额度决策}
\textbf{经济学问题(E4)}:在异质节点与时变供给下,固定强度/时长导致边际效用错配。

\textbf{技术问题(T4)}:如何结合 AOEI、路径效率与预算约束联合决定传输时长/额度?

\textbf{机制(M4)}:提供两套接口以兼容不同学习/优化框架:离散时长(适配 DQN)与连续时长(适配 DDPG)\cite{drlSurvey,lyapunovEnergy}。时长/额度评分与 M1(价格信号)与 M3(路径治理)耦合,使边际收益在预算内最大化,并对非平稳供给具备自适应能力。评价指标包括单位能量收益、过/欠供比例与训练前后对比。

\subsection{E5:公平性与弱势保护 \texorpdfstring{$\rightarrow$}{→} T5:公平约束内生化 \texorpdfstring{$\rightarrow$}{→} M5:弱势权重与保护规则}
\textbf{经济学问题(E5)}:单纯的“效率优先”易造成结构性不公平,弱势节点被持续边缘化。

\textbf{技术问题(T5)}:如何在不显著牺牲效率的前提下实现内生公平保护?

\textbf{机制(M5)}:在价格与选择评分中引入弱势权重(低能量分位、关键任务节点),并对过度集中化的路径选择加入惩罚项,以提升服务的覆盖均衡性\cite{fairnessAlloc}。性质层面,可给出受保护概率/最低服务频度的下界与参数依赖。指标包括低分位能量轨迹、CV/最小能量与死亡节点数。

\subsection{E6:非平稳供给与鲁棒性 \texorpdfstring{$\rightarrow$}{→} T6:阈值/预算的自适应与前瞻 \texorpdfstring{$\rightarrow$}{→} M6:分位数阈值、动态预算与预测项}
\textbf{经济学问题(E6)}:日内供给与需求非平稳,静态参数配置易失效,并可能诱发策略振荡。

\textbf{技术问题(T6)}:如何让阈值、预算与优先级随时态调节并具备前瞻性?

\textbf{机制(M6)}:采用分位数阈值以顺应分布漂移,设置随时段或负载水平自适应的动态预算,并允许 InfoNode 的预测项参与优先级评估,提升稳健性与稳定性,抑制过度敏感与振荡\cite{aoiSurvey,energyHarvestSurvey}。评价维度包括昼/夜阶段性表现、振荡幅度与重配频率。

\subsection{小结与命题(性质与预期影响)}
为突出“机制—性质—影响”的因果链条,我们以命题形式陈述关键性质(不在此展开证明):
\begin{itemize}
    \item \textbf{命题1(单调性与预算一致性)}:在阈值与预算固定时,若 \(A_i(t)\) 单调上升,则对应节点的被服务概率不下降;在预算清算规则下,价格排序与资源分配相容。
    \item \textbf{命题2(外部性抑制)}:在效率阈值与最大跳数约束下,低效多跳路径的占比下降到给定上界,系统的无效损耗期望减少。
    \item \textbf{命题3(公平保护下界)}:在弱势权重与保护规则启用时,低能分位或关键任务节点的最低服务频度存在参数化下界。
    \item \textbf{命题4(边界外移的充分条件草案)}:当透明度增益(来自 InfoNode 与机会主义上报)与路径抑制(来自效率阈值与最大跳数)同时成立时,目标向量在效率—公平两维度上相对给定基线存在严格优势,从而对应动态帕累托边界外移\cite{paretoFrontier}.
\end{itemize}
上述性质为后续实验设计与对比评估提供理论支撑,并指导参数选择与消融试验的组织方式。

\section{Experiments}
\subsection{实验目标与总体协议}
本章旨在验证第四章机制在效率、均衡/公平、时效与寿命四个维度上的综合效益,并回答三个核心问题:(i)AOEI 价格信号与 InfoNode 数字孪生是否能够在相同资源约束下显著改善目标向量(动态边界外移)?(ii)EETOR 路由与时长/额度自适应是否能抑制低效多跳并提升单位能量收益?(iii)弱势保护与非平稳自适应是否在不显著牺牲效率的前提下提高公平与稳定性。为降低偶然性,我们对所有配置进行多随机种子重复,并报告均值与置信区间。

\subsection{Setup:场景与配置}
\textbf{网络场景与拓扑:}选取三类典型拓扑以覆盖从规则到随机、从均衡到不均衡的情形:(S1)规则网格(近似均匀覆盖);(S2)均匀随机(Poisson-like);(S3)含能量空洞(部分区域节点稀疏或受遮挡)。规模 \(N\in\{15,30,60,100\}\);区域大小参考中等尺度 WSN(例如 \(100\times 100\) 米量级),以便与既有研究对齐\cite{surveyWSN}。

\textbf{能量采集与负载:}启用日内太阳能供给的时变模型(白天富余、夜间紧缺),并设置常规感知/通信负载(强度中等),以体现非平稳供需\cite{energyHarvestSurvey}。为避免实现细节渗出,本文仅描述统计特征(均值/振幅/周期),具体参数置于复现清单。

\textbf{链路与路由约束:}能量传输效率 \(\eta(d)=\eta_0/d^\gamma\);路径层面设定效率阈值与最大跳数;通信开销与接收开销按常见能耗模型计入比较\cite{energyBalSurvey}。

\textbf{调度与预算:}被动传能触发以 AOEI 与预算清算主导,默认采用“分位数阈值 + 动态预算”作为鲁棒配置(昼/夜门限不同分位数);紧急抢占在 AOEI 超阈或能量低于安全线时生效。时长决策在 DQN(离散)与 DDPG(连续)两种接口间切换(分别用于不同基线),并对比固定时长策略。

\textbf{重复与统计:}每个配置下运行 10 次以上独立随机种子试验,报告均值 ± 95\% 置信区间;显著性检验采用成对非参数检验(如 Wilcoxon)与自助法置信区间\cite{statTest,bootstrap}。

\subsection{Metrics:评估指标}
为全面评估各机制在多目标下的效益,本文报告:
\begin{itemize}
    \item \textbf{网络寿命}(首个节点死亡时间):越大越好;
    \item \textbf{能量均衡度}(CV 与方差):越低越好;同时报告最小能量轨迹与低分位能量;
    \item \textbf{能量效率}(有效接收/总消耗):越高越好;同时报告单位能量收益(接收/发送);
    \item \textbf{传输效率}(路径效率分布与低效路径占比):越高越好、低效占比越低越好;
    \item \textbf{公平性}(弱势保护指标):低能分位节点的服务覆盖率/最低服务频度;
    \item \textbf{信息新鲜度}(关键状态 AoEI/AoI 相关指标):越新鲜越好(越低龄越好);
    \item \textbf{稳健性}(昼/夜分段表现、振荡幅度、重配频率):越稳健越好。
\end{itemize}

\subsection{Baselines:对照方法}
我们选取五类具有代表性的基线(与第二章口径一致):
\begin{itemize}
    \item \textbf{无能量共享}:仅依赖采集与消耗,作为下界参考;
    \item \textbf{Lyapunov}:以能量均衡为主目标的优化基线,具稳定性保证\cite{lyapunovEnergy};
    \item \textbf{DurationAware}:在 Lyapunov 框架内纳入时长成本的改进基线;
    \item \textbf{DQN}(离散时长):在离散动作空间学习时长/强度策略\cite{drlSurvey};
    \item \textbf{DDPG}(连续时长):在连续动作空间精细控制时长/强度\cite{drlSurvey}。
\end{itemize}
为保证公平性,所有方法共享相同的能量与通信模型、场景与评估协议;超参数遵循各方法常规设置,并在复现清单中列明。

\subsection{Ablation Studies:消融与变体}
为洞察各机制贡献,设计如下消融:
\begin{itemize}
    \item \textbf{去除 AOEI 价格信号}:以固定或静态阈值替代价格化触发,考察触发准确度变化;
    \item \textbf{去除 InfoNode/机会主义上报}:改为周期上报或无上报,考察新鲜度与开销变化;
    \item \textbf{去除去重/动态等待}:计算信息冗余与上报负载对性能的影响;
    \item \textbf{去除 EETOR 约束}(无效率阈值/最大跳数):观察低效多跳比例与系统能效变化;
    \item \textbf{固定时长}:替换自适应时长为固定值,对比单位能量收益与欠/过供比例;
    \item \textbf{去除弱势权重/保护规则}:考察公平性与最低服务频度的变化;
    \item \textbf{静态预算/阈值}:替换分位数阈值与动态预算为静态配置,观察昼/夜鲁棒性与振荡。
\end{itemize}

\subsection{Visualization \& Pareto Analysis:可视化与边界分析}
可视化与前沿分析用于直观呈现机制影响:
\begin{itemize}
    \item \textbf{拓扑与路径图}:展示传能路径分布、低效多跳抑制效果与弱势保护的空间特征;
    \item \textbf{时序曲线}:平均/最小能量、CV、效率、触发频率、预算使用率、AoEI 指标的时间演化;
    \item \textbf{效率分布与占比}:路径效率直方图与低效区间占比(随消融/基线比较);
    \item \textbf{Pareto 图}:以(效率、均衡/公平)、(寿命、效率)、(新鲜度、开销)等二维组合绘制前沿,展示相对基线的边界外移\cite{paretoFrontier}。
\end{itemize}
统计上,我们对关键指标进行成对检验与多重比较校正,并给出效应量(effect size)。

\subsection{Reproducibility:复现与合规}
为便于复现与审计,我们提供:随机种子列表、场景与超参数配置文件、运行脚本入口、日志与导出数据的目录结构约定,以及绘图脚本。所有实验在相同硬件/软件栈下运行;若启用深度学习基线,注明训练轮次与停止准则;若启用 GPU 加速,仅用于数值计算而不改变算法逻辑\cite{drlSurvey}。此外,我们在报告中标注潜在威胁:随机初始化敏感性、极端天气(太阳能骤降)场景下的稳健性、路径效率模型对真实硬件的外推风险等,并通过情景应对(异常日、极端分布)测试给出补充证据。

\subsection{Results:基线对比与总体效益}
我们在三类拓扑(S1/S2/S3)与四个规模(\(N\in\{15,30,60,100\}\))下,对比本文机制与五类基线。总体观察如下。
\begin{itemize}
    \item \textbf{寿命与效率:}在中大规模网络(\(N\ge 30\))与非平稳供给下,本文机制在首死时间与能量效率上相对基线呈稳健优势;该优势在含能量空洞场景(S3)更为显著,表明路径外部性治理与弱势保护的协同作用。
    \item \textbf{均衡/公平:}相较于仅效率导向的策略,本文在 CV 与最低能量分位轨迹上兼顾改善,弱势节点的最低服务频度显著上升(显著性检验通过,效应量中等以上),体现了内生公平约束的有效性。
    \item \textbf{新鲜度与时效:}在 InfoNode 与机会主义上报支撑下,关键状态的新鲜度指标保持较优水平;相较周期上报,信息冗余显著降低,同时触发决策与路径选择更为精准。
    \item \textbf{单位能量收益:}结合 EETOR 的阈值与多跳抑制,单位发送能量对应的接收增量更高,低效路径占比降低,表明路径治理对系统能效的直接贡献。
\end{itemize}
在统计层面,我们对寿命、效率、CV 等核心指标进行成对检验(Wilcoxon)与自助法置信区间估计,结果显示本文机制相对各基线在多数配置下达到统计显著且具备实用意义的效应量\cite{statTest,bootstrap}。

\subsection{Pareto Frontier:边界外移证据}
为检验“动态帕累托边界外移”的主张,我们构造多组二维目标对,例如(效率,均衡/公平)、(寿命,效率)、(新鲜度,开销)。在相同资源与约束下,本文机制所得到的前沿曲线整体外包于基线曲线之外,即在不同权衡点上均能取得非劣且在多数区间内显著优于基线的组合。该现象在(S3,\(N\ge 60\))的重负载与能量非平稳条件下更为明显,符合第四章所述“透明度增益 + 路由抑制”共同构成的充分条件直觉\cite{paretoFrontier}。

\subsection{Ablation:机制贡献剖析}
我们逐项移除机制组件以评估边际贡献。
\begin{itemize}
    \item \textbf{去除 AOEI 价格信号:}触发准确性下降,出现过度或迟滞传能,效率与寿命均受影响;表明价格化触发对时机与对象选择的关键作用。
    \item \textbf{去除 InfoNode/机会主义上报:}新鲜度降低与上报开销上升并存;触发与路径决策误差增大,说明“低开销—高新鲜”的状态获取对机制落地至关重要。
    \item \textbf{去除去重/动态等待:}信息冗余上升,通信负担加重且无对应收益,新鲜度优势消失;验证信息治理细则的必要性。
    \item \textbf{去除 EETOR 约束:}低效多跳比例显著上升,单位能量收益下降,弱势节点受损概率升高;路径外部性治理的重要性得到直接印证。
    \item \textbf{固定时长:}单位能量收益降低,欠/过供比例上升;自适应时长/额度在非平稳供需下的匹配作用得到验证。
    \item \textbf{去除公平保护:}CV 与最低能量分位明显恶化,虽在个别配置下效率略升,但综合目标显著劣化;体现公平—效率的可控权衡与内生约束价值。
\end{itemize}

\subsection{Robustness:昼/夜稳定性与参数敏感性}
面向日内供给非平稳,我们统计昼/夜分段指标。本文机制在动态预算与分位数阈值下,触发频率与预算使用率曲线更为平滑,重配频率与振荡幅度显著低于静态阈值配置。参数敏感性方面,价格函数与弱势权重在中等范围内的扰动不改变总体结论,表现为相同趋势但效应量有差异;极端扰动下,EETOR 的阈值与最大跳数主导系统稳定性,建议在实际部署中优先稳固路径治理参数。

\subsection{Case Analysis:机制协同的微观证据}
在 S3 场景的典型时段中,记录低能节点在 AOEI 升高后获得优先传能的全过程:价格信号触发→EETOR 选择高效路径→时长自适应匹配边际收益→路径搭载补全邻域 InfoNode 状态→后续时段触发频率下降、能量轨迹回升。该个案显示出“价格—路径—强度—信息”闭环带来的自抑制与自稳定效应,与总体统计结论一致。

\subsection{Threats to Validity:威胁与缓解}
(i)\emph{模拟—现实差距}:能量与链路模型的简化可能影响外推性,故在多种拓扑与规模下重复,并通过参数敏感性与极端情景测试缓解;(ii)\emph{学习基线稳定性}:深度学习基线对超参与初始化敏感,我们提供训练/测试协议与多次重复结果;(iii)\emph{指标选择偏差}:采用多指标、多视角的 Pareto 分析,避免单一指标结论偏差;(iv)\emph{实现细节泄漏}:本文在主体中仅讨论机制与指标,具体实现与参数置于复现清单,避免干扰机制客观比较。
\section{Discussion}
本节从可解释性、互补性、复杂度与可扩展性、推广边界与伦理公平等方面讨论本文机制的意义与限制。

\subsection{可解释性与制度性价值}
与“黑箱”式策略学习相比,以 AOEI 为核心的价格信号与 InfoNode 数字孪生使触发时机、对象选择与强度决策具备明确的经济学语义,并通过去重/动态等待与 EETOR 路径治理形成“规则—行为—结果”的可追溯链条。这种制度化表达为跨场景迁移与运维策略审计提供依据,有助于在面向合规的工业/城市场景中落地\cite{marketMechanismWSN,digitalTwinSurvey}。

\subsection{与优化/学习方法的互补关系}
本文并不排斥 Lyapunov 或 DRL,而是为其提供价格化触发、路径治理与信息获取的基础设施:优化方法可在固定预算/阈值下提供稳定性保证,DRL 可在时长/额度的连续空间中学习更优匹配;二者与 AOEI—InfoNode—EETOR 的组合体现“规则优先、学习增益”的原则,有利于提升整体性能与鲁棒性\cite{lyapunovEnergy,drlSurvey}。

\subsection{复杂度、通信与计算开销}
路径治理引入的效率阈值与最大跳数限制抑制了长链搜索空间;机会主义上报与去重/动态等待降低了通信冗余;整体上,本文更强调“以规则换复杂度”,在不显著增加计算/通信成本的情况下提升可解释性与性能。尽管如此,在高密度网络与强非平稳场景中,参数调校(分位数阈值、动态预算、弱势权重)仍需结合离线校准与在线小步更新以避免振荡。

\subsection{推广边界与外推风险}
当能量传输效率曲线与通信能耗模型显著偏离本文设定(如强遮挡、非幂律损耗、硬件非理想性)时,路径治理的策略阈值需要重新标定;在信息高度稀疏或极端噪声场景下,机会主义上报的覆盖率与新鲜度权衡也会变化。针对这些情形,建议在部署前进行小规模标定与鲁棒性测试,并引入置信度触发/回退策略。

\subsection{公平与伦理考量}
弱势保护通过参数化权重实现,但其社会意义与工程后果需共同评估:在保障基本服务的同时,避免对高贡献节点形成长期抑制。公平与效率的可控权衡应由任务方与运营方共同制定边界与优先级,这也是经济学机制优于单纯工程优化的原因之一。

\section{Conclusion}
本文提出以 AOEI 为价格信号、以 InfoNode 为数字孪生账户,并结合机会主义上报、去重/动态等待与 EETOR 路径治理的一体化能量共享经济机制。在不改变物理资源总量的前提下,该机制通过价格化触发、状态可得性与路径外部性治理,协同自适应的时长/额度决策与弱势保护,在效率、均衡/公平、时效与寿命四维度上实现协同改进。大量对比、消融与 Pareto 前沿分析表明,相对典型基线,本文机制在多数配置下实现“动态帕累托边界外移”,并在非平稳供需条件与含能量空洞的困难场景中展现稳健优势。

\subsection{Future Work}
后续研究方向包括:
\begin{itemize}
    \item \textbf{多智能体分布式决策}:将价格化触发与路径治理扩展到去中心化的多智能体博弈框架,研究激励一致性与通信受限下的协调;
    \item \textbf{移动网络与动态拓扑}:在节点移动、链路短时断连与热点迁移的条件下,研究阈值/预算与时长/路径的联合自适应;
    \item \textbf{能量采集耦合与预测增强}:将更细粒度的环境预测与不确定度评估纳入 InfoNode,强化前瞻性预算配置与置信度触发;
    \item \textbf{可解释强化学习}:在自适应时长/额度与路径微调环节,研究可解释的策略学习与安全约束;
    \item \textbf{理论完善}:围绕边界外移的充分/必要条件、外部性抑制与公平下界,给出更严格的形式化证明;
    \item \textbf{工程验证}:开展原型验证与小规模实地试验,评估在不同硬件平台与能量传输技术(RF/WPT)的可落地性与改造成本。
\end{itemize}

\ifdraftbib
\begin{thebibliography}{10}
\bibitem{placeholder2025}
T.~B.~Decided, ``Placeholder Reference for Compilation,'' 2025.
\bibitem{surveyWSN}
To~Be~Updated, ``A Survey on Wireless Sensor Networks,'' 2015--2025, placeholder entry.
\bibitem{energyHarvestSurvey}
To~Be~Updated, ``Energy Harvesting in Wireless Sensor Networks: A Survey,'' 2015--2025, placeholder entry.
\bibitem{energyBalSurvey}
To~Be~Updated, ``Energy Balancing and Sharing Strategies in WSN: A Review,'' 2015--2025, placeholder entry.
\bibitem{lyapunovEnergy}
To~Be~Updated, ``Lyapunov Optimization for Energy Management in Networks,'' 2010--2025, placeholder entry.
\bibitem{drlSurvey}
To~Be~Updated, ``Deep Reinforcement Learning for Networked Systems: A Survey,'' 2016--2025, placeholder entry.
\bibitem{aoiSurvey}
To~Be~Updated, ``Age of Information: A Survey and Applications,'' 2017--2025, placeholder entry.
\bibitem{digitalTwinSurvey}
To~Be~Updated, ``Digital Twin for CPS/IoT: Concepts and Applications,'' 2018--2025, placeholder entry.
\bibitem{marketMechanismWSN}
To~Be~Updated, ``Market-based Resource Allocation in Wireless Networks,'' 2008--2025, placeholder entry.
\bibitem{eetor}
To~Be~Updated, ``Energy-Efficient Transfer Opportunistic Routing (EETOR),'' 2019--2025, placeholder entry.
\bibitem{opportunisticInfo}
To~Be~Updated, ``Opportunistic Information Reporting in Networked Systems,'' 2012--2025, placeholder entry.
\bibitem{fairnessAlloc}
To~Be~Updated, ``Fairness-aware Resource Allocation in Wireless Networks,'' 2005--2025, placeholder entry.
\bibitem{paretoFrontier}
To~Be~Updated, ``Pareto Optimality and Frontier Expansion in Networked Systems,'' 2000--2025, placeholder entry.
\bibitem{statTest}
To~Be~Updated, ``Statistical Tests for Paired Comparisons in Networking Experiments,'' 2000--2025, placeholder entry.
\bibitem{bootstrap}
To~Be~Updated, ``Bootstrap Methods for Confidence Intervals,'' 1990--2025, placeholder entry.
\end{thebibliography}
\else
\nocite{placeholder2025}
\bibliographystyle{IEEEtran}
\bibliography{references}
\fi

\end{CJK}
\end{document}


