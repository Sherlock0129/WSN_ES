\section{Conclusion}
本文提出以 AOEI 为价格信号、以 InfoNode 为数字孪生账户,并结合机会主义上报、去重/动态等待与 EETOR 路径治理的一体化能量共享经济机制。在不改变物理资源总量的前提下,该机制通过价格化触发、状态可得性与路径外部性治理,协同自适应的时长/额度决策与弱势保护,在效率、均衡/公平、时效与寿命四维度上实现协同改进。大量对比、消融与 Pareto 前沿分析表明,相对典型基线,本文机制在多数配置下实现"动态帕累托边界外移",并在非平稳供需条件与含能量空洞的困难场景中展现稳健优势。

\subsection{Future Work}
后续研究方向包括:
\begin{itemize}
    \item \textbf{多智能体分布式决策}:将价格化触发与路径治理扩展到去中心化的多智能体博弈框架,研究激励一致性与通信受限下的协调;
    \item \textbf{移动网络与动态拓扑}:在节点移动、链路短时断连与热点迁移的条件下,研究阈值/预算与时长/路径的联合自适应;
    \item \textbf{能量采集耦合与预测增强}:将更细粒度的环境预测与不确定度评估纳入 InfoNode,强化前瞻性预算配置与置信度触发;
    \item \textbf{可解释强化学习}:在自适应时长/额度与路径微调环节,研究可解释的策略学习与安全约束;
    \item \textbf{理论完善}:围绕边界外移的充分/必要条件、外部性抑制与公平下界,给出更严格的形式化证明;
    \item \textbf{工程验证}:开展原型验证与小规模实地试验,评估在不同硬件平台与能量传输技术(RF/WPT)的可落地性与改造成本。
\end{itemize}

