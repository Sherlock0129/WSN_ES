\section{Conclusion}
\label{sec:conclusion}

本文围绕无线传感器网络(WSN)在能量受限、信息滞后与路径外部性等结构性约束下难以突破静态帕累托边界的核心问题,提出了一套从制度机制视角重构能量共享网络的新框架。不同于传统主要依赖算法优化的思路,本研究从“价格化—透明化—治理化—自适应化”四重机制出发,将能量与信息两类资源的价值、供需与风险进行统一建模,实现了调度体系的可解释性、跨期性与制度化改造。在此基础上,本文在理论与实验两方面共同证实了多目标性能的同步改善和可行解空间的外延扩张。

从建模层面,本文系统揭示了传统能量共享 WSN 的三项根源性约束:一是信息价值未被定价导致触发错配;二是状态获取滞后导致调度在过时状态下作出决策;三是多跳能量传输的累积损耗与弱势节点过度暴露造成显著路径外部性。围绕这些机制缺口,本文构建了由节点能量动态、链路效率、可再生供给、数字孪生状态缓存以及能量信息年龄(AOEI)衰减等组成的统一系统模型,为后续制度创新提供了严格的语义空间与约束基础。

在机制设计方面,本文提出了四项核心制度模块:AoEI 内生化价格信号用于驱动传能触发与排序;InfoNode 数字孪生账户构建低开销高时效的状态透明层;EETOR 专用路由机制以效率阈值与跳数限制治理路径外部性;跨期自适应传输时长通过综合能量收益、损耗惩罚、时效损失与信息奖励等多维因素实现动态决策。此外,\emph{AdaptiveLyapunovScheduler} 和 \emph{AdaptiveDurationAwareLyapunovScheduler} 进一步引入了基于网络反馈分数的参数自适应调整机制,通过评估每次调度对网络整体状态的影响(能量均衡性、存活率、传输效率和整体能量水平),动态调整 Lyapunov 权重参数 \(V\),使系统能够在不同场景下自动优化权衡策略,进一步提升在非平稳环境下的鲁棒性和性能表现。上述机制之间并非孤立,而是构成了一个紧密耦合的“制度网络”,在触发、上报、路径构建和跨期分配四条链路上分别抑制结构性低效,从根本上扩展系统可行解空间,推动静态帕累托边界向动态帕累托边界外移。

为验证机制的有效性,本文设计了包括智能被动 vs 固定周期(E1)、信息价值剖析(E2)、上报制度对比(E3)以及自适应时长调度(E4)在内的四类对照实验,覆盖不同拓扑形式、不同初始能量分布以及强日内非平稳环境,并采用 Bootstrap 置信区间、Wilcoxon 显著性检验和帕累托前沿可视化进行系统评估。实验结果显示:(i)AoEI 与数字孪生显著提升触发准确性、降低振荡,并在效率—公平空间整体推动帕累托前沿外移;(ii)EETOR 有效压制低效多跳路径,使单位能量产出提升超过 1.5 倍;(iii)弱势节点保护在保持效率损失小于 5\% 的条件下,使能量均衡度(CV)改善超过 25\%,并将昼夜周期中的能量振荡控制在 0.1 以下;(iv)自适应调度与信息治理形成互补,使系统在多指标维度呈现结构性占优解;(v)基于网络反馈分数的参数自适应调整机制进一步提升了系统在非平稳环境下的鲁棒性,能量振荡幅度降低 \(15\%-25\%\),同时保持参数调整的稳定性。上述证据表明,通过制度化处理信息价值、路径外部性与跨期优化,并在调度层面引入反馈驱动的自适应机制,可在不增加能量预算的前提下,实现多目标的正和式提升。

综上所述,本文通过 AoEI 价格信号、数字孪生信息治理、EETOR 路径治理与非平稳自适应调度等机制,在理论与实验上验证了“边界外移—路径治理—公平鲁棒”三大方向的联合可行性,为构建可解释、可扩展且具经济理性的能量共享 WSN 提供了可行范式。更为重要的是,本研究展示了机制设计在资源受限网络中的普适性价值:通过改变制度结构,而非仅优化算法细节,可以推动系统从零和博弈转向正和博弈,从既定边界走向动态扩展的可行解空间,为未来的能源自治网络、可再生驱动的边缘系统以及大规模异质感知网络提供了统一的理论基础与实践方向。

