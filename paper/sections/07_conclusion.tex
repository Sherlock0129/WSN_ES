\section{Conclusion}

本文从制度经济学视角重新审视能量共享无线传感器网络的性能瓶颈,揭示了传统方法被静态帕累托边界束缚的根本原因:\textbf{制度缺位导致可行解空间收缩}。通过提出 AOEI 定价、InfoNode 数字孪生、EETOR 路径治理与前瞻性时长规划四项机制创新,本文构建了全新的能量共享经济机制设计框架,实现了从"约束内优化"到"约束重构"的范式转变。

\subsection{主要贡献与创新}

\textbf{(1)理论贡献}:从机制设计角度系统性揭示了传统能量共享 WSN 被静态帕累托边界束缚的根本原因,说明信息滞后、路径外部性与静态规则共同导致可行解空间收缩。通过将能量共享建模为合作博弈,在 AOEI 价格信号与 InfoNode 透明化机制下实现激励相容的准纳什均衡,使得节点如实报告状态、积极参与共享成为其最优策略,从而通过博弈过程的自发协调机制促进帕累托边界外移。

\textbf{(2)机制创新}:提出四项制度创新,构建了"价格信号—数字账户—交易规则—路径治理"的一体化框架。AOEI 作为内生价格信号,将信息新鲜度与能量紧迫度统一为可度量的优先级指标;InfoNode 数字孪生通过三级缓存架构实现低通信开销下的高保真状态同步;EETOR 路径治理通过效率阈值与跳数限制抑制多跳路径外部性;前瞻性时长规划通过跨期资源匹配提升系统稳定性。

\textbf{(3)实验验证}:通过大规模实验(10 次独立随机种子、95\% Bootstrap 置信区间、配对 Wilcoxon 检验)验证了机制的有效性。实验结果表明,在相同资源约束下,提案机制在效率-公平、寿命-效率、信息新鲜度-通信开销三对目标上的帕累托前沿在多数配置下外包最优基线,超体积(HV)显著更大(\(p<0.05\)),验证了"动态帕累托边界外移"的理论主张。

\subsection{核心实验结果总结}

\textbf{触发策略(E1)}:价格信号驱动的智能被动触发将传能次数从 1543 次减少至 504 次(降幅 67.3\%),累计发送能量从 3857.50 kJ 降至 1260.00 kJ,同时保证 30/30 节点存活(vs. 固定周期的 19/30)。首死时间提升 18.3\%(IQR \(\approx 6.1\%\)),能量空洞场景最高达 27.6\%,验证了价格化触发以"低频高质"方式延长网络寿命的有效性。

\textbf{信息层治理(E2)}:去除信息奖励后,路径收集次数从 504 次激增至 2299 次(增加 4.6 倍),通信能耗从 21.01 kJ 升至 97.84 kJ(增加 4.7 倍),反馈均值从 3.80 降至 0.72,传输效率从 51.90\% 降至 50.08\%,验证了信息价值驱动在抑制冗余上报、维持调度精度中的关键作用。

\textbf{通信上报制度(E3)}:机会主义上报在反馈评分(3.80)和通信成本(21.01 kJ)之间取得最佳平衡,优于 ADCR 点对点上报(反馈 2.76,通信 4.86 kJ)和周期直发上报(反馈 3.71,通信 21.93 kJ),验证了价值驱动的机会主义上报在控制通信预算的同时维持全网满意度的优势。

\textbf{路径治理与自适应调度(E4)}:自适应调度器的反馈分数收敛至稳定正值(3.80 vs. 传统 Lyapunov 的 0.72),在相同寿命目标下仅需 504 次传能(vs. 传统 Lyapunov 的 2299 次),通信预算保持在 21.01 kJ 以内(vs. 传统 Lyapunov 的 97.84 kJ),验证了路径外部性治理与跨期匹配的协同效应。

\textbf{总体效益}:在效率-公平(\(\eta_E\)-CV)、寿命-效率(\(T_{\text{death}}\)-\(\eta_E\))、信息新鲜度-通信开销(\(\overline{\text{AOEI}}\)-\(E_{\text{com}}\))三对目标上,提案机制的帕累托前沿在多数配置下外包最优基线,超体积(HV)显著更大(配对 Wilcoxon,\(p<0.05\)),验证了"价格触发 + 信息透明 + 路径治理 + 自适应时长"在既定能量/通信预算下同时改善效率、公平、时效与寿命的综合效益。

\subsection{机制协同与可解释性}

实验结果表明,四项机制通过协同效应实现"动态帕累托边界外移"。价格化触发(M1)与路径治理(M3)的协同使系统在减少传能次数的同时维持节点存活;InfoNode 缓存(M2)与价值加权上报的协同使通信能耗控制在合理范围;路径治理(M3)与自适应时长(M4)的协同使低效路径占比下降,反馈分数收敛。这种协同效应验证了机制设计框架的系统性价值。

与"黑箱"式策略学习相比,以 AOEI 为核心的价格信号与 InfoNode 数字孪生使触发时机、对象选择与强度决策具备明确的经济学语义,并通过去重/动态等待与 EETOR 路径治理形成"规则—行为—结果"的可追溯链条。这种制度化表达为跨场景迁移与运维策略审计提供依据,有助于在面向合规的工业/城市场景中落地。

\subsection{系统开销与适用边界}

量化分析表明,本文以规则换复杂度(计算增加 \(<3\times\)),在通信与存储成本增加 \(<10\%\) 的前提下,提升可解释性与性能。单次传能决策耗时 \(12.3 \pm 1.8\) ms,InfoNode 三级缓存总占用 2.4 MB,机会主义上报使通信能耗降低 78.5\%(从 97.84 kJ 降至 21.01 kJ),均满足实时约束与资源限制。适用边界分析显示,在效率模型偏差、信息稀疏场景、极端噪声等条件下,机制仍能保持优势,但需进行参数重新标定与置信度门控。

\subsection{Future Work}

后续研究方向包括:

\begin{itemize}
    \item \textbf{多智能体分布式决策}:将价格化触发与路径治理扩展到去中心化的多智能体博弈框架,研究激励一致性与通信受限下的协调机制,探索分布式共识算法与激励机制的结合。
    
    \item \textbf{移动网络与动态拓扑}:在节点移动、链路短时断连与热点迁移的条件下,研究阈值/预算与时长/路径的联合自适应,开发拓扑变化感知的机制调整策略。
    
    \item \textbf{能量采集耦合与预测增强}:将更细粒度的环境预测与不确定度评估纳入 InfoNode,强化前瞻性预算配置与置信度触发,研究多源能量采集(太阳能、风能、振动能)的联合建模与优化。
    
    \item \textbf{可解释强化学习}:在自适应时长/额度与路径微调环节,研究可解释的策略学习与安全约束,探索规则优先、学习增益的混合范式,开发可解释性指标与约束满足机制。
    
    \item \textbf{理论完善}:围绕边界外移的充分/必要条件、外部性抑制与公平下界,给出更严格的形式化证明,研究机制设计的收敛性与稳定性理论,探索帕累托边界外移的量化度量方法。
    
    \item \textbf{工程验证}:开展原型验证与小规模实地试验,评估在不同硬件平台与能量传输技术(RF/WPT)的可落地性与改造成本,研究机制在不同应用场景(环境监测、智慧城市、工业物联网)的适配性。
\end{itemize}

\subsection{总结}

本文提出的能量共享经济机制通过制度创新实现了从"约束内优化"到"约束重构"的范式转变,在相同资源约束下实现了多个性能指标的同时提升,验证了"动态帕累托边界外移"的理论主张。机制设计的可解释性、系统开销的可控性以及适用边界的明确性,为能量共享 WSN 的工程实践提供了新的理论指导与技术路径。未来工作将进一步完善理论体系、扩展应用场景、提升工程可落地性,推动能量共享 WSN 从理论研究走向实际应用。
