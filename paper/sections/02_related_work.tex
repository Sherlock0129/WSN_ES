\section{Related Work}

在无线传感器网络(WSN)的能量共享场景中,节点间的能量传输可视为一个"原始的微型市场",其核心特征表现为缺乏价格信号、信息透明度不足、激励机制缺失以及制度规则不明确\cite{PALADIN20211595}。在传统决策规则下,节点的信息价值未被量化为优先级信号(未定价),节点状态信息往往滞后不可得,而多跳能量传输的路径负外部性亦未得到有效治理\cite{8006703},供能强度与时长则以静态方式设定。这种"无价格、无透明、无激励、无规则"的制度缺位局面\cite{8239591},直接限制了网络协作的改进空间与帕累托前沿的外移。

从制度演化的视角审视,节点能量共享的研究进程可类比为一条制度演化的路径,大致经历以下五个阶段:

\textbf{阶段一:无规则盲目交换阶段}。在这一最初阶段,网络中的节点几乎缺乏任何统一规则来指导能量共享,每个节点按照本地经验或随机策略发出能量请求或传输。能量的流动缺乏任何市场化信号,节点间的交换呈现完全盲目状态,效率往往难以保障。该阶段的研究多停留在概念验证层面,缺乏系统性的理论分析与性能评估。

\textbf{阶段二:经验性局部优化阶段}。随后,研究者提出了一系列局部优化算法,试图改善网络性能。此类方法以降低节点能量不均衡、减少传输损耗或提高整体能量利用率为核心目标\cite{KHAN2015159,RAULT2014104},通常建立在确定性或准静态约束假设下,通过工程手段优化特定指标。然而,此类方法将信息时效性与价值视为外部给定条件,仅在固定假设下获得局部帕累托最优,难以适应动态环境变化与异质节点需求。

\textbf{阶段三:集中式调度阶段}。紧接着出现了基于Lyapunov优化或凸优化的集中式调度框架\cite{KHAN2015159,RAULT2014104},这些方法从全局视角出发,在理论上可以保证调度策略的收敛性与稳定性,具有较强的分析可解释性。然而,此类方法的能量触发时机与强度往往依赖预先设置的权衡参数(如虚拟队列权重)\cite{Zhu2020},对非平稳环境与异质场景的自适应能力受限,难以应对网络环境的突变与异构性。

\textbf{阶段四:黑箱式学习阶段}。近年来,深度强化学习(如DQN、DDPG)等黑箱式方法被引入能量调度领域\cite{9295428}。这类方法可在高维、非线性场景下学习近似最优的调度策略,可端到端地近似最优策略。然而,其内部决策过程缺乏可解释性,且对节点状态信息的可得性与时效性高度依赖\cite{9205292},在信息滞后或状态不可观测的场景下性能显著下降。

\textbf{阶段五:初代市场机制试验阶段}。最近,一些研究开始探索分散式与市场化的能量分配方式,涉及市场化资源分配、AOEI度量或数字孪生在网络管理中的局部应用\cite{Li2016GreenWSN}。然而,这些探索多停留在局部应用层面,尚未形成"价格信号—账户—交易规则—路径治理"的制度化一体框架\cite{RAULT2014104},亦未将其系统性地用于扩展可达解集。结果是,各类探索性方法依然局限于特定场景的局部优化,难以在系统层面实现跨多目标的帕累托改进。截至目前,已有工作的大多数停留于算法或度量层面:AOEI在多数文献中作为性能指标而非内生价格信号\cite{ABBAS2023199};数字孪生在相当多的应用中用于监测/仿真而未与资源分配闭环\cite{10767586};路由算法在绝大多数研究中遵循通用通信准则而缺少针对能量传输效率阈值\(\eta_{\text{th}}\)与多跳外部性的制度化治理。

尽管上述各阶段的方法实现了不同程度的性能提升,但它们均缺乏健全的制度基础,导致典型的"市场失灵"现象。具体而言,可识别出四方面关键制度缺口:

\textbf{(i)价格信号缺位}——"信息新鲜度—价值—紧急性"缺乏统一的经济学刻画与决策映射建模,触发传能的时机与强度难以与系统目标同构。信息价值未被量化为优先级信号,无法通过价格机制内生节点支付意愿,导致资源分配缺乏有效的引导机制。

\textbf{(ii)信息基础设施薄弱}——支撑能量共享决策闭环的状态感知、时间同步、AOEI度量与流处理、节点上报协议等基础机制存在系统性缺陷。在所查阅的若干代表性工作中,仍以固定周期上报为主,典型采样周期为数分钟至数十分钟,导致 AoI 较大,决策严重滞后\cite{Shi2024Optimize};缺乏事件驱动/差分上报机制,无法在关键状态变化时及时响应\cite{Shi2024Optimize};缺少跨层指标对齐与一致性缓存,使得调度层、路由层与信息层之间的状态不一致,诱发错误决策;信息上报与能量传输解耦,无法利用传能路径"搭载"状态更新,通信开销居高不下\cite{e24050596}。上述薄弱环节共同导致系统在时效性与通信开销之间难以实现帕累托改进,信息滞后成为制约能量共享效率提升的关键瓶颈。信息不对称进一步带来逆向选择问题——部分节点可能隐瞒真实剩余能量以图私利,损害了资源分配的公平性与效率。

\textbf{(iii)外部性治理缺失}——在多数路由算法中沿用通用通信路由准则(以时延或吞吐为目标),未体现能量传输的效率阈值\(\eta_{\text{th}}\)与多跳累积损耗的负外部性\cite{s120607350},亦缺少面向系统层的宏观调控机制(可类比"监管—激励—约束"的政策组合),难以跨周期维持一致性规则。多跳传输造成的负外部性无法被抵消\cite{s120607350},导致网络能量利用效率下降,系统只能停留在既定的静态帕累托前沿之内。

\textbf{(iv)激励相容机制缺位}——现有研究未将能量共享建模为合作博弈,缺乏对节点策略选择与激励相容性的系统分析。传统方法中,节点参与共享的动机与收益结构不明确,无法通过价格信号内生化支付意愿,难以实现个体理性、预算平衡与激励相容的均衡状态。缺乏博弈论视角的机制设计,使得系统无法通过激励相容的交易规则引导节点自发选择有利于全局帕累托改进的策略,从而限制了可行解空间的外延扩张。

基于制度经济学的分析逻辑,必须通过制度创新来突破这些僵局。因此,本文的设计出发点是构建一个结合"价格形成 + 信息透明 + 激励相容 + 路由治理"的一体化制度框架。针对上述缺口,本文在统一框架下构建的要素映射见表~\ref{tab:mechanism_mapping},形成"低开销—高时效—强可解释"的一体化机制,并以"帕累托边界外移"作为统一评估主张。特别地,本文通过将能量共享建模为合作博弈,在AOEI价格信号与InfoNode透明化机制下实现激励相容的准纳什均衡,使得节点如实报告状态、积极参与共享成为其最优策略,从而通过博弈过程的自发协调机制促进帕累托边界外移。与已有工作不同,本文建立"机制—行为—结果"的完整可解释链条,从制度设计层面系统性地揭示机制创新如何驱动系统行为变化并最终实现性能提升,以期为WSN的能量共享提供一种可解释、高效且公平的新范式。

\begin{table}[t]
\centering
\caption{机制要素与实现载体的对应关系}
\label{tab:mechanism_mapping}
\begin{tabular}{p{0.36\linewidth} p{0.56\linewidth}}
\toprule
机制要素 & 实现载体/功能 \\
\midrule
信息价值信号 & AOEI(内生化价格)、信息量 \\
账户与状态 & InfoNode(数字账户) \\
上报与同步 & 机会主义上报(交易/同步规则) \\
路径治理 & EETOR(效率阈值、最大跳数、外部性抑制) \\
公平与鲁棒 & 弱势权重、动态预算 \\
\bottomrule
\end{tabular}
\end{table}

为确保比较的代表性与可复现性,本文选取"无共享、Lyapunov、DurationAware、DQN、DDPG"五类代表性基线(覆盖优化、学习与混合三类范式)进行对标,不展开冗长综述。