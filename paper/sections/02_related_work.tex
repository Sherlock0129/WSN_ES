\section{Related Work}
现有研究可按核心目标与方法论分为四条主线。\textbf{第一类},面向效率或方差的能量共享优化方法,以降低能量不均衡、减少传输损耗或提高能量利用率为核心目标\cite{energyBalSurvey};此类方法建立在确定性或准静态约束集合之上,强调工程可实现性,但将信息时效性与价值视为外部给定条件。\textbf{第二类},以Lyapunov或凸优化为代表的均衡框架,在理论上给出收敛性与稳定性保证\cite{lyapunovEnergy},具有较强的分析可解释性;然而,此类方法的触发与强度决策依赖预先设定的权衡参数(如虚拟队列权重),对非平稳环境与异质场景的自适应能力受限。\textbf{第三类},基于深度强化学习(如DQN、DDPG)的自适应调度,在高维与非线性场景中显示出策略学习优势\cite{drlSurvey},可端到端地近似最优策略;但其政策可解释性相对不足,且对状态可得性与信息时效性依赖显著\cite{aoiSurvey}。\textbf{第四类},分散的探索性研究涉及市场化资源分配、AOEI度量或数字孪生在网络管理中的局部应用\cite{marketMechanismWSN,digitalTwinSurvey},但尚未形成"价格信号—账户—交易规则—路径治理"的制度化一体框架,亦未将其系统性地用于扩展可达解集,导致各探索性方法仍局限于特定场景的局部优化,难以实现系统层面的帕累托改进。综上所述,已有工作的大多数停留于算法或度量层面:AOEI在多数文献中作为性能指标而非内生价格信号\cite{aoiSurvey};数字孪生在相当多的应用中用于监测/仿真而未与资源分配闭环\cite{digitalTwinSurvey};路由算法在绝大多数研究中遵循通用通信准则而缺少针对能量传输效率阈值与多跳外部性的制度化治理。

综上分析,识别出三方面关键缺口:\textbf{(i)机制层缺位}——"信息新鲜度—价值—紧急性"缺乏统一的经济学刻画与决策映射建模,触发传能的时机与强度难以与系统目标同构;\textbf{(ii)信息基础设施薄弱}——支撑能量共享决策闭环的状态感知、时间同步、AOEI度量与流处理、节点上报协议等基础机制存在系统性缺陷。在我们调研的范围内——覆盖2018–2025年WSN/IoT能量共享相关研究——多数方法仍以静态或固定周期上报为主(典型间隔\(\ge 30\)分钟),导致信息年龄常超过60分钟,决策严重滞后;缺乏事件驱动/差分上报机制,无法在关键状态变化时及时响应;缺少跨层指标对齐与一致性缓存,使得调度层、路由层与信息层之间的状态不一致,诱发错误决策;信息上报与能量传输解耦,无法利用传能路径"搭载"状态更新,通信开销居高不下。上述薄弱环节共同导致系统在时效性与通信开销之间难以实现帕累托改进,信息滞后成为制约能量共享效率提升的关键瓶颈;\textbf{(iii)外部性治理缺失}——在多数路由算法中沿用通用通信路由准则(以时延或吞吐为目标),未体现能量传输的效率阈值\(\eta_{\text{th}}\)与多跳累积损耗的负外部性,亦缺少面向系统层的宏观调控机制(可类比"监管—激励—约束"的政策组合),难以跨周期维持一致性规则。\textbf{(iv)博弈机制缺位}——现有研究未将能量共享建模为合作博弈,缺乏对节点策略选择与激励相容性的系统分析。传统方法中,节点参与共享的动机与收益结构不明确,无法通过价格信号内生化支付意愿,难以实现个体理性、预算平衡与激励相容的均衡状态;信息不对称导致节点可能隐藏真实状态以获取不当收益,削弱了资源分配的公平性与效率;缺乏博弈论视角的机制设计,使得系统无法通过激励相容的交易规则引导节点自发选择有利于全局帕累托改进的策略,从而限制了可行解空间的外延扩张。针对上述缺口,本文在统一框架下构建的要素映射见表~\ref{tab:mechanism_mapping},形成"低开销—高时效—强可解释"的一体化机制,并以"帕累托边界外移"作为统一评估主张。特别地,本文通过将能量共享建模为合作博弈,在AOEI价格信号与InfoNode透明化机制下实现激励相容的准纳什均衡,使得节点如实报告状态、积极参与共享成为其最优策略,从而通过博弈过程的自发协调机制促进帕累托边界外移。与已有工作不同,本文建立"机制—行为—结果"的完整可解释链条,从制度设计层面系统性地揭示机制创新如何驱动系统行为变化并最终实现性能提升。

\begin{table}[t]
\centering
\caption{机制要素与实现载体的对应关系}
\label{tab:mechanism_mapping}
\begin{tabular}{p{0.36\linewidth} p{0.56\linewidth}}
\toprule
机制要素 & 实现载体/功能 \\
\midrule
信息价值信号 & AOEI(内生化价格)、信息量 \\
账户与状态 & InfoNode(数字账户) \\
上报与同步 & 机会主义上报(交易/同步规则) \\
路径治理 & EETOR(效率阈值、最大跳数、外部性抑制) \\
公平与鲁棒 & 弱势权重、动态预算 \\
\bottomrule
\end{tabular}
\end{table}

为确保比较的代表性与可复现性,本文选取"无共享、Lyapunov、DurationAware、DQN、DDPG"五类代表性基线(覆盖优化、学习与混合三类范式)进行对标,不展开冗长综述。

