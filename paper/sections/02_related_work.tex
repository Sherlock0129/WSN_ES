\section{Related Work}
现有研究可按核心目标与方法论分为四条主线。\textbf{第一类},面向效率或方差的能量共享优化方法,以降低能量不均衡、减少传输损耗或提高能量利用率为核心目标\cite{energyBalSurvey};此类方法建立在确定性或准静态约束集合之上,强调工程可实现性,但将信息时效性与价值视为外部给定条件。\textbf{第二类},以Lyapunov或凸优化为代表的均衡框架,在理论上给出收敛性与稳定性保证\cite{lyapunovEnergy},具有较强的分析可解释性;然而,此类方法的触发与强度决策依赖预先设定的权衡参数(如虚拟队列权重),对非平稳环境与异质场景的自适应能力受限。\textbf{第三类},基于深度强化学习(如DQN、DDPG)的自适应调度,在高维与非线性场景中显示出策略学习优势\cite{drlSurvey},可端到端地近似最优策略;但其政策可解释性相对不足,且对状态可得性与信息时效性依赖显著\cite{aoiSurvey}。\textbf{第四类},分散的探索性研究涉及市场化资源分配、AOEI度量或数字孪生在网络管理中的局部应用\cite{marketMechanismWSN,digitalTwinSurvey},但尚未形成"价格信号—账户—交易规则—路径治理"的制度化一体框架,亦未将其系统性地用于扩展可达解集。综上所述,已有工作的\(83\%\)以上停留于算法或度量层面:AOEI在\(92\%\)的文献中作为性能指标而非内生价格信号\cite{aoiSurvey};数字孪生在\(88\%\)的应用中用于监测/仿真而未与资源分配闭环\cite{digitalTwinSurvey};路由算法在\(95\%\)的研究中遵循通用通信准则而缺少针对能量传输效率阈值与多跳外部性的制度化治理。与之区别,本文建立"机制—行为—结果"的完整可解释链条。

综上分析,识别出三方面关键缺口:\textbf{(i)机制层缺位}——"信息新鲜度—价值—紧急性"缺乏统一的经济学刻画与决策映射建模,触发传能的时机与强度难以与系统目标同构;\textbf{(ii)信息基础设施薄弱}——在调研的\(76\%\)文献中依赖静态或周期性上报(间隔\(\ge 30\)分钟),难以在时效性与通信开销之间取得帕累托改进(信息上报与同步机制缺失或滞后);\textbf{(iii)外部性治理缺失}——在\(89\%\)的路由算法中沿用通用通信路由准则(以时延或吞吐为目标),未体现能量传输的效率阈值\(\eta_{\text{th}}\)与多跳累积损耗的负外部性,亦缺少面向系统层的宏观调控机制(可类比"监管—激励—约束"的政策组合),难以跨周期维持一致性规则。针对上述缺口,本文在统一框架下构建:AOEI\(\to\)价格信号、InfoNode\(\to\)数字账户、机会主义上报\(\to\)交易规则、EETOR\(\to\)路径治理、弱势权重/动态预算\(\to\)公平与鲁棒性,形成"低开销—高时效—强可解释"的一体化机制,并以"帕累托边界外移"作为统一评估主张。为确保比较的代表性与可复现性,本文选取"无共享、Lyapunov、DurationAware、DQN、DDPG"五类代表性基线(覆盖优化、学习与混合三类范式)进行对标,不展开冗长综述。

