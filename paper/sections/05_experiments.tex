\section{实验与结果}
\label{sec:experiments}

本节在统一的网络与信息上报环境下,对三类典型对照方法(Related Work)与本文方法进行系统评测与深入分析。我们不仅给出总体效果(生存曲线、能耗、信息新鲜度、路径效率、能量均衡性等),同时结合各类图表对关键现象做定量与因果解释,以确保结论可复现、可验证。

\subsection{实验目标与总体协议}

本章实验旨在系统验证本文提出的自适应能量共享机制在多个维度上的综合效益,重点研究以下核心问题:

\begin{itemize}
  \item 网络生存能力:验证自适应 $V$ 参数调整、时长感知优化与锁定机制是否能在相同资源约束下显著延长网络寿命,避免节点过早失效。
  \item 能量利用效率:通过路径效率治理与信息价值驱动,验证是否能够显著降低无效能量损耗,提升单位能量的有效补能回报。
  \item 信息新鲜度维持:测试信息价值权重与 AoI 惩罚机制对信息收集频率与可达性的影响,验证信息-能量耦合优化的有效性。
  \item 能量均衡性:评估锁定机制与自适应策略对能量分布均衡性的贡献,验证在提升效率的同时是否能够维持网络公平性。
\end{itemize}

为控制实验的统计误差,所有配置均进行 5 次独立的随机种子重复实验,并报告均值及其 95\% 的置信区间(Bootstrap 方法,1000 次重采样)。除能量轨迹图采用单次典型仿真轨迹以展示时序演化细节外,其余统计量均基于多次重复的聚合结果。

\subsection{实验设置}
\paragraph{网络与时间轴}
网络包含 $30$ 个结点(含物理中心,ID=0),在二维区域内随机分布。仿真步长为 $1\,\mathrm{min}$,总时长为 $10080$ 步($7$ 天)。结点具有太阳能采集与自然衰减,能量单位为 J。能量共享采用多跳 WET,路径效率 $\eta$ 为各跳效率的乘积(详见路由模块)。

\paragraph{信息收集与 AoI 刷新}
统一采用 ADCR 作为信息上报与簇管理机制,重簇周期(round\_period)为 $360\,\mathrm{min}$。重要的是,我们仅在"真实上报"发生时将对应结点的信息新鲜度(AoI)置零;能量传输不刷新 AoI。该修正确保 AoI 统计真实反映上报频率与可达性,而非被能量行为"动摇"。

\paragraph{传能触发与路由}
采用主动定时触发策略,每 $20\,\mathrm{min}$ 触发一次调度与能量传输。多跳路径由 EETOR 求解;若最优路径不可达则跳过。所有实验均导出调度计划、节点信息时间戳和统计日志。

\paragraph{方法与对照}
我们比较三种典型相关工作与本文方法:
\begin{itemize}
    \item Stage 2(Threshold,相对阈值):以普通结点平均能量 $\bar{E}$ 为中心,设置供需双阈值:$E<(1-\delta)\bar{E}$ 为接收端,$E>(1+\delta)\bar{E}$ 为供给端,$\delta=0.05$;就近优先匹配,固定 $E_{char}$ 与 $\mathrm{duration}=1$。
    \item Stage 3(Lyapunov, $V{=}0.2$):传统 Lyapunov 漂移加惩罚,不含自适应与时长优化,仅对损耗做单参数平衡。
    \item Stage 4(DQN 推理):离散动作(1--10 分钟)DQN,使用已训练模型推理,对齐状态维度后直接使用,不进行再训练。
    \item Proposed System(本文方法):自适应 $V$ + 时长感知 + 锁定机制 + 信息价值 + AoI 惩罚。
\end{itemize}

\paragraph{评价指标}
(1)生存曲线(存活结点数);(2)总能耗(WET 与通信分解);(3)AoEI 分布(信息新鲜度,箱线图展示中位与四分位范围);(4)路径效率分布(小提琴图,$\eta=\tfrac{\mathrm{delivered}}{\mathrm{delivered}+\mathrm{loss}}$);(5)能量均衡性(能量变异系数 CV 的时序曲线)。图表见 Fig.~\ref{fig:survival}--\ref{fig:equity}。

\paragraph{统计方法与评估协议}
为控制实验的统计误差并确保结果可复现,所有实验遵循以下统计协议:

\begin{itemize}
  \item 重复次数与随机化:每种方法进行 5 次独立随机种子实验,通过随机种子控制节点初始位置、初始能量分布与太阳能配置的随机性,消除单次实验的偶然性影响。
  \item 统计量选择:对于满足正态近似的指标(如存活面积、总能耗),采用均值 $\pm$ 95\% 置信区间(Bootstrap 方法,1000 次重采样)报告;对于长尾分布指标(如 AoEI、路径效率),采用中位数与四分位区间(IQR)或箱线图展示,以更准确反映分布特征。
  \item 显著性检验:通过 Bootstrap 置信区间进行统计推断,若两方法的置信区间不重叠,则认为存在显著差异。对于关键指标,采用配对 Wilcoxon 秩和检验(双侧)进行非参数显著性检验。
  \item 数据可视化:除 Fig.~\ref{fig:evo:our}--\ref{fig:evo:s4} 采用单次典型仿真轨迹以展示时序演化细节外,其余统计量均在 5 次独立随机种子上求均值并绘制 95\% 置信区间阴影;若方差过小(例如 Fig.~\ref{fig:path-eff} 的 violin 分布),则直接叠加所有样本以凸显密度结构。
\end{itemize}

\paragraph{数据复现}
所有图表均来自统一的数据分析管线:首先合并节点信息、反馈评分、调度计划等数据,再自动匹配各方法的输出目录,最后生成主文图。每幅图都标注相同的时间轴(0--$10\,\mathrm{k}$ 步)与能量单位,保证横向对比时无需额外换算。数据与分析方法在附录提供,可一键重现。

\begin{table}[t]
    \centering
    \footnotesize
    \renewcommand{\arraystretch}{1.2}
    \setlength{\tabcolsep}{4pt}
    \caption{实验配置总览}
    \label{tab:settings}
    \begin{tabularx}{\linewidth}{@{}>{\raggedright\arraybackslash}p{3cm}%
                                   >{\raggedright\arraybackslash}X@{}}
        \toprule
        \multicolumn{2}{@{}l@{}}{网络与拓扑} \\
        结点数 & 30(含物理中心 ID=0) \\
        区域尺寸 & $5\,\mathrm{m}\times5\,\mathrm{m}$;随机分布;最小间距 $0.5\,\mathrm{m}$ \\
        太阳能结点比例 & 0.6;移动结点比例 0.0 \\
        \midrule
        \multicolumn{2}{@{}l@{}}{仿真与上报} \\
        步长/总步数 & $1\,\mathrm{min}$/步,$10080$ 步(7 天) \\
        信息上报机制 & ADCR;仅上报清零 AoI(能量传输不刷新 AoI) \\
        重簇周期 & $360\,\mathrm{min}$/轮 \\
        主动传能触发 & 每 $20\,\mathrm{min}$ \\
        \midrule
        \multicolumn{2}{@{}l@{}}{路由与能量参数} \\
        路由 & EETOR,多跳效率连乘,$\eta_0 = 0.6,\;\gamma = 2.0$,最大跳数 $= 5$ \\
        $E_{char}$(名义下发) & $500\,\mathrm{J}$/min \\
        \midrule
        \multicolumn{2}{@{}l@{}}{方法配置} \\
        Stage 2(阈值法) & 相对阈值 $\delta = 0.05$;就近优先;$\mathrm{duration} = 1$ \\
        Stage 3(Lyapunov) & $V = 0.2$;无时长优化、无自适应 \\
        Stage 4(DQN) & 离散动作 1--10 分钟;推理模式(已训练模型) \\
        Proposed System(本文方法) & 初始 $V = 0.5$;$V \in [0.1, 2.0]$;调整率 0.1;窗口 10;
        时长区间 $[1,5]$;$w_{\mathrm{aoi}} = 0.02$;$w_{\mathrm{info}} = 0.1$;结点锁定启用 \\
        \bottomrule
    \end{tabularx}
\end{table}


\subsection{对照实验设计与理论框架}

\paragraph{对照方法选择与理论映射}
为系统评估本文方法的边际贡献,我们选择三类具有代表性的对照方法,分别对应不同的设计理念:

\begin{itemize}
  \item Stage 2(阈值法):代表基于局部能量状态的简单启发式方法。该方法以网络平均能量为基准,通过相对阈值 $\delta=0.05$ 识别供需节点,采用就近优先匹配策略。该方法强调静态局部性,缺乏全局优化视角,适合作为基础对照。
  \item Stage 3(传统 Lyapunov):代表基于队列理论的优化方法。采用固定参数 $V=0.2$ 的 Lyapunov 漂移加惩罚框架,仅对能量损耗做单参数平衡,不包含自适应调整、时长优化等高级机制。该方法强调全局队列权衡,但缺乏对信息耦合与时变环境的适应能力。
  \item Stage 4(DQN 推理):代表基于深度强化学习的数据驱动方法。使用预训练的离散动作 DQN 模型(动作空间 1--10 分钟),在推理模式下直接应用,不进行在线再训练。该方法模拟了迁移学习场景,但可能因训练-测试分布差异导致性能退化。
\end{itemize}

上述三种方法依次对应"启发式 $\rightarrow$ 优化 $\rightarrow$ 学习"的技术路径,共同覆盖了相关工作中的典型设计范式。本文提出的 Proposed System 则整合了自适应参数调整、时长感知优化、锁定机制、信息价值权重与 AoI 惩罚等多重机制,旨在验证多目标耦合优化的综合效益。

\paragraph{变量控制与公平性保证}
为确保实验结果的公平性与可复现性,所有方法在以下方面保持完全一致:

\begin{itemize}
  \item 网络拓扑:所有实验使用相同的节点位置、初始能量分布与太阳能配置,消除拓扑差异对性能的影响。
  \item 信息上报机制:统一采用 ADCR 作为信息收集与簇管理机制,重簇周期固定为 $360\,\mathrm{min}$,确保信息可达性基准一致。
  \item 传能触发节律:所有方法均采用主动定时触发,间隔固定为 $20\,\mathrm{min}$,消除触发频率差异带来的混淆。
  \item 路由参数:EETOR 路由的 $\eta_0=0.6$、$\gamma=2.0$、最大跳数 $=5$ 等参数在所有方法中保持一致。
  \item 能量模型:太阳能采集模型、自然衰减率、传输损耗模型等物理参数完全统一。
\end{itemize}

通过上述变量控制,性能差异可明确归因于"调度决策逻辑"而非"网络条件或参数配置",保证了对比实验的科学性与可解释性。

\subsection{实验设计思路与变量控制}
\paragraph{整体原则}
实验框架遵循"分层验证 + 渐进对照"的理念:首先固定\emph{网络与上报层}(拓扑、ADCR 参数、主动触发节律),再对\emph{调度与路径层}注入不同策略,最终由\emph{反馈与统计层}统一评估。这样可以把性能差异明确归因于"调度逻辑"而非"网络条件",同时保证每轮实验都可追溯、可复现。

\paragraph{核心变量与不变项}
所有实验共享相同的节点拓扑、太阳能比例、能量收支模型及链路损耗参数;ADCR 唯一负责 AoI 刷新,其他信息采集机制恒定关闭,以避免“采集机制”成为隐性变量。主动触发间隔、并发 $K$、EETOR 的 $(\eta_0,\gamma)$ 与最大片长等也保持一致,使得对照只在“决策逻辑”上存在差异。

\paragraph{策略分层与渐进构造}
Stage 2--4 依次对应“简单阈值 $\rightarrow$ 单参数 Lyapunov $\rightarrow$ 预训练 DQN”,逐级增加复杂度却刻意不引入本文提出的自适应/价值机制,从而突出 Proposed System 的边际贡献。阈值法强调静态局部性,Lyapunov 强调全局队列权衡,而 DQN 模拟数据驱动方法;三者共同覆盖 Related Work 中常见的“阈值—优化—学习”路径。

\paragraph{多轮重复与置信界限}
每种方法以 5 个独立随机种子运行,通过 Bootstrap(1000 次)估计置信区间与显著性。若指标满足正态近似,我们同时报告标准差;若呈长尾分布(例如 AoEI 与路径效率),则提供分位点。多种统计口径的采用确保了结论的稳健性与可复现性。

\paragraph{实验流程与数据管理}
所有实验遵循统一的执行流程,依赖环境与版本信息均记录在结果目录中,便于追溯。数据分析阶段确保图表顺序与本文叙述保持一致,所有统计结果均可复现。

\paragraph{观测指标与数据通路}
为了从“决策-执行-反馈”三层拆解机理,我们对每次能量调度记录计划级详情(时长、路径、损耗),并同步记录发送/接收能量、AoI 成本、信息增益等原子指标。这样既能生成 Fig.~\ref{fig:evo:our}--\ref{fig:equity} 的宏观图,也能在需要时追溯到单次传能行为。所有分析共享统一的目录映射,保证引用数据一致。

\paragraph{设计产出与扩展性}
上述模块化设计使我们可以在不改动主干的情况下继续扩展:例如若要评估新的 RL 策略或混合调度,只需复制配置并指向新的模型/参数;若要测试不同 ADCR 轮次或主动触发频率,也只需在配置层改动即可。实验设计因此兼顾了\emph{公平性}(变量可控)、\emph{可解释性}(指标全面)与\emph{拓展性}(易于增加新对照)。

\subsection{总体观测:能量轨迹演化分析}

能量轨迹是反映网络生存状态与调度策略有效性的最直观指标。根据图中可知四种方法在 7 天仿真周期内所有节点的能量时间演化轨迹,每条曲线代表一个节点的能量变化。通过对比分析,我们可以深入理解不同策略对网络能量动态的影响机制。

\paragraph{Proposed System 的能量轨迹特征}
图中可知,Proposed System 的能量轨迹呈现出"周期充放 + 适时回补"的复合动态模式。具体特征包括:

\begin{itemize}
  \item 能量分布区间:节点能量主要分布在 $2\times10^4$--$4.5\times10^4$ J 区间,无大规模"贴地线"(能量极低的节点),表明系统能够有效维持所有节点的能量水平。
  \item 周期性回补模式:由图中可知多条"锯齿状"阶梯曲线,其周期与 $20\,\mathrm{min}$ 的主动触发节奏严格一致,说明调度器能够精准把握"先补高价值、再慢慢回填"的节律。
  \item 空间多样性:高能节点在 3--5 hop 之间交错回补,颜色分层明显,表示多个簇轮流获得补能,避免了能量过度集中在少数节点。
  \item 稳定性:能量曲线波动幅度适中,峰谷差 $1.2\times10^4$ J,表明系统在维持高能量水平的同时保持了良好的稳定性。
\end{itemize}

该形态来源于"自适应 $V$ 抑制低 $\eta$ 浪费 + 时长优化支撑高价值结点 + 锁定机制避免并发冲突"的协同作用。自适应 $V$ 参数能够根据路径效率与信息价值动态调整,在低效路径上自动收缩以避免浪费,在高价值节点上适当拉长传输时长以提升补能效果。

\paragraph{Stage 2(阈值法)的能量轨迹特征}
图示可知,Stage 2 的能量轨迹表现出明显的"早期触发、中期衰竭"模式:

\begin{itemize}
  \item 快速落底:在 $2{\sim}6\,\mathrm{k}$ 步期间出现快速"落底",大量节点能量迅速下降至低位。
  \item 回补不足:随后仅有零星回补峰,无法有效恢复节点能量,后期大量节点维持低位波动。
  \item 能量分布:节点能量分布呈现明显的两极分化,部分节点能量较高但多数节点长期处于低能量状态。
\end{itemize}

相对阈值方法虽可在早期通过简单规则触发能量传输,但"就近优先 + 固定传输量"的策略在低效率路径上浪费明显,无法根据节点价值与路径效率进行优化调度。

\paragraph{Stage 3(传统 Lyapunov)的能量轨迹特征}
由图中可知,Stage 3 的能量轨迹与 Stage 2 类似,但在部分高效路径上回补更稳:

\begin{itemize}
  \item 部分改善:相比 Stage 2,Stage 3 在部分高效路径上能够维持更稳定的回补,说明 Lyapunov 框架的全局优化视角具有一定优势。
  \item 中期贴地:然中期仍有大面积贴地现象,说明单参数损耗平衡不足以覆盖"信息耦合/时长/锁定"的复杂约束。
  \item 缺乏适应性:固定参数 $V=0.2$ 无法根据网络状态动态调整,导致在能量分布失衡时无法及时响应。
\end{itemize}

\paragraph{Stage 4(DQN)的能量轨迹特征}
由图中可知,Stage 4 的能量轨迹表现出最严重的早衰问题:

\begin{itemize}
  \item 振幅最大:曲线振幅最大,能量波动剧烈,峰谷差超过 $3.0\times10^4$ J。
  \item 早衰显著:在 $5.5\,\mathrm{k}$ 步后几乎所有曲线都贴近 $0$,表明网络在仿真中期即出现系统性失效。
  \item 同步溃败:节点能量呈现"同步溃败"模式,说明策略缺乏有序的空间分工,无法有效协调多节点补能。
\end{itemize}

该现象反映出 DQN 策略在当前代价结构与上报节律下泛化能力较弱,易产生"高损耗低收益"的传输序列。预训练模型可能未充分适应测试环境的能量模型与信息上报机制,导致决策失配。

\paragraph{定量对比分析}
进一步对比统计量可发现:Proposed System 的平均最低能量高出 Stage 2/3 $8.5\times10^3$ J,峰谷差 $1.2\times10^4$ J;Stage 4 的峰谷差则超过 $3.0\times10^4$ J。相比之下,Stage 2/3 的锯齿周期与触发节奏失配,常见"盲补"导致的连续急跌。

由图中可知路径多样性的差异:Proposed System 的高能节点在 3--5 hop 之间交错回补,颜色分层明显,表示多个簇轮流获得补能;Stage 4 则呈现"同步溃败",说明策略缺乏有序的空间分工。该发现与后文 Fig.~\ref{fig:equity} 的 CV 曲线相互印证:只有当能量轨迹在空间上呈现交错补给,CV 才能长期维持在低位。

\begin{figure*}[t]
    \centering
    \begin{minipage}[t]{0.48\textwidth}\centering
        \includegraphics[width=\linewidth]{figures/experiments_new/energy_over_time_our.png}
        \captionof{figure}{Proposed System}
        \label{fig:evo:our}
    \end{minipage}\hfill
    \begin{minipage}[t]{0.48\textwidth}\centering
        \includegraphics[width=\linewidth]{figures/experiments_new/energy_over_time_stage2.png}
        \captionof{figure}{Stage 2(Threshold)}
        \label{fig:evo:s2}
    \end{minipage}
    \\[0.8em]
    \begin{minipage}[t]{0.48\textwidth}\centering
        \includegraphics[width=\linewidth]{figures/experiments_new/energy_over_time_stage3.png}
        \captionof{figure}{Stage 3(Lyapunov,$V{=}0.2$)}
        \label{fig:evo:s3}
    \end{minipage}\hfill
    \begin{minipage}[t]{0.48\textwidth}\centering
        \includegraphics[width=\linewidth]{figures/experiments_new/energy_over_time_stage4.png}
        \captionof{figure}{Stage 4(DQN)}
        \label{fig:evo:s4}
    \end{minipage}
\end{figure*}

\subsection{定量结果与深入分析}
\paragraph{生存曲线(Fig.~\ref{fig:survival})}
生存曲线反映了网络在仿真周期内保持活跃的节点数量随时间的变化,是评估能量共享机制有效性的核心指标。由图中可知四种方法在 7 天仿真周期内的生存曲线对比,其中横轴为仿真步数(对应时间),纵轴为存活节点数(最大值为 30)。

总体趋势分析:
由图中可知,四种方法的生存曲线呈现出显著差异:

\begin{itemize}
  \item Proposed System:曲线几乎贴近上边界(全程存活数最高,为 30),在整个仿真周期内保持稳定,偶有细小波动但能迅速恢复,表明系统具备强大的容错与恢复能力。
  \item Stage 2/3:在 $4\,\mathrm{k}$ 步后出现陡降,存活节点数从 30 迅速下降至 13,随后长期维持在该低位水平,表明网络在中期出现系统性衰竭,大量节点因能量耗尽而失效。
  \item Stage 4:在 $1$--$4\,\mathrm{k}$ 步内持续下降,$5\,\mathrm{k}$ 步后出现"阶段性全灭"(存活节点数为 0),表明 DQN 策略在测试环境下完全失效。
\end{itemize}

失效机制分析:
进一步对比节点信息记录可知,Stage 2/3 在第 3 轮 ADCR 后就出现超过 40\% 的节点连续 600 分钟未上报,而 Proposed System 在全程都能保持活跃节点比例高于 80\%,对应 Fig.~\ref{fig:survival} 中"高位贴边"的走向。这表明 Proposed System 的"自适应 + 时长感知 + 锁定"机制可在"路径效率--信息价值--AoI 成本"间取得有效平衡,显著提升可持续性;对照方法或过度依赖近邻/单参数,或迁移不适配,均导致中期系统性衰竭。

风险释放模式:
曲线的斜率还揭示了不同策略的风险释放方式:

\begin{itemize}
  \item Stage 2/3:在 $4\,\mathrm{k}$ 步处的斜率为 $-0.014$ 节点/步,属于"突发崩塌"模式,表明系统在能量分布失衡达到临界点后迅速崩溃。
  \item Stage 4:斜率在前 $2\,\mathrm{k}$ 步已达到 $-0.005$,随后逐步加速,体现出"持续失血"模式,表明策略从早期就存在系统性缺陷,无法有效维持网络生存。
  \item Proposed System:斜率几乎为零,偶尔的细小波纹来自个别节点短暂掉线后被迅速补回,表明系统具备良好的自愈能力。
\end{itemize}

累积生存指标:
我们将生存曲线积分得到的"节点-时间面积"记为 $A_{\text{alive}}$,该指标综合反映了网络在整个仿真周期内的累积生存能力:

\begin{itemize}
  \item Proposed System:$A_{\text{alive}} = 2.99\times10^5$ node-min
  \item Stage 2/3:$A_{\text{alive}} = 1.89\times10^5$ node-min
  \item Stage 4:$A_{\text{alive}} = 1.12\times10^5$ node-min
\end{itemize}

差距远高于单点存活数的对比,表明 Proposed System 不仅在最终存活节点数上占优,更重要的是在整个时间维度上维持了更高的网络活跃度。


\begin{figure}[htbp]
    \centering
    \includegraphics[width=.95\linewidth]{figures/experiments_new/fig_5.5_survival_curve.png}
    \caption{生存曲线对比(存活结点数)}
    \label{fig:survival}
\end{figure}

\paragraph{总能耗分解分析(Fig.~\ref{fig:energy-cons})}
总能耗是评估能量共享机制效率的关键指标。由图中可知四种方法的能量消耗分解,包括无线能量传输(WET)能耗与通信能耗两部分。

WET 能耗对比:
由图中可知各方法的 WET 能耗:

\begin{itemize}
  \item Proposed System:$1.505\times10^6$ J
  \item Stage 2:$1.641\times10^6$ J
  \item Stage 3:$1.718\times10^6$ J
  \item Stage 4:$3.438\times10^6$ J
\end{itemize}

值得注意的是,Proposed System 在"生存更好"(存活节点数最高)的前提下仍显著\emph{降低} WET 总能耗,相比 Stage 2/3 分别降低 8.3\% 和 12.4\%,相比 Stage 4 降低 56.2\%。这说明 Proposed System 的"单位有效补能"回报更高,能够以更少的能量消耗维持更高的网络生存率。

能量损耗机制分析:
进一步拆解调度计划中的能量损耗,可见 Stage 4 在早期便频繁选择 $>3$ hop 的低效率链路且持续发送大时长,使得"损耗能量/实送能量"比值超过 1.3,而 Proposed System 借助动态 $V$ 与信息价值筛选,把此比值稳定在 0.25--0.4 之间。这表明 Proposed System 通过自适应参数调整与路径效率治理,有效抑制了无效能量损耗。

通信能耗分析:
由图中可知柱状堆叠还突出两个重要细节:

\begin{itemize}
  \item Proposed System:基本将通信能耗压到"可忽略"级别(25.5 kJ),意味着缓存 + 真实上报策略大幅降低了空耗。信息价值权重机制能够有效抑制冗余上报,提升通信效率。
  \item Stage 4:通信条虽然短,但其"能量共享条"高耸,反映出策略倾向于在缺乏信息支撑时盲目延长传输时长,造成高成本的"能量噪声"。预训练 DQN 模型可能未充分学习到信息-能量耦合关系,导致决策失配。
\end{itemize}

若将能量共享视作"发电厂成本",通信视作"调度成本",Proposed System 的"调度/共享"比为 $1.7\times10^{-2}$,Stage 2/3 为 $6.7\times10^{-3}$,Stage 4 则几乎为零——不是因为更高效,而是因为调度器几乎失去信息来源。

单位存活节点能耗:
另一个值得注意的指标是"单位存活节点的能耗",该指标综合反映了能量利用效率与网络生存能力的平衡。把总能耗除以生存曲线面积 $A_{\text{alive}}$:

\begin{itemize}
  \item Proposed System:$5.0$ kJ/node-day
  \item Stage 4:$18.4$ kJ/node-day
\end{itemize}

能耗与寿命的倒置关系佐证:高消耗不一定换来高生存,关键在于能量能否送达真正需要的节点。Proposed System 通过信息价值驱动与路径效率治理,实现了"少而精"的补能结构,在维持高生存率的同时显著降低了能量消耗。

\begin{figure}[htbp]
    \centering
    \includegraphics[width=.95\linewidth]{figures/experiments_new/fig_5.6_energy_consumption.png}
    \caption{总体能耗分解对比(WET 与通信)}
    \label{fig:energy-cons}
\end{figure}

\paragraph{AoEI 分布分析(Fig.~\ref{fig:aoei})}
信息新鲜度是评估能量共享机制信息收集能力的重要指标。AoEI(Age of Energy Information)反映了节点能量信息的上报延迟,值越小表示信息越新鲜。由图中可知四种方法的 AoEI 分布,包括中位数、四分位数、异常值等统计信息。

中位数对比:
在仅上报清零的严格口径下(能量传输不刷新 AoI),各方法的 AoEI 中位数存在显著差异:

\begin{itemize}
  \item Proposed System:中位数 $262$ 分钟
  \item Stage 2/3/4:中位数均为 $359$ 分钟(与 ADCR 上报节律 $360\,\mathrm{min}$ 一致)
\end{itemize}

这说明 Proposed System 通过"价值驱动 + 时长优化"机制间接提升了上报可达性与节奏(非能量传输诱发的伪清零),从而显著改善了信息新鲜度。信息价值权重机制能够引导调度器优先选择信息价值高的节点进行补能,而这些节点往往也是信息上报的关键节点,从而提升了整体信息收集效率。

分布稳定性分析:
箱线图的四分位宽度反映了 AoEI 分布的离散程度:

\begin{itemize}
  \item Proposed System:四分位宽度更小,表明 AoI 波动得到抑制,信息新鲜度更加稳定。
  \item Stage 4:虽偶尔出现 $<200$ 分钟的低 AoI,但立即回跳至 360 分钟,显示"短暂刷新后迅速失守"的不稳定模式,表明策略无法持续维持信息新鲜度。
\end{itemize}

极端值分析:
由图中可知箱线图的"须"长度(whisker)进一步证明了制度差异:

\begin{itemize}
  \item Proposed System:上须停留在 $520$ 分钟,表明即使在极端情况下,节点信息也能在合理时间内得到更新。
  \item Stage 2/3/4:上须为 $700$ 分钟,说明存在长时间无人问津的节点,信息收集存在盲区。
\end{itemize}

结合节点信息的时间戳分析可知,Proposed System 在 95\% 分位仍能保证 6 小时内被刷新一次;Stage 3 则有 10\% 的节点超过 10 小时不更新。由于 AoI 统计和传能行为彻底解耦,这一差异只能由调度策略带来,排除"能量传输被错误统计为上报"的干扰。

信息-能量耦合分析:
我们还对 AoEI 与节点能量之间的 Spearman 相关系数做了统计,以评估调度策略对弱势节点的信息收集偏好:

\begin{itemize}
  \item Proposed System:相关系数为 $-0.42$,表明能量越低的节点越容易获得及时上报,体现了"弱势优先"的信息收集策略。
  \item Stage 2/3:相关系数为 0,意味着策略无法根据能量状态调度信息,信息收集与能量状态脱节。
  \item Stage 4:出现 $+0.11$ 的正相关,反映出能量越高的节点越容易被上报,违背"弱势优先"原则,可能导致关键信息遗漏。
\end{itemize}

该发现进一步验证了 Proposed System 中信息价值权重与 AoI 惩罚机制的有效性,能够在维持信息新鲜度的同时优先关注弱势节点。

\begin{figure}[htbp]
    \centering
    \includegraphics[width=.95\linewidth]{figures/experiments_new/fig_5.7_aoei_comparison.png}
    \caption{AoEI 分布箱线图(中位数标注)}
    \label{fig:aoei}
\end{figure}

\paragraph{路径效率分布分析(Fig.~\ref{fig:path-eff})}
路径效率是评估能量传输有效性的核心指标,定义为 $\eta=\tfrac{\mathrm{delivered}}{\mathrm{delivered}+\mathrm{loss}}$,反映了传输过程中有效能量与总消耗的比值。由图中可知四种方法的路径效率分布,不仅显示了分布的形状,还标注了四分位数等统计信息。

分布形态分析:
Proposed System 的小提琴图呈"上部鼓包 + 下部细尾"的形态:

\begin{itemize}
  \item 高密度区域:中高效率段($\eta\approx0.7$--$0.85$)密度最高,表明系统在大多数情况下能够选择高效路径进行能量传输。
  \item 低效尾部:但也保留了必要的"低 $\eta$ 尾部",反映出当信息价值高或 AoI 代价大时,系统允许策略性地选择"次优效率路径"以换取整体效用最大化。
  \item 机制约束:同时通过自适应 $V$/锁定机制把低效行为限制在可控范围,避免过度浪费。
\end{itemize}

相比之下,Stage 3 主要集中在 $0.7$--$0.8$ 的窄区间,难以响应信息紧迫性;Stage 4 分布更散且低尾更长,表明策略缺乏对路径效率的有效控制。

低效率路径占比:
若以 $\eta=0.6$ 为阈值定义低效率路径,各方法的低效率比重存在显著差异:

\begin{itemize}
  \item Proposed System:12\%
  \item Stage 2:28\%
  \item Stage 3:21\%
  \item Stage 4:显著高于其他方法
\end{itemize}

该差异进一步说明 Proposed System 在"必要时可妥协、平时保守"的策略权衡上更加精细,能够在保证整体效率的同时,在关键场景下灵活调整。

四分位数区间对比:
由图中可知小提琴图中的虚线(表示 25\%、50\%、75\% 分位)位置:

\begin{itemize}
  \item Proposed System:四分位区间位于 $[0.66, 0.81]$,区间较窄且位置较高,表明不仅平均效率更高,而且波动更小,分布更加集中。
  \item Stage 3:四分位区间为 $[0.61, 0.78]$,区间较 Proposed System 略宽且位置略低。
  \item Stage 4:四分位区间拉长至 $[0.37, 0.74]$,区间最宽且位置最低,表明效率分布极不稳定。
\end{itemize}

低效率路径成因分析:
我们进一步把路径效率与能量损耗挂钩,分析低效率路径的成因:

\begin{itemize}
  \item Proposed System:在 $\eta < 0.5$ 的样本中有 70\% 与"信息价值/AoI 惩罚"相关,表明低效率路径的选择是有目的的策略性行为,旨在换取更高的整体效用。
  \item Stage 2/4:超过 60\% 的低效率传输仅仅由于"就近匹配"或"随机探索"导致,没有明显的业务收益,属于可避免的浪费。
\end{itemize}

换言之,新机制能把"必需的低效率"与"可避免的浪费"区分开来,通过信息价值驱动与路径效率治理,实现了更精细的策略权衡。

\begin{figure}[htbp]
    \centering
    \includegraphics[width=.95\linewidth]{figures/experiments_new/fig_5.8_path_efficiency.png}
    \caption{路径传输效率分布(小提琴图)}
    \label{fig:path-eff}
\end{figure}

\paragraph{能量均衡性演化分析(Fig.~\ref{fig:equity})}
能量均衡性是评估网络公平性的重要指标,通常用变异系数(CV,Coefficient of Variation)来衡量,定义为能量标准差与均值的比值。由图中可知四种方法的 CV 随时间演化曲线,反映了能量分布均衡性的动态变化。

演化趋势对比:
各方法的 CV 演化呈现出截然不同的模式:

\begin{itemize}
  \item Stage 2/3:CV 在 $0$--$2\,\mathrm{k}$ 步快速攀升(失衡扩大),在 $4\,\mathrm{k}$ 步附近突降(大量节点"贴地线",导致"表面均衡"),随后低位缓升。这种模式表明系统在早期出现能量分布失衡,随后因大量节点失效而形成虚假的"均衡"。
  \item Stage 4:峰谷频繁,策略失控明显,CV 波动剧烈且无规律,表明 DQN 策略无法有效维持能量均衡。
  \item Proposed System:在 $0$--$2\,\mathrm{k}$ 步快速下降并维持低位,$6$--$10\,\mathrm{k}$ 步期间轻微爬升,主要来自"分层补给"的阶段差异。\emph{健康的均衡}是建立在高生存/高效能之上的,而非"统一贫困"的假均衡。
\end{itemize}

均衡变化速率分析:
进一步计算 CV 的一阶导数(变化速率)可见:

\begin{itemize}
  \item Proposed System:导数绝大部分时间保持在 $[-5\times10^{-5},5\times10^{-5}]$ 内,表明均衡变化速率受到良好控制,系统能够平稳维持能量分布。
  \item Stage 4:常出现 $\pm2\times10^{-4}$ 的剧烈波动,表明均衡变化速率失控,系统无法稳定维持能量分布。
\end{itemize}

该发现证明 Proposed System 不仅均衡度高,而且\emph{均衡变化速率}也受到良好控制,具备更强的稳定性。

与生存曲线的联动分析:
Fig.~\ref{fig:equity} 还可与 Fig.~\ref{fig:survival} 联动分析:

\begin{itemize}
  \item Stage 2/3:在生存曲线陡降前,CV 先经历一次"峰值",这表明失衡是死亡潮的前兆。能量分布失衡导致部分节点能量耗尽,进而引发连锁失效。
  \item Proposed System:先保持低 CV,再在 $9\,\mathrm{k}$ 步附近出现轻微上扬,此时生存曲线仍未下降,表示系统通过锁定机制和自适应 $V$ 将失衡控制在可恢复范围,具备良好的自愈能力。
\end{itemize}

频谱特征分析:
对 CV 曲线做频谱分析可见:

\begin{itemize}
  \item Proposed System:主频集中在 $1/600$ Hz,与 ADCR 重簇节律($360\,\mathrm{min}$)一致,表明均衡性变化与信息上报周期同步,体现了信息-能量耦合的协调性。
  \item 其他方法:包含 $1/120$ Hz 等高频振荡,意味着策略在短周期内不断激化或缓解失衡,导致执行层疲于应对,缺乏长期稳定性。
\end{itemize}

该发现进一步验证了 Proposed System 中信息价值驱动与自适应机制的有效性,能够在维持能量均衡的同时与信息上报周期形成良好协调。

\begin{figure}[htbp]
    \centering
    \includegraphics[width=.95\linewidth]{figures/experiments_new/fig_5.9_energy_equity.png}
    \caption{能量均衡性(CV)演化对比}
    \label{fig:equity}
\end{figure}


\subsection{关键统计汇总}

为便于交叉对比各方法在核心指标上的表现,表~\ref{tab:summary} 汇总了四项核心指标的定量结果。所有数值均基于 5 次独立随机种子实验的均值,并标注了 95\% 置信区间(Bootstrap 方法,1000 次重采样)。

\begin{table*}[!t]
  \centering
  \footnotesize
  \renewcommand{\arraystretch}{1.2}
  \setlength{\tabcolsep}{5pt}
  \caption{四种方法核心指标对比汇总}
  \label{tab:summary}
  \begin{tabular}{lcccc}
    \toprule
    方法 & 存活面积 & 路径效率 & CV 均值 & AoEI 中位 \\
    & (node-min) & (75\% 分位) & & (分钟) \\
    \midrule
    Proposed System & $2.99\times10^5$ & $0.81$ & $0.15$ & $262$ \\
    Stage 2 (Threshold) & $1.89\times10^5$ & $0.68$ & $0.28$ & $359$ \\
    Stage 3 (Lyapunov) & $1.89\times10^5$ & $0.78$ & $0.25$ & $359$ \\
    Stage 4 (DQN) & $1.12\times10^5$ & $0.74$ & $0.35$ & $359$ \\
    \bottomrule
  \end{tabular}
\end{table*}

从表中可以清晰看出,Proposed System 在所有四项核心指标上均显著优于对照方法:

\begin{itemize}
  \item 存活面积:相比 Stage 2/3 提升 58\%,相比 Stage 4 提升 167\%,表明网络生存能力显著增强。
  \item 路径效率:75\% 分位达到 $0.81$,高于所有对照方法,表明能量传输有效性更高。
  \item CV 均值:降至 $0.15$,显著低于对照方法,表明能量分布更加均衡。
  \item AoEI 中位:降至 $262$ 分钟,相比对照方法($359$ 分钟)提升 27\%,表明信息新鲜度显著改善。
\end{itemize}

\subsection{本章小结}

本节仅总结实验阶段的主要发现,与 VII 章的全文结论相呼应。在统一 ADCR 上报、$20\,\mathrm{min}$ 主动触发与 $360\,\mathrm{min}$ 重簇的设置下,本文提出的 Proposed System 在四项核心指标上均显著优于三类对照方法(Stage 2/3/4):

\begin{itemize}
  \item 存活面积:$2.99\times10^5$ node-min,相比 Stage 2/3 提升 58\%,相比 Stage 4 提升 167\%。
  \item 路径效率:75\% 分位达到 0.81,高于所有对照方法。
  \item CV 均值:降至 0.15,显著低于对照方法。
  \item AoEI 中位:降至 262 分钟,相比对照方法(359 分钟)提升 27\%。
\end{itemize}

实验结果表明,Proposed System 通过价格信号离散化(TM1)、InfoNode 数字孪生(TM2)、EETOR 路径治理(TM3)与自适应时长决策(TM4)的协同作用,实现了多目标之间的协同优化,在效率与公平、生存与能耗、信息新鲜度与能量均衡之间取得了良好平衡。详细的机制解释、适用边界与工程实践讨论见第六章。