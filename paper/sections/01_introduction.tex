\section{Introduction}

无线传感器网络(Wireless Sensor Networks, WSN)作为典型的边缘感知基础设施,已在环境监测、智慧城市、工业物联网与精准农业等场景中实现规模化部署(截至 2024 年全球节点规模超过 $10^9$)\cite{surveyWSN}。然而,由于节点受限于有限的电池容量($10^3$–$10^4$ J)以及能量采集的强非平稳性(日内波动系数 $>2$)\cite{energyHarvestSurvey},当前 WSN 在长期运行中普遍面临两类根本性系统瓶颈:其一是能量的空间与时间分布不均衡导致网络寿命缩短与功能退化;其二是状态信息严重滞后(信息年龄常超过 60 分钟)\cite{aoiSurvey},使得路由、调度与能量共享决策建立在过时状态之上,进而诱发 15\%–30\% 的系统能量效率损失\cite{energyHarvestSurvey}。

现有研究多采用工程优化思路,包括节能调度、链路选择、机会式路由、Lyapunov 优化以及多目标调度等方法\cite{energyBalSurvey,lyapunovEnergy}。这些方法确实能够在既定系统条件下提升能量利用效率或在特定配置下实现局部帕累托最优,但它们都存在一个共同的结构性限制:只能在既定的可行解空间内进行目标间的权衡。

在传统能量共享 WSN 中,能量预算、链路容量、信息时延、通信开销与公平性约束均作为外生且固定的制度前提存在。这些固定约束导致系统的最优性能组合被限制在一个 静态帕累托边界(Static Pareto Frontier) 之内:当节点能量、调度规则与信息获取机制不发生制度性变化时,系统只能在寿命、效率、公平性与信息时效性之间相互牺牲以换取局部改进。例如,提高弱势节点公平性往往必然导致整体效率下降;提高信息新鲜度常常又以加速能量消耗为代价;而延长网络寿命通常会牺牲部分实时性与吞吐能力。当系统机制本身不发生改变时,更多的调度算法、更复杂的启发式方法或更精细的控制律,都只能在既定边界上“移动位置”,而无法让网络整体性能实现“同时变好”。这正是传统能量共享 WSN 难以进一步突破的重要原因:其性能瓶颈不是技术层面的,而是 制度机制层面的。只有通过制度创新改变可行解空间的形状与规模,使系统允许新的能量分配路径、新的信息流形式与新的激励约束结构,才可能从根本上实现帕累托边界的外移——即在不增加总能量预算的前提下,同时提升多个性能指标\cite{paretoFrontier}。

造成这一制度性瓶颈的更深层原因,并非算法复杂度或计算资源不足等技术层面的问题,而是制度性和经济机制层面的结构性缺陷。传统能量共享 WSN 在其默认的决策规则中隐含了多项制度性限制:信息价值未被量化为优先级信号(即未被"定价")\cite{marketMechanismWSN}、节点状态不可得或滞后、多跳能量传输的路径外部性未被治理、供能强度与时长以静态方式设定,以及公平性约束以外生方式施加等。上述因素共同收缩了系统可行解空间,使得传统框架难以突破静态帕累托边界。

基于此,本文提出通过AOEI定价、InfoNode数字孪生、交易规则(机会主义上报与交易博弈)与EETOR路径治理等机制设计扩展系统制度结构,从根本上推动可行解空间的外延扩张,实现帕累托边界的动态外移。图~\ref{fig:system-model}给出了本文提出的系统模型,采用五层处理架构:信息层(Information Layer)负责传感器输入、事件检测、AOEI监测与信息价值评估;虚拟节点层(Virtual Node Layer)通过InfoNode代理与多级状态缓存实现状态聚合;调度层(Scheduling Layer)基于需求定价策略进行全局调度决策;路由层(Routing Layer)与传输层(Transmission Layer)分别负责路径收集、中继决策与被动能量传输、联合传输执行。系统形成闭环反馈:传输层通过虚线箭头将信息反馈回信息层,实现状态更新;左侧可视化层从虚拟节点层获取聚合状态,实时监控帕累托前沿、网络状态与能量变化等关键指标。

\begin{figure*}[t]
    \centering
    \includegraphics[width=0.7\linewidth]{figures/system_model.png}
    \caption{系统模型示意图。信息自下而上汇聚形成虚拟节点与调度决策,能量自上而下沿路由执行传输,同时在传输链路上回传状态信息,最终在可视化层统一展示。}
    \label{fig:system-model}
\end{figure*}

本文针对能量共享 WSN 的核心限制,提出四项机制创新,遵循"价格信号—账户—交易规则—路径治理"的制度化一体框架:

\textbf{1) Age-of-Energy-Information(AOEI)定价机制:}
以"能量信息新鲜度 $\times$ 能量紧迫度"作为非中心节点的内生价格信号,动态驱动能量共享触发、排序和预算清算,突破传统效率—公平权衡。

\textbf{2) InfoNode 数字孪生机制:}
通过在每个节点构建数字孪生账户,实现低通信开销下的实时高保真状态同步,显著减少因信息滞后导致的错误调度与资源浪费\cite{digitalTwinSurvey}。

\textbf{3) 交易规则机制(机会主义上报与交易博弈):}
通过机会主义信息上报实现"传能即上报、低冗余、保新鲜"的状态更新范式;同时将能量共享建模为合作博弈,节点通过AOEI信号表达支付意愿,在价格信号与弱势保护机制下达成准纳什均衡,实现激励相容的资源分配。

\textbf{4) Energy-Efficient Transfer Opportunistic Routing (EETOR) 能量传输路径治理机制:}
通过路径效率阈值、最大跳数限制及弱势节点保护策略治理多跳路径外部性,从结构上提升能量传输效率\cite{eetor}。

本文的主要贡献如下:

1) 从机制设计角度系统性揭示了传统能量共享 WSN 被静态帕累托边界束缚的根本原因,说明信息滞后、路径外部性与静态规则共同导致可行解空间收缩;

2) 提出 AOEI定价、InfoNode数字孪生、交易规则(机会主义上报与交易博弈)与EETOR路径治理等四项制度创新,构建了"价格信号—账户—交易规则—路径治理"的一体化能量共享机制设计框架;

3) 从理论层面证明所提出机制能够扩展系统可行解空间,实现帕累托边界的动态外移,使多个目标可同时获得正和式提升;

4) 通过大规模仿真验证了机制的有效性,表明本文方法可在不增加总能量消耗的前提下同时提升网络寿命、能量效率、信息时效性与弱势节点公平性。
