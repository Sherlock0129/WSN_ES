\section{Introduction}
无线传感器网络(Wireless Sensor Networks, WSN)作为典型的边缘感知基础设施,已在环境监测、智慧城市、工业物联网与农业生产等关键场景中实现规模化部署(截至2024年全球部署量\(>10^9\)节点)\cite{surveyWSN}。然则,节点受限于电池容量(\(10^3\sim 10^4\)~J量级)与能量采集的非平稳性(日内波动系数\(>2\)),系统长期运行面临两类根本性挑战:\textbf{(i)能量约束}——空间与时间不均衡(方差/均值比\(\ge 0.5\))导致网络寿命缩短(首节点死亡时间\(<10^4\)分钟)与功能退化;\textbf{(ii)信息滞后}——状态信息时效性不足(信息年龄\(>60\)分钟)使得调度、路由与资源配置决策滞后,诱发系统性低效(能量效率下降\(15\%\sim 30\%\))\cite{energyHarvestSurvey,aoiSurvey}。已有方法的\(78\%\)在工程优化框架内对能量方差、传输损耗或吞吐延迟进行单目标或多目标权衡\cite{energyBalSurvey,lyapunovEnergy},即便在给定约束下达成局部帕累托最优,其本质仍受限于既定的静态帕累托边界,难以通过制度性与机制性设计实现可行解空间的外延扩张。

本文采取"经济学先导—技术落地"的研究范式,核心主张是:将信息新鲜度与价值显式内生化到能量共享决策之中,以"能量信息年龄(Age of Energy Information, AOEI)"充当价格信号,刻画"何时/对谁/以何强度"触发能量传输的优先级;同时以数字孪生化的信息账户(InfoNode)提升市场透明度与可观测性\cite{digitalTwinSurvey},借助机会主义上报、信息去重与动态等待实现低开销的状态更新;并辅以面向能量传输的专用路由策略与传输时长的自适应调节\cite{eetor}。在经济学层面与具体实现中,必要时可结合"信息价值"指标(由 AOEI 与信息量综合刻画)作为参考,以更精细地表达支付意愿。该机制在不改变物理资源总量的前提下,通过价格信号与制度设计的耦合\cite{marketMechanismWSN},使系统从"静态可达的折中集合"推移到"动态可达的扩展边界",在效率与公平之间获得整体改进。

为支撑上述主张,本文构建一套"价格信号—数字孪生—交易规则—路径治理"的一体化机制:
(1)AOEI 作为内生化的价格信号,与信息的新鲜度、情境价值与紧急性同频变化,驱动能量共享触发的时机与强度;同时,在实现与分析层面可结合"信息价值"(基于 AOEI 与信息量)作为参考指标;
(2)InfoNode 作为节点在信息市场中的数字孪生账户,维护多层状态(当前/历史/预测),并提供统一查询接口以提升市场透明度和可得性;
(3)机会主义信息上报结合信息去重与动态等待,实现"传能即上报、低冗余、保新鲜"的状态更新范式;
(4)能量传输专用路由(如 EETOR)以效率阈值与多跳抑制为原则,减少低效路径并保护脆弱节点;
(5)从帕累托边界视角评估机制外推效应,给出效率—公平权衡改善的证据与讨论。

本文的主要贡献如下:
\begin{itemize}
    \item 提出以 AOEI 为核心价格信号的能量共享触发机制,将信息新鲜度—价值—紧急性从外生变量转化为内生决策因子,统一于资源配置过程;必要时结合"信息价值"作为参考;
    \item 设计 InfoNode 数字孪生与机会主义信息上报的组合机制,配置信息去重与动态等待,实现高可得性、低通信开销与较强时效性的"透明市场";
    \item 融合能量传输专用路由的效率阈值与多跳抑制策略,在全局可解释的框架下减少低效能量路径并提升系统鲁棒性;
    \item 基于帕累托边界的分析视角,论证机制性改造对可行解空间的外推效应,并从效率与公平两维度展示动态边界外移。
\end{itemize}

