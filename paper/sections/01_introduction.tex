\section{Introduction}

无线传感器网络(Wireless Sensor Networks, WSN)作为典型的边缘感知基础设施,已在环境监测、智慧城市、工业物联网与精准农业等场景中实现规模化部署 \cite{ZHONG20181}。然而,由于节点受限于有限的电池容量\cite{YU2024100066}以及能量采集的强非平稳性\cite{EH2018Adu},当前 WSN 在长期运行中普遍面临两类根本性系统瓶颈:其一是能量的空间与时间分布不均衡导致网络寿命缩短与功能退化\cite{YU2024100066,s20010277};其二是状态信息严重滞后 \cite{ABBAS2023199},使得路由、调度与能量共享决策建立在过时状态之上\cite{Suryavansh2021},进而诱发相应的系统能量效率损失。

现有研究多采用工程优化思路,包括节能调度、链路选择、机会式路由、Lyapunov 优化以及多目标调度等方法\cite{YU2024100066}。这些方法确实能够在既定系统条件下提升能量利用效率或在特定配置下实现局部最优。在经济学意义上,这种局部最优即为帕累托最优\cite{roy2023optimization},即:在给定系统约束下,无法在不降低其他性能指标的前提下进一步提升任一指标的状态。在本研究中,将这一概念定义为\textbf{网络帕累托最优}(Network Pareto Optimality),以区别于通过制度创新实现的动态帕累托边界扩展。然而,这些方法都存在一个共同的结构性限制:其优化过程本质上受限于既定的制度框架,仅能在预设的可行解域内进行目标间的权衡与折衷,而无法通过制度性重构来拓展可行解域的边界\cite{s20010277, Asharioun2015}。从方法论视角看,此类方法遵循的是"约束内优化"范式\cite{Asharioun2015},其改进表现为在既定约束条件下寻求更优解\cite{Li2018cooperation},而非通过改变约束结构本身来创造新的更优解空间。

在传统能量共享 WSN 中,能量预算、链路容量、信息时延、通信开销与公平性约束均作为外生且固定的制度前提存在。这些固定约束使系统整体性能被锁定在一个\textbf{静态帕累托边界}(Static Pareto Frontier)内\cite{ELHABYAN201857},帕累托边界在经济学上是指提高某一性能指标必然以牺牲另一个指标为代价\cite{roy2023optimization}。在WSN系统中存在类帕累托边界,即在提高弱势节点公平性往往会降低整体效率;增强信息新鲜度通常会加速能量消耗;延长网络寿命常需牺牲部分实时性或吞吐能力\cite{Suryavansh2021}。在此框架下,无论设计多少新的调度算法或启发式方法,都只能在既定的类帕累托边界里“移动位置”,而无法实现多个性能指标的同时改善。这反映出一种典型的资源博弈特征:在固定制度下,系统内部呈现“准零和”的结构,使其难以突破传统性能上限。其中,弱势节点是指能量水平低于低能阈值或处于能量分布低分位的节点。这些节点由于能量储备不足,在网络运行中容易因频繁参与中继转发或能量共享而被耗尽能量,导致过早死亡并形成能量空洞。弱势节点的存在加剧了网络能量分布的不均衡性,使得能量变异系数(CV)增大,网络寿命缩短\cite{YU2024100066}。传统调度方案往往忽视弱势节点的特殊需求,或仅通过外生公平性约束进行事后补偿,无法从根本上解决能量分布失衡问题。

当系统机制本身不发生改变时,更多的调度算法、更复杂的启发式方法或更精细的控制律,都只能在既定边界里“移动位置”,而无法让网络整体性能实现同时提升。这正是传统能量共享 WSN 难以进一步突破的重要原因:其性能瓶颈不是技术层面的,而是\textbf{制度机制层面的}。传统能量共享 WSN 在其默认的决策规则中隐含了多项制度性限制:信息价值未被量化为优先级信号、节点状态不可得或滞后、多跳能量传输的路径外部性未被治理、供能强度与时长以静态方式设定,以及公平性约束以外生方式施加等。上述因素共同收缩了系统可行解空间,使得传统框架难以突破静态帕累托边界。只有通过制度创新改变可行解空间的形状与规模,使系统允许新的能量分配路径、新的信息流形式与新的激励约束结构,才可能从根本上实现帕累托边界的外移——即在不增加总能量预算的前提下,同时提升多个性能指标。

基于此,本文提出通过Age-of-Energy-Information(AOEI)定价、InfoNode数字孪生、Energy-Efficient Transfer Opportunistic Routing(EETOR)路径治理与前瞻性时长规划等机制设计扩展系统制度结构,从根本上推动可行解空间的外延扩张,实现帕累托边界的动态外移。图~\ref{fig:system-model}给出了本文提出的系统模型,采用六层处理架构:信息层(Information Layer)负责传感器输入、事件检测、AOEI监测与信息价值评估;虚拟节点层(Virtual Node Layer)通过InfoNode代理与多级状态缓存实现状态聚合;调度层(Scheduling Layer)基于需求定价策略进行全局调度决策,寻找最优方案,在追求传输效率与能量平衡间博弈;路由层(Routing Layer)与传输层(Transmission Layer)分别负责路径收集、中继决策与被动能量传输、联合传输执行;左侧可视化层从虚拟节点层获取聚合状态,实时监控帕累托前沿、网络状态与能量变化等关键指标。系统形成闭环反馈:传输层通过将节点信息反馈回信息层,实现状态更新;虚线表示节点信息流,实线表示能量调度信息流;左侧可视化层从虚拟节点层获取聚合状态,实时监控帕累托前沿、网络状态与能量变化等关键指标;右侧信息层通过信息价值对虚拟节点层进行决策。

\begin{figure*}[t]
    \centering
    \includegraphics[width=0.7\linewidth]{figures/system_model.png}
    \caption{系统模型示意图。信息自下而上汇聚形成虚拟节点与调度决策,能量自上而下沿路由执行传输,同时在传输链路上回传状态信息,最终在可视化层统一展示。}
    \label{fig:system-model}
\end{figure*}

本文针对能量共享 WSN 的核心限制,提出四项机制创新:

\textbf{1) Age-of-Energy-Information(AOEI)定价机制:}
以“能量信息新鲜度 $\times$ 能量紧迫度”作为非中心节点的内生价格信号,动态驱动能量共享触发、排序和预算清算,突破传统效率—公平权衡。

\textbf{2) InfoNode 数字孪生机制:}
通过在每个节点构建数字孪生账户,实现低通信开销下的实时高保真状态同步,显著减少因信息滞后导致的错误调度与资源浪费。

\textbf{3) Energy-Efficient Transfer Opportunistic Routing (EETOR) 能量传输路径治理机制:}
通过路径效率阈值、最大跳数限制及弱势节点保护策略治理多跳路径外部性,从结构上提升能量传输效率。

\textbf{4) 前瞻性时长规划:}
通过预测与实验驱动的跨期资源分配,使供能强度与传输时长随供需变化动态调整,避免静态配给导致的长期效用损失。

本文的主要贡献如下:

1) 从机制设计角度系统性揭示了传统能量共享 WSN 被静态帕累托边界束缚的根本原因,说明信息滞后、路径外部性与静态规则共同导致可行解空间收缩;

2) 提出 AOEI、InfoNode、EETOR 与前瞻性调度等四项制度创新,构建了全新的能量共享机制设计框架;

3) 从理论层面证明所提出机制能够扩展系统可行解空间,实现帕累托边界的动态外移,使多个目标可同时获得正和式提升;

4) 通过大规模实验验证了机制的有效性,表明本文方法可在不增加总能量消耗的前提下同时提升网络寿命、能量效率、信息时效性与弱势节点公平性。
