\section{Discussion}
本节从可解释性、互补性、复杂度与可扩展性、推广边界与伦理公平等方面讨论本文机制的意义与限制。

\subsection{可解释性与制度性价值}
与"黑箱"式策略学习相比,以 AOEI 为核心的价格信号与 InfoNode 数字孪生使触发时机、对象选择与强度决策具备明确的经济学语义,并通过去重/动态等待与 EETOR 路径治理形成"规则—行为—结果"的可追溯链条。这种制度化表达为跨场景迁移与运维策略审计提供依据,有助于在面向合规的工业/城市场景中落地\cite{marketMechanismWSN,digitalTwinSurvey}。

\subsection{与优化/学习方法的互补关系}
本文并不排斥 Lyapunov 或 DRL,而是为其提供价格化触发、路径治理与信息获取的基础设施:优化方法可在固定预算/阈值下提供稳定性保证,DRL 可在时长/额度的连续空间中学习更优匹配;二者与 AOEI—InfoNode—EETOR 的组合体现"规则优先、学习增益"的原则,有利于提升整体性能与鲁棒性\cite{lyapunovEnergy,drlSurvey}。

\subsection{复杂度、通信与计算开销}
量化分析系统开销(基于\(N=60\)配置):\textbf{(a)计算复杂度}——单次传能决策耗时\(T_{\text{comp}}=12.3 \pm 1.8\)~ms(前瞻K值优化占\(8.1\)~ms,路径规划占\(3.2\)~ms,InfoNode查询占\(0.6\)~ms),相对基线Lyapunov(\(4.2\)~ms)增加\(2.93\times\)但仍满足实时约束(决策周期10分钟\(=6\times 10^5\)~ms);\textbf{(b)存储开销}——InfoNode三级缓存总占用\(2.4\)~MB(L1:180~KB,L2:1.8~MB,L3:420~KB),相对节点总存储(假设32~MB)占比\(7.5\%\);\textbf{(c)通信开销}——机会主义上报降低通信能耗\(42.7\%\)(见§5.8),单次上报数据包大小\(B_{\text{info}}=1000\)~bits(含能量16bits、位置32bits、AOEI 16bits、时间戳32bits等),传输能耗\(E_{\text{tx}}=0.12\)~J/包(相对传能功率1000~J可忽略)。路径治理的效率阈值与跳数限制将搜索空间从\(O(N^{H_{\max}})\)降至\(O(N^2)\)(通过剪枝拒绝\(88\%\)候选路径)。整体而言,本文以规则换复杂度(计算增加\(<3\times\)),在通信与存储成本增加\(<10\%\)的前提下,提升可解释性与性能(寿命\(+18.3\%\)、公平\(+26.4\%\))。然而,在高密度网络(\(N>100\))与强非平稳场景(昼夜比\(>10:1\))中,分位数阈值、动态预算、弱势权重等\(>15\)个参数的联合调校需结合离线贝叶斯优化(耗时\(\approx 24\)小时)与在线梯度下降(步长\(\alpha=0.01\))以避免振荡(振荡判据:连续5次决策的方差\(>0.2\))。

\subsection{推广边界与外推风险}
定量分析适用边界:\textbf{(a)效率模型偏差}——当实际传输效率偏离\(\eta(d)=0.6/d^2\)幂律模型(如室内多径衰落导致\(\eta_{\text{实测}}/\eta_{\text{模型}} \in [0.6, 1.4]\)),EETOR的阈值\(\eta_{\text{th}}=0.05\)需重新标定(建议范围\([0.03, 0.08]\),依实测数据分位数);在强遮挡环境(\(\eta_0<0.3\)),低效路径占比可能升至\(15\%\sim 20\%\)(仍优于无约束的\(>30\%\))。\textbf{(b)信息稀疏场景}——当节点密度\(\rho<10^{-3}\)节点/m\(^2\)(稀疏网络)且通信半径\(R_c<20\)~m时,机会主义上报的覆盖率从\(>95\%\)降至\(72\%\sim 85\%\)(部分节点孤立),通信能耗收益(节省\(E_{\text{com}}\)/额外\(E_{\text{tx}}\))从\(3.5:1\)降至\(1.8:1\);建议此场景下混合周期上报(间隔60分钟)。\textbf{(c)极端噪声}——在环境噪声\(\zeta(t)\)方差\(\sigma_{\zeta}^2>0.3\)(极端天气,辐照波动\(>50\%\))时,前瞻K值优化的预测误差从\(<8\%\)升至\(18\%\sim 25\%\),导致寿命提升幅度从18.3\%降至9.7\%(仍显著,\(p<0.05\))。针对上述情形,建议部署前在目标环境进行小规模标定实验(\(N=10\sim 15\)节点,运行72小时),并引入置信度门控:当InfoNode的能量估算置信度\(<0.7\)(估算误差\(>15\%\))时,触发保守策略(冷却期延长至45分钟,\(K\)值下调\(-1\));当连续3次传能后CV未改善时,回退至静态Lyapunov基线运行30分钟后重新评估。

\subsection{公平与伦理考量}
弱势保护通过参数化权重\(\omega_{\text{weak}}=1.5\)实现,但社会意义与工程后果需联合评估:\textbf{(a)基本服务保障}——弱势节点(\(E<q_{0.3}\))的最低服务频度提升至\(2.3\)次/小时(vs.无保护的\(0.12\)次/小时,提升\(19.2\times\)),确保\(>99.5\%\)节点能量维持在安全水平(\(>15\%\theta\));\textbf{(b)效率成本}——在12组配置中,弱势保护导致效率降低\(3.2\% \pm 1.1\%\)(中位数),但公平性改善\(26.4\%\),综合目标(\(\eta_E \times (1-\text{CV})\))提升\(18.7\%\),验证公平-效率的正净收益;\textbf{(c)高贡献节点影响}——定义高贡献节点为供能次数\(>q_{0.7}\)(70\%分位),弱势保护使其被选中概率从\(0.52\)降至\(0.47\)(降低\(9.6\%\)),但总供能量仅降低\(4.1\%\)(因路径效率优化),未形成长期抑制(连续10天仿真中,高贡献节点的能量轨迹稳定)。公平与效率的权衡系数\((\omega_{\text{weak}}, w_4)\)应由应用场景(如环境监测要求全覆盖、智能电网优先关键节点)与运营方的社会偏好联合确定,本文提供参数化接口与量化评估框架(弹性系数\(\epsilon_{\text{公平-效率}}=0.23\)),体现经济学机制相对单纯工程优化的制度化优势。

