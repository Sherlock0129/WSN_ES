\section{讨论}
本章围绕本文的研究动机与机制设计,结合第五章已报告的结果与图表,对方法的有效性、协同机理、适用边界与工程实践进行系统讨论。全章严格以第五章给出的实验证据为依据,不引入未验证的数据与统计结论。同时,为增强自洽性,本章在不引入新数据的前提下,适度回扣第三章(建模)与第四章(机制设计)的关键设定与符号,使讨论与模型—机制表述保持一致。

\subsection{回答研究问题}
引言中提出的核心问题在于:传统能量共享 WSN 被静态帕累托边界束缚,难以在不增加总能量预算的前提下同时提升多个性能指标。本文通过制度创新(TM1--TM4)试图实现动态帕累托边界外移。本节基于第五章的实验结果,系统回答以下关键问题:

\paragraph{问题一:机制栈是否实现静态帕累托边界的外移?}
基于表~\ref{tab:summary},四项核心指标(存活面积、路径效率、CV 均值、AoEI 中位)的对比显示 Proposed System 在所有指标上都显著优于 Stage 2--4。具体而言:在效率—公平维度,Proposed System 的路径效率为 0.81、CV 均值为 0.15,实现了"高效率伴随更好公平"的协同;在寿命—效率维度,存活面积为 $2.99\times10^5$ node-min、路径效率为 0.81,实现了"高寿命、高效率"的协同提升;在信息新鲜度—通信开销维度,AoEI 中位数为 262 分钟、通信能耗为 25.5 kJ,实现了"低通信成本、高信息新鲜度"的协同优化。该结果验证了引言中的核心主张:通过制度创新改变可行解空间的形状与规模,可在不增加总能量预算的前提下同时提升多个性能指标,实现从静态帕累托边界到动态帕累托边界的跨越。

\paragraph{问题二:AoEI+InfoNode 是否在保证信息新鲜度的同时压低通信能耗?}
右图中可知(Fig.~\ref{fig:aoei}),Proposed System 的 AoEI 中位数为 262 分钟,显著低于 Stage 2/3/4 的 359 分钟;通信能耗(Fig.~\ref{fig:energy-cons})仅为 25.5 kJ,相比其他方法降低一个数量级。该结果验证了 TM2(InfoNode 数字孪生与机会主义上报)的设计理念:通过信息价值函数 $V_{\mathrm{info}}(t)$ 与自适应等待时间 $T_{\max}^{\mathrm{adaptive}}$,在低通信开销下维持状态可得性。信息价值权重机制有效抑制了冗余上报,实现了"以低通信成本提供可用信息"的目标。进一步分析发现,Proposed System 的 AoEI 与节点能量之间的 Spearman 相关系数为 $-0.42$,体现了"弱势优先"的信息收集策略,验证了信息-能量耦合优化的有效性。

\paragraph{问题三:EETOR+弱势节点权重是否兼顾生存与公平?}
右图中可知(Fig.~\ref{fig:path-eff}),Proposed System 的路径效率 75\% 分位达到 0.81,低效率路径($\eta<0.6$)占比仅为 12\%,显著低于 Stage 2/3 的 28\%/21\%。右图中可知(Fig.~\ref{fig:equity}),CV 均值为 0.15,显著低于对照方法,且 CV 的一阶导数绝大部分时间保持在 $[-5\times10^{-5},5\times10^{-5}]$ 内,表明均衡变化速率受到良好控制。该结果验证了 TM3(EETOR 专用路由与效率阈值)通过硬约束与路径评分函数治理多跳负外部性的有效性,以及 TM4(自适应传输时长决策)结合自适应 $V$ 参数调整实现跨期收益最大化的有效性。实验结果表明,Proposed System 在存活面积($2.99\times10^5$ node-min)与 CV 均值(0.15)两个看似冲突的目标上实现了协同提升,验证了"路径治理 + 自适应时长"能够兼顾生存与公平。

\subsection{机制协同机理}
从机制链条看,四个要素(TM1--TM4)并非孤立,而是形成"分工—耦合—反馈"的结构逻辑:价格信号离散化(TM1)减少无效触发,为信息层治理(TM2)提供了更健康的能量与频率基线;InfoNode 与价值加权上报(TM2)提升触发判断的准确性,使路径治理(TM3)更少面临突发的高频干预;EETOR 路径治理(TM3)稳定了链路成本,使价格触发与信息层决策具备可预期的效果;自适应时长决策(TM4)结合自适应 $V$ 参数,实现了"收益—损耗—时效—信息增益"的统一平衡。

第五章的实验现象共同验证了该闭环的正反馈结构:传能次数从固定周期的 1543 次降至 504 次(降幅 67.3\%),通信能耗降至 25.5 kJ(降低一个数量级),WET 总能耗从 $1.641\times10^6$ J 降至 $1.505\times10^6$ J,同时存活面积提升 58\%--167\%,路径效率 75\% 分位达到 0.81,CV 均值降至 0.15,AoEI 中位数降至 262 分钟。该综合效果表明,更少的无效触发带来更低的通信与损耗,进而使路径与时长决策处于更健康的能量与信息基线之上,最终在多维指标上体现出更稳健的表现。该闭环与第三章的能量演化方程(式(1))与链路特性模型(式(2)--(3))一致,并与第四章 TM1--TM4 的跨层耦合设计互为印证。

\subsection{经济学理论的有效性验证}
第四章从机制设计理论视角阐述了四项机制(TM1--TM4)的经济学基础,本节基于第五章的实验结果,验证这些经济学理论在 WSN 能量共享场景中的有效性,并讨论实验结果对理论预测的支撑。

\paragraph{价格信号引导资源配置的有效性验证}
价格理论预测,价格信号通过反映资源稀缺性能够引导资源流向高价值用途。第四章提出的价格信号离散化机制(TM1)将连续价格信号 $P_i(t)$ 映射为网络级触发分量,理论上应能减少无效资源配置。实验结果表明,传能次数从固定周期的 1543 次降至 504 次(降幅 67.3\%),验证了价格信号在减少无效触发方面的引导作用。该结果支持了价格理论在 WSN 能量共享场景中的适用性:通过将信息新鲜度与能量紧迫度统一映射为价格信号,系统能够更准确地识别高价值需求,从而在相同能量预算下实现更高的配置效率。

\paragraph{信息透明化抑制逆向选择的实证证据}
信息经济学理论预测,信息不对称会导致逆向选择,降低市场效率。第四章提出的 InfoNode 机制(TM2)通过理论能量计算消除信息不对称,理论上应能抑制逆向选择并提升市场效率。实验结果表明,通信能耗降至 25.5 kJ 的同时,AoEI 中位数从 359 分钟降至 262 分钟,验证了信息透明化机制在抑制逆向选择方面的有效性。该结果支持了信息经济学理论在 WSN 场景中的适用性:通过消除节点隐瞒状态的收益空间,系统能够避免资源错配,在低通信成本下实现更高的信息新鲜度。进一步分析发现,AoEI 与节点能量之间的 Spearman 相关系数为 $-0.42$,体现了"弱势优先"的信息收集策略,验证了信息-能量耦合优化的有效性。

\paragraph{外部性内部化提升系统效率的验证}
外部性理论预测,负外部性的内部化能够使私人成本接近社会成本,提升系统整体效率。第四章提出的 EETOR 机制(TM3)通过效率阈值与路径评分函数将路径外部性成本内化,理论上应能减少低效率路径的使用。实验结果表明,低效率路径占比从 Stage 2/3 的 28\%/21\% 降至 12\%,路径效率 75\% 分位达到 0.81,验证了外部性内部化在提升系统整体效率方面的有效性。该结果支持了外部性理论在 WSN 场景中的适用性:通过将路径外部性成本内化到路径选择决策中,系统能够选择社会成本更低的路径,从而在不增加总能量预算的前提下提升传输效率。

\paragraph{激励相容机制引导帕累托改进的验证}
机制设计理论预测,激励相容机制能够引导参与者自发选择有利于全局帕累托改进的策略。第四章提出的四项机制(TM1--TM4)通过价格信号、信息透明与路径治理的协同,理论上应能实现激励相容的均衡状态。实验结果表明,Proposed System 在存活面积、路径效率、CV 均值、AoEI 中位四项核心指标上均显著优于对照方法,验证了机制设计在引导节点自发选择有利于全局帕累托改进策略方面的有效性。该结果支持了机制设计理论在 WSN 场景中的适用性:通过消除节点隐瞒状态的收益并内生化支付意愿,系统能够实现个体理性与集体理性的协调,推动系统向帕累托改进方向演化。

\paragraph{跨期优化实现动态帕累托改进的验证}
跨期选择理论预测,跨期优化能够在不同时间维度上权衡收益与成本,实现长期收益最大化。第四章提出的前瞻性时长规划(TM4)通过综合考虑能量收益、损耗惩罚、时效惩罚与信息奖励,理论上应能实现跨期多维收益的平衡。实验结果表明,Proposed System 在存活面积($2.99\times10^5$ node-min)与 CV 均值(0.15)两个看似冲突的目标上实现了协同提升,验证了跨期优化在实现动态帕累托改进方面的有效性。该结果支持了跨期选择理论在 WSN 场景中的适用性:通过将外生的时长约束转化为内生的跨期优化决策,系统能够在长期视角下实现资源配置的最优化,避免短期最优与长期最优之间的冲突。

\paragraph{制度创新推动帕累托边界外移的实证支持}
帕累托效率理论预测,制度创新通过改变可行解空间的形状与规模,能够实现帕累托边界的动态外移。第四章提出的四项机制(TM1--TM4)通过价格信号内生化、状态透明化、路径外部性治理与跨期优化,理论上应能扩展可行解空间并推动帕累托边界外移。实验结果表明,Proposed System 在效率—公平维度(路径效率 0.81、CV 均值 0.15)、寿命—效率维度(存活面积 $2.99\times10^5$ node-min、路径效率 0.81)、信息新鲜度—通信开销维度(AoEI 中位数 262 分钟、通信能耗 25.5 kJ)等多个维度实现了协同提升,验证了"在不增加总能量预算的前提下,同时提升多个性能指标"的核心主张。该结果支持了制度创新理论在 WSN 场景中的适用性:通过改变可行解空间的形状与规模,系统能够实现多个性能指标的正和式改进,体现了制度创新在扩展可行解空间、实现多目标协同优化方面的经济学价值。
\subsection{与现有工作的比较}
第二章 Related Work 将能量共享研究分为五个阶段(无规则、局部优化、集中式调度、黑箱学习、初代市场机制),并识别出四个制度缺口:(i)价格信号缺位、(ii)信息基础设施薄弱、(iii)外部性治理缺失、(iv)激励相容机制缺位。本节基于第五章的实验结果,明确说明本文在哪些维度真正填补了这些缺口:

\begin{itemize}
  \item 价格信号缺位(i)的填补:TM1 通过将连续价格信号 $P_i(t)$ 离散化为网络级触发分量,实现了"信息新鲜度—价值—紧急性"的统一经济学刻画。实验结果表明,该机制使传能次数从固定周期的 1543 次降至 504 次,降幅达 67.3\%,验证了价格信号在资源分配中的引导作用。
  \item 信息基础设施薄弱(ii)的填补:TM2 通过 InfoNode 数字孪生与机会主义上报,实现了低通信开销下的实时高保真状态同步。实验结果表明,通信能耗降至 25.5 kJ,AoEI 中位数从 359 分钟降至 262 分钟,验证了信息透明化机制在抑制逆向选择与道德风险方面的有效性。
  \item 外部性治理缺失(iii)的填补:TM3 通过 EETOR 的效率阈值 $\eta_{\text{th}}$ 与最大跳数 $H_{\max}$,实现了路径负外部性的内部化。实验结果表明,低效率路径占比从 Stage 2/3 的 28\%/21\% 降至 12\%,验证了路径治理在抑制低效多跳方面的有效性。
  \item 激励相容机制缺位(iv)的填补:TM1--TM4 的组合通过价格信号、信息透明与路径治理的协同,实现了个体理性、预算平衡与激励相容的均衡状态。实验结果表明,Proposed System 在存活面积、路径效率、CV 均值、AoEI 中位四项核心指标上均显著优于对照方法,验证了机制设计在引导节点自发选择有利于全局帕累托改进策略方面的有效性。
\end{itemize}

与已有 market-based WSN 机制相比,本文的 AoEI+InfoNode+EETOR+时长规划在以下方面更强:(1)将价格信号从单一 AoEI 扩展为包含信息价值、能量缺口与公平权重的综合价格;(2)通过 InfoNode 三级缓存架构实现更细粒度的状态管理;(3)通过 EETOR 硬约束与路径评分函数实现更严格的路径外部性治理;(4)通过自适应时长决策实现跨期多维收益最大化。在以下方面仍然类似:均采用市场机制进行资源分配,均强调信息透明化的重要性,均关注路径效率与公平性的平衡。

\subsection{局限性}
基于第五章的实验设定(30 个节点、$5\,\mathrm{m}\times5\,\mathrm{m}$ 区域、ADCR 重簇周期 $360\,\mathrm{min}$、主动触发间隔 $20\,\mathrm{min}$)与结果,本节集中讨论建模假设和实验条件对结论的影响,明确"在哪些条件下,本机制可能不再最优":

\paragraph{建模层面的局限性}
\begin{itemize}
  \item 硬件参数简化:第三章的链路特性模型(式(2)--(3))假设传输效率 $\eta(d_{ij})=\min\{1,\max\{0,\tfrac{\eta_0}{d_{ij}^{\gamma}}\}\}$,其中 $\eta_0=0.6$(最大效率 60\%)、$\gamma=2.0$。若真实硬件效率低于 60\%(如受遮挡、多径等因素影响),EETOR 的阈值 $\eta_{\text{th}}$ 与最大跳数 $H_{\max}$ 仍提供方向性约束,但阈值取值与收益幅度会随环境变化,需通过小规模标定获取代表性效率分布并调整参数。
  \item 环境模型的理想化:第三章的环境供给模型(式(1))通过日内辐照度 $G(t)$ 与天气扰动因子 $\zeta(t)$ 刻画非平稳性,但未考虑极端天气事件(如连续阴雨)对能量采集的长期影响。强非平稳供给下,自适应 $V$ 参数的调整速率 $\alpha$、窗口大小 $W$ 等需结合离线回放与在线试运行调优,避免参数漂移导致的性能退化。
  \item sink 节点能量充足假设:第三章假设 sink 节点能量充足且不作为优化对象。若 sink 也能量受限,会改变机制栈的最优性,需要将 sink 纳入能量共享网络并调整触发与路径选择策略。
\end{itemize}

\paragraph{实验层面的局限性}
\begin{itemize}
  \item 节点数量与拓扑类型有限:实验仅在 30 个节点的单一拓扑(随机分布)上进行,未覆盖第三章建模中支持的"规则网格、能量空洞、节点移动"等多场景。节点数量从 30 扩展到 100 时,TM1--TM4 的复杂度与收益如何变化尚需进一步验证。建议在更大规模、异构硬件与复杂业务负载下进行泛化性验证。
  \item 仿真 vs 实物验证的差距:本章结论基于统一的仿真协议(5 次独立随机种子、Bootstrap 置信区间),真实部署中的链路不确定性、硬件离散性与运维噪声可能放大机制的滞后性。需配置额外的缓冲策略(如冗余储能、安全裕度、触发冷却期延长)才能保持预期表现,并建议推进原型系统与外场试验以评估在 RF/WPT 等不同供能技术下的落地门槛与改造成本。
\end{itemize}

\paragraph{算法层面的局限性}
\begin{itemize}
  \item 参数依赖:自适应 $V$ 参数调整机制(初始值 $V_0=0.5$、范围 $[0.1,2.0]$、调整率 $\alpha=0.1$、窗口 $W=10$)需根据网络规模与能量模型进行标定,存在参数敏感性。建议采用"小规模标定—轻量规则—可审计日志"的部署顺序,通过离线回放与在线试运行相结合的方式优化参数配置。
  \item 复杂度与扩展性:TM1 的价格信号离散化为 $O(N)$ 复杂度,TM3 的 EETOR 路径选择为 $O(N^2)$ 复杂度。当网络规模扩展到数百节点时,计算开销可能成为瓶颈,需要优化算法或采用分布式实现。
  \item 稀疏网络与覆盖要求:在低密度或通信半径受限场景,机会主义上报可能出现覆盖断裂,导致 InfoNode 状态更新不及时。此时应以周期上报为兜底(如 ADCR 的 $360\,\mathrm{min}$ 重簇周期),并用信息价值权重限制其干预成本,确保在维持 InfoNode 一致性的同时控制通信开销。
\end{itemize}

\subsection{工程实践与实现要点}
基于第四章的机制设计(TM1--TM4)与第五章的实验验证,工程实践中的实现要点包括:

\begin{itemize}
  \item 实现复杂度:TM1 的价格信号离散化为 $O(N)$ 复杂度,TM2 的 InfoNode 维护为规则化逻辑,TM3 的 EETOR 路径选择为 $O(N^2)$ 复杂度,TM4 的时长自适应可在离散候选集上枚举打分。工程重点在于参数标定与状态一致性,而非单次计算成本。
  \item 通信与数据面:信息捎带、路径内去重与自适应等待机制应作为默认策略;若长时间无传能事件或拓扑稀疏,可回退到低频周期上报以维持 InfoNode 一致性。
  \item 路径与公平配置:效率阈值 $\eta_{\text{th}}$ 与最大跳数 $H_{\max}$ 需要与弱势权重 $\omega_i(t)$ 联动,避免中继节点长期过载。权重可基于 InfoNode 的历史统计进行阶段性再标定。
  \item 参数自适应配置:自适应 $V$ 参数调整机制需根据网络规模与能量模型进行标定,建议采用"小规模标定—轻量规则—可审计日志"的部署顺序。
\end{itemize}

\subsection{小结}
对照引言中的动机,实证结果支持如下判断:以价格信号离散化为触发(TM1)、以 InfoNode 数字孪生为透明基础(TM2)、以 EETOR 为路径治理(TM3)、以自适应时长为跨期匹配(TM4)的组合,能够在既定资源预算与通信约束下减少无效动作,保持信息新鲜度,并呈现更平稳的能量演化。具体而言,Proposed System 在存活面积($2.99\times10^5$ node-min)、路径效率(0.81)、CV 均值(0.15)、AoEI 中位(262 分钟)四项核心指标上均显著优于三类对照方法(Stage 2/3/4),实现了从静态帕累托边界到动态帕累托边界的跨越。本研究展示了跨层机制如何以低开销实现更稳定的能量—信息前沿,为长期运行的 WSN 提供了结构化、可解释且具推广性的制度化方案。
