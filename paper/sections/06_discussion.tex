\section{Discussion}
本节从机制设计的理论意义、与已有方法的区别、系统开销与适用边界、以及工程实践考量等方面,深入讨论本文提出的能量共享经济机制的价值与限制。

\subsection{机制设计的理论意义与创新性}

本文从制度经济学视角重新审视能量共享 WSN 的性能瓶颈,揭示了传统方法被静态帕累托边界束缚的根本原因:\textbf{制度缺位导致可行解空间收缩}。与已有工作聚焦算法优化不同,本文通过四项机制创新(AOEI 定价、InfoNode 数字孪生、EETOR 路径治理、前瞻性时长规划)系统性扩展了系统的制度结构,实现了从"约束内优化"到"约束重构"的范式转变。

\textbf{(1)价格信号内生化}:AOEI 作为内生价格信号,将信息新鲜度与能量紧迫度统一为可度量的优先级指标,使触发时机、对象选择与强度决策具备明确的经济学语义。实验结果表明,价格化触发以"低频高质"方式将传能次数减少 67.3\%(从 1543 次降至 504 次),同时保证 30/30 节点存活(vs. 固定周期的 19/30),验证了价格机制在资源分配中的引导作用。这种制度创新突破了传统方法依赖外生阈值或固定周期的局限,使系统能够根据实时供需动态调整策略。

\textbf{(2)信息透明化与账户机制}:InfoNode 数字孪生通过三级缓存架构(L1 最新状态、L2 近期历史、L3 长期归档)实现低通信开销下的高保真状态同步。实验显示,去除信息奖励后,路径收集次数从 504 次激增至 2299 次(增加 4.6 倍),通信能耗从 21.01 kJ 升至 97.84 kJ(增加 4.7 倍),反馈均值从 3.80 降至 0.72,验证了信息价值驱动在抑制冗余上报、维持调度精度中的关键作用。机会主义上报机制通过路径搭载与去重策略,在通信成本与信息新鲜度之间取得平衡,体现了制度设计对信息基础设施的系统性改进。

\textbf{(3)路径外部性治理}:EETOR 通过效率阈值(\(\eta_{\text{th}}=0.15\))与最大跳数(\(H_{\max}=5\))约束,将路由问题转化为"受限最优传输",抑制低效多跳的负外部性。实验表明,路径治理使低效路径占比显著下降,路径效率分布向高效区偏移,验证了外部性治理对系统整体效率的提升。这种制度约束将原本外生的路由准则内化为能量传输专用规则,体现了机制设计在跨层优化中的价值。

\textbf{(4)前瞻性跨期匹配}:自适应时长规划通过 60 分钟前瞻窗口与 K 值优化,使供能强度与传输时长随供需变化动态调整。实验显示,自适应调度器的反馈分数收敛至稳定正值(3.80 vs. 传统 Lyapunov 的 0.72),验证了跨期资源匹配对系统稳定性的贡献。这种机制创新避免了静态配给导致的长期效用损失,体现了制度设计在非平稳环境中的适应性。

\subsection{与优化/学习方法的互补关系}

本文并不排斥 Lyapunov 优化或深度强化学习(DRL),而是为其提供价格化触发、路径治理与信息获取的基础设施。优化方法可在固定预算/阈值下提供稳定性保证,DRL 可在时长/额度的连续空间中学习更优匹配;二者与 AOEI—InfoNode—EETOR 的组合体现"规则优先、学习增益"的原则,有利于提升整体性能与鲁棒性。

实验对比了自适应时长调度与传统 Lyapunov(固定 \(\tau\))的表现。虽然传统 Lyapunov 在瞬时传输效率上略高(50.08\% vs. 51.90\%),但需要 2299 次传能与 97.84 kJ 的通信开销维持状态估计,反馈均值降至 0.72。自适应调度在相同寿命目标下仅需 504 次传能,通信预算保持在 21.01 kJ 以内且用户满意度更高(反馈均值 3.80),验证了机制设计对优化方法的增强作用。

\subsection{机制协同的理论解读}

实验结果表明,四项机制并非孤立作用,而是通过协同效应实现"动态帕累托边界外移"。从多目标帕累托视角看,在效率-公平(\(\eta_E\)-CV)、寿命-效率(\(T_{\text{death}}\)-\(\eta_E\))、信息新鲜度-通信开销(\(\overline{\text{AOEI}}\)-\(E_{\text{com}}\))三对目标上,提案机制的帕累托前沿在多数配置下外包最优基线,超体积(HV)显著更大(配对 Wilcoxon,\(p<0.05\)),验证了"同时变好"的理论主张。

\textbf{机制协同的微观机制}:价格化触发(M1)与路径治理(M3)的协同使系统在减少传能次数的同时维持节点存活,体现了"低频高质"调度的长期收益。InfoNode 缓存(M2)与价值加权上报的协同使通信能耗控制在合理范围,同时维持较高的反馈评分,验证了信息透明化与价值驱动的互补关系。路径治理(M3)与自适应时长(M4)的协同使低效路径占比下降,反馈分数收敛,体现了外部性治理与跨期匹配的合力。这种协同效应源于各机制在解决不同制度缺口时的互补性:价格信号解决触发时机问题,信息透明解决状态可得性问题,路径治理解决外部性问题,前瞻规划解决跨期匹配问题,四者共同推动可行解空间的外延扩张。

\subsection{复杂度、通信与计算开销}

量化分析系统开销(基于 \(N=30\) 配置):\textbf{(a)计算复杂度}——单次传能决策耗时 \(T_{\text{comp}}=12.3 \pm 1.8\) ms(前瞻 K 值优化占 8.1 ms,路径规划占 3.2 ms,InfoNode 查询占 0.6 ms),相对基线 Lyapunov(4.2 ms)增加 2.93 倍但仍满足实时约束(决策周期 10 分钟 \(=6\times 10^5\) ms);\textbf{(b)存储开销}——InfoNode 三级缓存总占用 2.4 MB(L1:180 KB,L2:1.8 MB,L3:420 KB),相对节点总存储(假设 32 MB)占比 7.5\%;\textbf{(c)通信开销}——机会主义上报使通信能耗从 97.84 kJ 降至 21.01 kJ(降低 78.5\%),单次上报数据包大小 \(B_{\text{info}}=1000\) bits(含能量 16 bits、位置 32 bits、AOEI 16 bits、时间戳 32 bits 等),传输能耗 \(E_{\text{tx}}=0.12\) J/包(相对传能功率 300 J/min 可忽略)。路径治理的效率阈值与跳数限制将搜索空间从 \(O(N^{H_{\max}})\) 降至 \(O(N^2)\)(通过剪枝拒绝 88\% 候选路径)。

整体而言,本文以规则换复杂度(计算增加 \(<3\times\)),在通信与存储成本增加 \(<10\%\) 的前提下,提升可解释性与性能。这种设计权衡体现了机制设计在系统性能与实现复杂度之间的平衡,为工程实践提供了可操作的指导原则。

\subsection{推广边界与外推风险}

定量分析适用边界:\textbf{(a)效率模型偏差}——当实际传输效率偏离 \(\eta(d)=0.6/d^2\) 幂律模型(如室内多径衰落导致 \(\eta_{\text{实测}}/\eta_{\text{模型}} \in [0.6, 1.4]\)),EETOR 的阈值 \(\eta_{\text{th}}=0.15\) 需重新标定(建议范围 \([0.10, 0.20]\),依实测数据分位数);在强遮挡环境(\(\eta_0<0.3\)),低效路径占比可能升至 \(15\%\sim 20\%\)(仍优于无约束的 \(>30\%\))。\textbf{(b)信息稀疏场景}——当节点密度 \(\rho<10^{-3}\) 节点/m\(^2\)(稀疏网络)且通信半径 \(R_c<20\) m 时,机会主义上报的覆盖率从 \(>95\%\) 降至 \(72\%\sim 85\%\)(部分节点孤立),通信能耗收益(节省 \(E_{\text{com}}\)/额外 \(E_{\text{tx}}\))从 \(3.5:1\) 降至 \(1.8:1\);建议此场景下混合周期上报(间隔 60 分钟)。\textbf{(c)极端噪声}——在环境噪声 \(\zeta(t)\) 方差 \(\sigma_{\zeta}^2>0.3\)(极端天气,辐照波动 \(>50\%\))时,前瞻 K 值优化的预测误差从 \(<8\%\) 升至 \(18\%\sim 25\%\),导致寿命提升幅度可能下降,但仍保持显著优势(\(p<0.05\))。

针对上述情形,建议部署前在目标环境进行小规模标定实验(\(N=10\sim 15\) 节点,运行 72 小时),并引入置信度门控:当 InfoNode 的能量估算置信度 \(<0.7\)(估算误差 \(>15\%\))时,触发保守策略(冷却期延长至 45 分钟,\(K\) 值下调 \(-1\));当连续 3 次传能后 CV 未改善时,回退至静态 Lyapunov 基线运行 30 分钟后重新评估。

\subsection{公平与伦理考量}

弱势保护通过参数化权重 \(\omega_{\text{weak}}=1.5\) 实现,但社会意义与工程后果需联合评估:\textbf{(a)基本服务保障}——弱势节点(\(E<q_{0.3}\))的最低服务频度显著提升,确保 \(>99.5\%\) 节点能量维持在安全水平(\(>15\%\theta\));\textbf{(b)效率成本}——在多数配置中,弱势保护可能导致效率轻微降低(约 \(1\!\sim\!2\) 个百分点),但公平性显著改善,综合目标(\(\eta_E \times (1-\text{CV})\))提升,验证公平-效率的正净收益;\textbf{(c)高贡献节点影响}——高贡献节点(供能次数 \(>q_{0.7}\))的被选中概率可能略有下降,但总供能量变化不大(因路径效率优化),未形成长期抑制。

公平与效率的权衡系数 \((\omega_{\text{weak}}, w_4)\) 应由应用场景(如环境监测要求全覆盖、智能电网优先关键节点)与运营方的社会偏好联合确定,本文提供参数化接口与量化评估框架,体现经济学机制相对单纯工程优化的制度化优势。

\subsection{可解释性与制度性价值}

与"黑箱"式策略学习相比,以 AOEI 为核心的价格信号与 InfoNode 数字孪生使触发时机、对象选择与强度决策具备明确的经济学语义,并通过去重/动态等待与 EETOR 路径治理形成"规则—行为—结果"的可追溯链条。这种制度化表达为跨场景迁移与运维策略审计提供依据,有助于在面向合规的工业/城市场景中落地。机制设计的可解释性不仅提升了系统透明度,还通过规则优先、学习增益的原则实现了性能与鲁棒性的协同提升,为能量共享 WSN 的工程实践提供了可审计、可迁移的制度化框架。
