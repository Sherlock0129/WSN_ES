\section{Discussion}
\label{sec:discussion}

本章基于第五章的实验结果与分析,从“边界外移”“路径治理”“公平鲁棒”三条主线,对本文提出的制度体系进行系统讨论。与第四章的理论推导相呼应,本章的讨论严格围绕实验设计中的四类对照(E1--E4)、多指标体系(效率、公平、寿命、振荡等)以及帕累托前沿可视化展开,旨在给出机制有效性的统一解释框架。

\subsection{AoEI 与数字孪生驱动的边界外移}

实验 E1(智能被动 vs 固定周期)与 E2(信息价值剖析)直接验证了 AoEI 价格信号以及 InfoNode 数字孪生缓存机制的理论效应。结果表明,在统一的资源约束和网络规模下,引入 AoEI 与数字孪生后,系统在多维指标空间的可行解集合出现了显著的“边界外移”。

首先,在能量预算相同的前提下,价格驱动的智能被动触发替代固定周期触发,使能量传输事件与“能量稀缺—信息价值—网络状态”三者紧密耦合。实验结果显示:价格触发机制能够有效减少无效或过早的传能行为,显著降低调度振荡与重配频率,并在网络寿命、能量效率与能量均衡度(CV 与方差)上同时优于固定周期策略。

其次,InfoNode 数字孪生与机会主义上报显著降低了信息获取成本。与周期直发或简单聚类上报相比,数字孪生提供了一种在“有限通信成本”下获得“近似全局态势”的方式,从而提高触发判断的准确性并减少冷却周期内的误触发。实验数据表明,引入 InfoNode 后,触发事件在时间上的分布更加平滑,能量分布的振荡幅度能够控制在 0.1 以下。

在效率与公平(例如,以有效接收/总消耗作为效率指标、以能量 CV 作为均衡指标)构成的二维空间中,启用 AoEI 与数字孪生的制度配置,其实验点整体位于基线方法的帕累托前沿之外,呈现出明显向高效率、低 CV 区域的整体迁移。这一现象验证了第四章提出的“动态帕累托扩展”观点:通过提升信息价值感知与降低全局状态获取成本,系统在既定资源约束下实现了可行解空间的整体外移,而非仅在原有边界内进行重新分配。

\subsection{EETOR 与前瞻强度优化的路径治理}

路径治理能力主要通过实验 E3(上报制度对比)和 E4(自适应时长调度)加以验证,并结合路径效率分布、低效路径占比和重配频率等指标进行定量分析。

一方面,EETOR 将路径选择从传统的“可达优先”提升为“投资回报驱动”。在最大跳数约束与路径效率下限 $\eta_{\text{th}}=0.10$ 的条件下,高成本、低回报的中继路径被显式剔除,低效路径的占比在规则网格、泊松随机和能量空洞等不同拓扑中均能稳定控制在 10\% 以下。与无路径治理的基线相比,单位能量信息收益提升超过 1.5 倍,路径平均长度缩短,同时路径重配频率下降,表明 EETOR 能够有效抑制由路径外部性引发的系统性浪费。

另一方面,前瞻候选规模优化和动态强度控制机制(Dynamic-$k$)在高负载和非平稳供给场景中发挥了关键作用。通过前瞻时间窗对未来 60 分钟的需求进行预测,并结合滞回带宽防止频繁切换,系统能够在确保覆盖能力的前提下限制同时激活的供能路径数量,从而避免“过度竞争”导致的大量能量消耗和路径震荡。实验结果显示,Dynamic-$k$ 有效降低了重配频率与高成本路径的同时激活比例,使路径集合更接近理论上的高效率子集。

此外,E3 中对机会主义上报、ADCR 聚合上报与周期直发上报的对比进一步表明,信息层治理与路径治理是相辅相成的:机会主义上报减少了低价值状态的注入,使路径构建基于更精确的收益评估;数字孪生与缓存机制则缓解了即时感知不足带来的估计偏差。这些结果从实验上支持了“路径治理与信息治理协同抑制外部性”的制度设计初衷。

\subsection{弱势保护与非平稳自适应机制的公平鲁棒性}

公平性与鲁棒性主要通过 E1、E2 与 E4 三组实验综合验证,考察了弱势节点保护与非平稳自适应时长调度在多种拓扑与昼夜周期供能条件下的表现。

实验结果表明,引入弱势保护机制(包括低能节点优先的权重设计、服务下限约束等)并未导致预期中的显著效率损失。整体来看,在效率下降小于 5\% 的前提下,能量变异系数和方差均改善超过 25\%,低分位能量节点的服务覆盖率与最低服务频度显著提升。这一结果说明,在当前制度配置下,公平性提升与效率下降之间的权衡是温和且可控的,印证了第四章中关于“弱势保护在一定区间内具有净效用”的理论判断。

在昼夜周期明显的可再生供给模型下,非平稳自适应时长调度展现了显著的稳定性优势。若采用固定参数 Lyapunov 调度,系统容易在白天快速累积能量、夜间快速枯竭,能量分布和触发事件序列均出现明显的高频振荡。加入自适应传输时长后,调度策略能够根据当前能量状态与预期供给变化动态调整传输时长与触发节奏,将能量振荡幅度压制在 0.1 以下,且传能事件在昼夜周期中的分布更加平滑,有效降低了因过度调度导致的能量浪费与路径抖动。

进一步地,\emph{AdaptiveLyapunovScheduler} 和 \emph{AdaptiveDurationAwareLyapunovScheduler} 引入的基于网络反馈分数的参数自适应调整机制,使系统能够根据调度效果自动优化 Lyapunov 权重参数 \(V\)。该机制通过维护滑动窗口记录最近调度的反馈分数,综合考虑能量均衡性、存活率、传输效率和整体能量水平四个维度,形成综合反馈评分。当系统检测到持续负反馈时,能够诊断具体问题(效率过低、均衡性差或节点死亡)并采取针对性调整:效率过低时增大 \(V\) 以更重视损耗惩罚,均衡性差时减小 \(V\) 以增强均衡性,节点死亡时显著减小 \(V\) 以优先救活。实验结果表明,这种自适应调整机制在非平稳环境下能够将能量振荡幅度进一步降低 \(15\%-25\%\),同时保持参数调整的稳定性,避免了过度调整导致的系统抖动。该机制与自适应传输时长决策形成互补,共同构成了系统在动态环境下的鲁棒性基础。

此外,信息奖励与 AoEI 驱动的服务优先级调整,使得高价值信息节点能够在保障弱势节点基本服务的前提下,获得相对稳定的调度机会,从而在公平与时效两个维度上同时取得改善。整体上,弱势保护与非平稳自适应机制共同构成了系统的“公平鲁棒性”基础,使网络在面对可再生供给波动和负载变化时仍能维持较高的服务质量与稳定性。

\subsection{多维指标的制度映射与帕累托结构}

基于第五章的可视化与帕累托分析,可以将本文提出的关键制度要素与各项评估指标之间建立较为清晰的映射关系:

\begin{itemize}
    \item 效率提升(有效接收/总消耗)主要来源于 AoEI 价格信号驱动的智能触发与 EETOR 路径治理;
    \item 公平性改善(CV 降低、弱势覆盖率提升)主要由弱势保护权重设计与自适应调度机制保障;
    \item 网络寿命延长(首个节点死亡时间和存活曲线改善)则是价格信号稳定、误触发减少和路径生产率提升的综合结果;
    \item 稳定性提升(振荡幅度、重配频率降低)归因于数字孪生提供的平滑态势感知、滞回控制与有限候选集约束;
    \item 通信负担的下降与信息新鲜度的维持,则依赖机会主义上报与缓存预测之间的平衡。
\end{itemize}

这些映射关系表明,本文提出的制度体系并非简单叠加多个机制,而是围绕“价格信号—信息治理—路径治理—自适应调度”这一统一逻辑有机耦合。帕累托前沿的外移进一步说明,在给定能量预算与拓扑约束下,通过制度创新可以将原本近似零和的效率—公平博弈重构为正和博弈:系统不仅在单一指标上取得提升,而且在多维指标空间中实现了一组占优解集。

\subsection{局限性与未来工作}

尽管实验结果从多个维度支持了本文机制设计的有效性与可行性,本研究仍然存在一些局限,值得在后续工作中深入探索:

\begin{itemize}
    \item \textbf{硬件真实性不足:} 当前仿真模型虽已考虑能量采集模型、传输功率及效率衰减,但尚未完全覆盖硬件非线性、能量测量噪声、无线链路突发衰落等现实因素,仍需在实际平台或硬件在环环境中进一步验证。
    \item \textbf{参数自适应能力有限:} AoEI 权重、信息奖励系数、弱势保护权重以及 Lyapunov 参数 $V$ 等目前依赖经验设定,在跨场景迁移时可能需要重新调参。未来可引入强化学习或元调度(meta-scheduling)方法,实现对权重与参数的在线自适应优化。
    \item \textbf{对极端与对抗性情形的适应性有待加强:} 现有实验主要针对典型拓扑与非对抗环境,在节点存在策略性行为或恶意攻击的场景下,价格信号与路径治理机制的鲁棒性仍需进一步研究。
    \item \textbf{帕累托结构的理论刻画尚不完备:} 虽然通过实验观察到了帕累托边界外移,但对不同制度参数下前沿形态的理论刻画仍不充分,未来可从多目标优化与博弈论视角,对“边界外移”的充要条件进行更严格的分析。
\end{itemize}

总体而言,本文通过 AoEI 价格信号、数字孪生信息治理、EETOR 路径治理与非平稳自适应调度等机制,在实验上验证了“边界外移”“路径治理”“公平鲁棒”三大目标之间可以在相对温和的权衡下实现兼顾。未来,将这些机制进一步嵌入真实硬件平台,并结合学习型调参与博弈论分析,有望进一步提升能量协同网络在复杂环境中的自适应性与长期可持续性。
