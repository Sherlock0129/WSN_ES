\section{Modeling}
\subsection{Node and Scenario Modeling}
本章建立不依赖具体实现细节的抽象化模型。

\textbf{节点建模:}网络由普通传感器节点与物理中心节点组成。物理中心节点不参与能量传输,其职责在于信息汇聚与全网状态维护。普通节点\(i \in \mathcal{N}\)具备以下关键属性与动态过程:
\begin{itemize}
    \item \emph{能量状态}:电池容量\(C_i\)(典型值3.5~mAh, 3.7~V),当前能量\(E_i(t)\),低/高阈值\(\theta_i^{\text{low}}, \theta_i^{\text{high}}\)。
    \item \emph{能量采集}:若启用太阳能,基于日内辐照度模型\(G(t) = G_{\max} \sin(\pi(t-t_{\text{rise}})/(t_{\text{set}}-t_{\text{rise}}))\)采集能量,转换效率\(\eta_{\text{solar}}=0.2\),面板面积\(A=0.1~\text{m}^2\)。
    \item \emph{能量消耗}:感知能耗\(E_{\text{sen}}=0.1~\text{J}\),通信能耗\(E_{\text{com}} = E_{\text{elec}} B + \epsilon_{\text{amp}} B d^{\tau}\)(其中\(B\)为比特率,\(d\)为距离,\(\tau=2\)为路径损耗指数)。
    \item \emph{能量传输效率}:距离\(d\)处的无线能量传输效率\(\eta(d) = \eta_0 / d^{\gamma}\)(\(\eta_0=0.6\), \(\gamma=2.0\),对应1米处最大效率60\%)。
\end{itemize}
每个物理节点在信息空间一一映射为轻量级信息节点(InfoNode),用于调度与路由过程中的状态访问与推理,从而弱化对全局完美信息假设的依赖。

\textbf{场景建模:}从网络结构、环境供给与链路特性三个层面描述:
\begin{itemize}
    \item \emph{网络结构:}设节点集合为 \(\mathcal{N}=\{1,\dots,N\}\),物理中心节点记为 0 且不参与 WET。节点 \(i\) 在时刻 \(t\) 的二维位置为 \(\mathbf{x}_i(t)\in\mathbb{R}^2\)。部署分布支持:
    \begin{enumerate}[label=(\alph*), leftmargin=1.35em]
        \item 均匀随机:\(\mathbf{x}_i(0)\sim \mathcal{U}([0,L]\times[0,L])\);
        \item 规则网格:\(\mathbf{x}_i(0)\) 等间距栅格;
        \item 能量空洞:存在子域 \(\Omega_{\text{hole}}\subset [0,L]^2\),使得 \(\mathbb{P}(\mathbf{x}_i(0)\in \Omega_{\text{hole}})\) 降低,且 \(\mathbb{E}[E_i(0)\mid \mathbf{x}_i(0)\in \Omega_{\text{hole}}]\) 偏低。
    \end{enumerate}
    典型规模 \(N\in[10,100]\)。可选启用移动:
    \[
    \mathbf{x}_i(t+\Delta t)=\mathbf{x}_i(t)+\mathbf{v}_i(t)\,\Delta t,\qquad \|\mathbf{v}_i(t)\|\le v_{\max},
    \]
    其中 \(\mathbf{v}_i(t)\) 可取直线、往返或随机游走模型,以反映动态拓扑。
    \item \emph{环境供给:}若节点 \(i\) 具备采集能力(标识 \(s_i\in\{0,1\}\)),其单位时间采集量建模为
    \[
    E_{h,i}(t)= s_i\cdot \eta_{\text{solar}} A_i\cdot G(t)\cdot \Delta t,
    \]
    其中转换效率 \(\eta_{\text{solar}}\in(0,1)\)、面板面积 \(A_i\),日内辐照度 \(G(t)=G_{\max}\cdot \max\{0,\sin(\pi(t-t_{\text{rise}})/(t_{\text{set}}-t_{\text{rise}}))\}\cdot \zeta(t)\),\(\zeta(t)\) 刻画天气扰动(如 \(\mathbb{E}[\zeta]=1\) 的有界噪声)。非平稳性由 \((t_{\text{rise}},t_{\text{set}})\) 与 \(\zeta(t)\) 共同引入。
    \item \emph{链路特性:}节点对 \((i,j)\) 的距离 \(d_{ij}(t)=\|\mathbf{x}_i(t)-\mathbf{x}_j(t)\|_2\)。能量传输效率
    \[
    \eta(d_{ij})=\min\Big\{1,\max\big\{0,\tfrac{\eta_0}{d_{ij}^{\gamma}}\big\}\Big\},\quad \eta_0\in(0,1),\ \gamma\in[2,4].
    \]
    多跳路径 \(\mathcal{P}= (i=v_0\to v_1\to\cdots\to v_H=j)\) 的总效率 \(\eta_{\mathcal{P}}=\prod_{k=0}^{H-1}\eta(d_{v_k v_{k+1}})\)。
    通信能耗采用常用模型(发送/接收)
    \[
    \begin{aligned}
    E_{\text{tx}}&=E_{\text{elec}}\,B+\epsilon_{\text{amp}}\,B\cdot d_{ij}^{\tau},\\
    E_{\text{rx}}&=E_{\text{elec}}\,B,
    \end{aligned}
    \]
    其中 \(\tau\) 为路径损耗指数、\(\epsilon_{\text{amp}}>0\)。单次 WET 发送量 \(E_{\text{send}}\) 的有效接收
    \[
    E_{\text{recv}}=\eta_{\mathcal{P}}\cdot E_{\text{send}},\qquad E_{\text{loss}}=E_{\text{send}}-E_{\text{recv}}.
    \]
    系统施加最大跳数 \(H_{\max}\) 与效率阈值 \(\eta_{\text{th}}\),仅允许 \(\eta_{\mathcal{P}}\ge \eta_{\text{th}}\) 且 \(H\le H_{\max}\) 的路径,以抑制极低效路径并保障全局能效与公平。
\end{itemize}
在能量守恒下,节点能量演化为
\[
E_i(t+\Delta t)=\min\!\Big\{C_i,\ E_i(t)-E_{\text{sen},i}(t)-E_{\text{com},i}(t)-E_{\text{send},i}(t)+E_{h,i}(t)+E_{\text{recv},i}(t)\Big\},
\]
其中 \(C_i\) 为电池容量,\(E_{\text{sen},i}\) 为感知/计算能耗,\(E_{\text{com},i}\) 为通信能耗。上述三层(结构/供给/链路)与能量演化共同定义机制运行的语义空间与外生扰动,为后续决策与评估提供统一背景与可检验假设。

\subsection{Economic Interpretation and AOEI as Price Signal}
经济学解释围绕"价格信号—数字账户—交易规则"的结构展开。核心价格信号采用 AOEI 描述信息新鲜度(越新越有价值),用于触发与排序;在经济学层面与具体实现中,可结合\emph{信息价值}(由 AOEI 的时效性维度与\emph{信息量}维度共同构成)作为参考,以更精细地表达"支付意愿"与时机强度。

\textbf{(1)AOEI 作为价格信号:}信息年龄\(\text{AOEI}_i(t)\)定义为信息到达物理中心后的时间:
\begin{equation}
\text{AOEI}_i(t) = t - t_{\text{arrival},i}
\end{equation}
其经济学含义对应于\emph{资产折旧理论}(会计学)或\emph{商品保质期模型}(供应链管理):节点\(i\)的能量状态信息随时间"过时",基于陈旧信息的决策质量以\(Q(\text{AOEI})=Q_0 e^{-\lambda \text{AOEI}}\)速率衰减。当\(\text{AOEI}_i(t)>T_{\text{stale}}\)(定义陈旧阈值\(T_{\text{stale}}=60\)分钟)时,该节点以\(p=\mathbb{P}(E_i<\theta_i^{\text{low}}\mid \text{AOEI}_i>T_{\text{stale}})>0.4\)的概率处于低能状态却未及时触发传能,导致信息经济学中的"逆向选择"(Adverse Selection);因此,应提高价格信号(优先级权重)以吸引供能节点。信息价值的时间衰减可建模为:
\begin{equation}
V_{\text{info},i}(t) = V_0 \cdot e^{-\lambda \cdot \text{AOEI}_i(t)}
\end{equation}
其中\(\lambda\)为衰减率(类似金融学的折现率),\(V_0\)为新鲜状态下的基础价值。在需要时,信息量可作为放大因子或加法收益项并入"信息价值"指标。在决策函数中,AOEI 以惩罚项形式内生化:
\begin{equation}
C_{\text{AOEI},i}(\tau) = w_{\text{aoei}} \cdot \tau \cdot Q_i
\end{equation}
其中\(\tau\)为传输时长(导致 AOEI 增量),\(Q_i\)为节点\(i\)的能量虚拟队列长度(需求紧迫度),\(w_{\text{aoei}}\)为权重系数(默认0.1)。

\textbf{(2)InfoNode 数字账户:}维护三级缓存架构:
\begin{itemize}
    \item \emph{L1-最新状态表}:维护能量、AOEI、位置、是否太阳能等关键字段,支持快速查询。
    \item \emph{L2-近期历史}:固定大小1000的FIFO队列,用于趋势分析与异常检测。
    \item \emph{L3-长期归档}:CSV批量写入,支持离线分析与模型训练。
\end{itemize}
该架构提供"透明市场视图",弱化信息不对称,避免道德风险。

\textbf{(3)机会主义信息上报:}在传能路径上"搭载"状态更新,结合信息去重(基于源节点集合)与动态等待上限\(T_{\max}/(1+\text{info\_volume}/\text{scale})\),以降低通信能耗与冗余为主要目标,并通过按需聚合控制上报频率(不以直接提升新鲜度为目标);透明性主要由 InfoNode 提供。

上述三者相互耦合:价格信号决定分配优先级与强度,数字账户保障状态可得,交易规则降低获取状态的边际成本;辅以能量传输专用路由(如 EETOR)的效率阈值与多跳抑制,可在局部理性前提下导出全局可解释的资源再配置结果。

\subsubsection{博弈论视角:能量共享的合作博弈与机制设计}

本文将多节点能量共享建模为\emph{合作博弈},每个节点在"保守能量"与"参与共享"之间进行策略选择。

\textbf{博弈模型:}设局中人集合为\(\mathcal{N} = \{1,2,\ldots,N\}\)。节点\(i\)的策略空间包括:(i)作为需求方,通过 AOEI 信号表达"支付意愿"\(p_i(t) = f(\text{AOEI}_i, E_i, \theta_i)\);(ii)作为供给方,决定是否响应传能请求及传输时长\(\tau_i \in [0, \tau_{\max}]\)。收益函数为:
\begin{equation}
u_i(E_i, E_{-i}, \tau) = \alpha_1 E_{\text{received},i} - \alpha_2 E_{\text{sent},i} - \alpha_3 \text{AOEI}_i \cdot \tau
\end{equation}
其中\(E_{\text{received},i}\)为节点\(i\)接收的净能量,\(E_{\text{sent},i}\)为作为供能方发送的能量(含损耗),\(\text{AOEI}_i \cdot \tau\)为传能时长导致的信息过时成本。

\textbf{准纳什均衡:}在 AOEI 价格信号与弱势保护机制下,系统存在\emph{准纳什均衡}(Quasi-Nash Equilibrium),满足:
\begin{enumerate}[label=(\roman*), leftmargin=1.6em]
    \item \emph{个体理性}(Individual Rationality):每个节点的收益不低于不参与合作的收益;
    \item \emph{预算平衡}(Budget Balance):系统总能量守恒;
    \item \emph{激励相容}(Incentive Compatibility):节点通过 InfoNode 如实报告能量状态是其最优策略(因信息透明化消除了隐藏信息的收益)。
\end{enumerate}

\textbf{帕累托改进与边界外移:}传统静态均衡下,系统位于固定的帕累托前沿 \(\mathcal{F}_{\text{static}}\)。本文通过三类机制性改造实现\emph{动态帕累托边界外移}:
\begin{enumerate}[label=(\alph*), leftmargin=1.6em]
    \item 价格内生化:将外生的"谁需要能量"转化为内生的市场定价(AOEI 驱动),减少信息不对称;
    \item 信息透明化:InfoNode 数字孪生降低逆向选择与道德风险;
    \item 外部性治理:能量传输专用路由的效率阈值与跳数限制抑制低效多跳的负外部性。
\end{enumerate}
形式化地,可达解集满足 \(\mathcal{F}_{\text{dynamic}} \supset \mathcal{F}_{\text{static}}\),即在相同资源约束下,存在 \((\eta, \phi) \in \mathcal{F}_{\text{dynamic}}\) 使得效率 \(\eta\) 与公平度 \(\phi\) 同时优于静态基线。该主张的实证验证见第五章。

\subsection{Objectives and Constraints}
系统目标与约束定义如下:
\begin{itemize}
    \item \emph{寿命目标:}最大化首个节点死亡时间,体现系统持续服务能力;
    \item \emph{均衡与公平:}降低能量方差或变异系数(CV),对低能节点实施优先保护;
    \item \emph{效率目标:}提升有效接收能量占比,减少路径与链路的无效损耗;
    \item \emph{时效目标:}提升关键状态的新鲜度,降低决策滞后带来的系统性低效;
    \item \emph{约束条件:}效率阈值、最大跳数、预算与调度频率等策略层与系统层限制。
\end{itemize}
从帕累托视角看,静态边界对应于在既定约束集合下的最优效率—公平权衡;通过将 AOEI 价格信号与数字账户、交易规则制度化内生引入,可对可行解空间实现外推,即在相同资源与约束条件下获得更优的目标组合(动态边界外移)。本文在实验部分将以多指标对标与消融试验展示该外推效应。

