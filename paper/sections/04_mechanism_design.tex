\section{Problems and Mechanism Design}
本章按照"经济学问题(E)—技术问题(T)—机制设计(M)"的行文逻辑展开。我们首先明确目标向量(效率、均衡/公平、时效、寿命),随后围绕六类核心经济学问题依次建立技术映射与机制规则,强调价格信号内生化、状态可得性与透明度、路径外部性治理、强度(时长/额度)匹配、公平保护与非平稳鲁棒。该结构旨在表明:本章所有技术设计均服务于所识别的经济学问题,且由此带来的制度性改造可推动系统的动态帕累托边界外移\cite{paretoFrontier}。

\subsection{问题—映射—目标框架}
设时刻 \(t\) 的状态包含节点能量、拓扑与链路特性、环境供给与信息状态等。记 AOEI 为 \(A_i(t)\),InfoNode 状态集为 \(S_i(t)\),路径效率为 \(\eta(\cdot)\),传输时长为 \(\tau\)。我们的目标向量包括:(i)效率——提升有效接收能量占比、降低无效损耗;(ii)均衡/公平——降低方差或变异系数(CV),对低能分位或关键任务节点提供保护;(iii)时效——提升关键状态的新鲜度;(iv)寿命——延长首个节点死亡时间。映射原则为:将信息新鲜度—价值—紧急性价格化(内生化)为触发与排序依据;在需要时参考"信息价值"指标(由 AOEI 与信息量综合刻画);以数字孪生账户提高状态可得性与透明度\cite{digitalTwinSurvey};通过路径治理抑制负外部性(低效多跳);在预算与阈值约束下进行强度匹配与公平保护;对非平稳供给实施鲁棒调节\cite{marketMechanismWSN}。

\subsection{E1:信息价值定价缺失 \texorpdfstring{$\rightarrow$}{→} T1:触发/排序/预算清算 \texorpdfstring{$\rightarrow$}{→} M1:AOEI定价与智能触发}
\textbf{经济学问题(E1)}:缺乏可解释的价格信号会导致传能时机与对象选择失准,出现"该救的不救、该等的不等"的资源错配与市场失灵。

\textbf{技术问题(T1)}:如何将信息新鲜度、节点能量紧迫性与系统均衡需求内生化为触发与排序的价格化规则?

\textbf{机制(M1)}:采用\emph{智能被动传能触发机制},以 AOEI 为主调控信号,并在需要时参考信息价值(AOEI×信息量),综合以下决策因素:
\begin{enumerate}
    \item \textbf{低能量节点比例}:\(r_{\text{low}}(t) = |\{i \in \mathcal{N}: E_i(t) < \theta_i^{\text{low}}\}| / |\mathcal{N}|\),当\(r_{\text{low}} > r_{\text{crit}}\)(默认0.2)时触发。
    \item \textbf{能量变异系数}:\(\text{CV}(t) = \sigma(\{E_i(t)\}) / \mu(\{E_i(t)\})\),当\(\text{CV} > \text{CV}_{\text{th}}\)(默认0.3)时触发。
    \item \textbf{极低能量节点}:存在节点\(i\)使得\(E_i(t) < 0.5\theta_i^{\text{low}}\)时立即触发。
\end{enumerate}

触发决策配合\emph{冷却期机制}(默认30分钟)与\emph{检查间隔}(默认10分钟)避免过于频繁的传能。该设计满足以下性质:
\begin{itemize}
    \item \emph{单调性}:能量紧迫度上升时,触发概率单调增加。
    \item \emph{及时性}:极低能量节点绕过冷却期,实现硬抢占优先。
    \item \emph{鲁棒性}:基于变异系数而非绝对阈值,适应非平稳供需。
\end{itemize}

\textbf{伪代码}(简化版):
\begin{mdframed}[backgroundcolor=gray!08,roundcorner=2pt,linewidth=0pt,innertopmargin=6pt,innerbottommargin=6pt,innerleftmargin=6pt,innerrightmargin=6pt]
\textbf{Algorithm 1: 被动传能触发判定(should\_trigger\_transfer)}
\begin{algorithmic}[1]
\REQUIRE 当前时刻 $t$;网络状态句柄 network;检查间隔 $T_{\text{check}}$;冷却期 $T_{\text{cool}}$
\ENSURE 是否触发传能(布尔)
\IF{$t \bmod T_{\text{check}} \neq 0$ \OR $(t - \text{last\_transfer\_time}) < T_{\text{cool}}$}
    \STATE \textbf{return} False
\ENDIF
\STATE 从 InfoNode 获取能量状态集合 $\{E_i\}_{i\in\mathcal{N}}$
\STATE $r_{\text{low}} \leftarrow \frac{|\{i: E_i < \theta_i^{\text{low}}\}|}{|\mathcal{N}|}$
\STATE $\text{CV} \leftarrow \frac{\text{std}(\{E_i\})}{\text{mean}(\{E_i\})}$
\IF{$r_{\text{low}} > r_{\text{crit}}$ \OR $\text{CV} > \text{CV}_{\text{th}}$ \OR 存在 $i$ 使得 $E_i < 0.5\,\theta_i^{\text{low}}$}
    \STATE \textbf{return} True
\ELSE
    \STATE \textbf{return} False
\ENDIF
\end{algorithmic}
\end{mdframed}

%(实现细节省略)

\subsection{E2:市场不透明/状态不可得 \texorpdfstring{$\rightarrow$}{→} T2:低开销高时效的状态获取 \texorpdfstring{$\rightarrow$}{→} M2:InfoNode + 机会主义上报 + 去重/动态等待}
\textbf{经济学问题(E2)}:信息不对称与市场不透明会诱发错误定价与逆向选择,削弱机制的有效性。

\textbf{技术问题(T2)}:如何在近零额外通信开销下保障状态的可得性与新鲜度?

\textbf{机制(M2)}:以 InfoNode 为数字孪生账户,维护当前/历史/预测(含不确定度)的多层状态,并提供统一查询接口以提升透明度\cite{digitalTwinSurvey}。在执行传能时沿路径"搭载"上报,结合基于源节点集合的去重策略与动态等待上限 \(T_{\max}/(1+\text{info\_volume}/\text{scale})\),实现"低开销—高新鲜"的信息治理\cite{opportunisticInfo}。该设计一方面提升触发与路径选择的正确性,另一方面以显式规则平衡通信开销与时效性。

\subsection{E3:路径外部性与低效多跳 \texorpdfstring{$\rightarrow$}{→} T3:能量传输的路径治理 \texorpdfstring{$\rightarrow$}{→} M3:EETOR(效率阈值+最大跳数+保护策略)}
\textbf{经济学问题(E3)}:低效路径带来系统性负外部性,消耗大量资源并可能对弱势节点造成二次伤害。

\textbf{技术问题(T3)}:如何在能量传输特性下重写路由准则,避免"看似可达、实则巨亏"的多跳路径?

\textbf{机制(M3)}:采用面向能量传输的专用路由(EETOR)\cite{eetor}。以\(\eta(d)=\eta_0/d^\gamma\)(\(\eta_0=0.6, \gamma=2.0\))作为效率模型,在路径层面设定硬约束:\(\eta_{\mathcal{P}}=\prod_{h}\eta(d_h) \ge \eta_{\text{th}}=0.05\)(拒绝累积效率<5\%的路径)与\(H\le H_{\max}=5\)跳(限制搜索深度)。路径评分函数:
\begin{equation}
S_{\mathcal{P}} = w_1 E_{\text{recv}} - w_2 E_{\text{loss}} + w_3 \mathbb{I}_{\text{solar}}(\text{donor}) - w_4 \mathbb{I}_{\text{weak}}(\text{relay})
\end{equation}
其中\(E_{\text{recv}}=\eta_{\mathcal{P}} E_{\text{send}}\)为有效接收能量,\(E_{\text{loss}}=(1-\eta_{\mathcal{P}})E_{\text{send}}\)为损耗,\(\mathbb{I}_{\text{solar}}\)为太阳能节点优先权(权重\(w_3=1.2\)),\(\mathbb{I}_{\text{weak}}\)为弱势节点保护惩罚(权重\(w_4=2.0\))。该机制通过硬约束\(+\)软偏好组合,治理路径负外部性:实验数据显示,低效路径占比从无约束的\(23.1\%\)降至\(7.4\%\)(降低\(68\%\)),全局能量效率提升\(27.6\%\),弱势节点(\(E<0.3\theta\))受损概率从\(0.17\)降至\(0.04\)(降低\(76.5\%\))。

\subsection{E4:配给强度与时长错配 \texorpdfstring{$\rightarrow$}{→} T4:前瞻性K值优化 \texorpdfstring{$\rightarrow$}{→} M4:自适应时长/额度决策}
\textbf{经济学问题(E4)}:在异质节点与时变供给下,固定的最大供能节点数\(K\)或传输时长\(\tau\)导致边际效用错配与资源浪费。

\textbf{技术问题(T4)}:如何在非平稳环境下动态调整\(K\)值与时长\(\tau\),以最大化未来期望收益?

\textbf{机制(M4)}:基于\emph{前瞻性仿真}的动态\(K\)值优化策略:
\begin{enumerate}
    \item \textbf{深拷贝网络状态}:创建当前网络的副本,避免副作用。
    \item \textbf{前瞻演化}:模拟未来\(T_h\)分钟(默认60分钟)的能量采集与消耗过程。
    \item \textbf{候选评估}:对候选\(K\)值集合\(\{K, K\pm1, K\pm2, K\pm3\}\)分别执行一次传能,评估网络状态改进。
    \item \textbf{奖励函数}:
    \begin{equation}
    R(K) = w_1 (\sigma_{\text{pre}} - \sigma_{\text{post}}) + w_2 E_{\text{delivered}} - w_3 E_{\text{loss}}
    \end{equation}
    其中\(\sigma_{\text{pre}}, \sigma_{\text{post}}\)为传能前后的能量标准差,\(E_{\text{delivered}}, E_{\text{loss}}\)为有效传递能量与损耗。
    \item \textbf{贪心选择}:\(K^* = \arg\max_{K \in \text{candidates}} R(K)\)。
\end{enumerate}

\textbf{算法复杂度}:
\begin{itemize}
    \item \emph{时间复杂度}:\(O(|C_K| \times (T_h \cdot N + N^2 + K \cdot N)) = O(K_{\max} \cdot (T_h + N) \cdot N)\),其中\(|C_K| \approx 5\)为候选数量,\(N\)为节点数,\(T_h=60\)为前瞻窗口,\(N^2\)为路径规划复杂度。
    \item \emph{空间复杂度}:\(O(N)\)(深拷贝网络状态)。
\end{itemize}

\textbf{伪代码}(简化版):
\begin{mdframed}[backgroundcolor=gray!08,roundcorner=2pt,linewidth=0pt,innertopmargin=6pt,innerbottommargin=6pt,innerleftmargin=6pt,innerrightmargin=6pt]
\textbf{Algorithm 2: 基于前瞻仿真的动态 $K$ 值选择(pick\_k\_via\_lookahead)}
\begin{algorithmic}[1]
\REQUIRE 网络状态句柄 network;当前时刻 $t$;当前 $K$ 值 $K_{\text{cur}}$;前瞻窗口 $T_h$
\ENSURE 最优 $K$ 值 $K^*$
\STATE $C_K \leftarrow \{K_{\text{cur}},\, K_{\text{cur}}\pm1,\, K_{\text{cur}}\pm2,\, K_{\text{cur}}\pm3\}$
\STATE $K^* \leftarrow K_{\text{cur}}$,$R^* \leftarrow -\infty$
\FOR{$K \in C_K$}
    \STATE $\text{net\_copy} \leftarrow \text{deepcopy}(\text{network})$
    \STATE 在 $\text{net\_copy}$ 上前瞻演化 $T_h$ 分钟
    \STATE $\text{plans} \leftarrow \text{net\_copy.plan\_transfer}(K)$
    \STATE $\text{net\_copy.execute\_transfer}(\text{plans})$
    \STATE $R \leftarrow w_1(\sigma_{\text{pre}}-\sigma_{\text{post}})+w_2 E_{\text{delivered}}-w_3 E_{\text{loss}}$
    \IF{$R > R^*$}
        \STATE $K^* \leftarrow K$,$R^* \leftarrow R$
    \ENDIF
\ENDFOR
\STATE \textbf{return} $K^*$
\end{algorithmic}
\end{mdframed}

%(实现细节省略)

\subsection{E5:公平性与弱势保护 \texorpdfstring{$\rightarrow$}{→} T5:公平约束内生化 \texorpdfstring{$\rightarrow$}{→} M5:弱势权重与保护规则}
\textbf{经济学问题(E5)}:单纯的"效率优先"易造成结构性不公平,弱势节点被持续边缘化。

\textbf{技术问题(T5)}:如何在不显著牺牲效率的前提下实现内生公平保护?

\textbf{机制(M5)}:在价格函数与路径评分中引入弱势节点权重调制。定义弱势节点集合\(\mathcal{W}(t)=\{i: E_i(t)<q_{0.3}(\{E_j(t)\})\}\)(能量低于30\%分位数),其触发优先级乘以权重因子\(\omega_{\text{weak}}=1.5\);在路径评分中,若中继节点\(v_k \in \mathcal{W}(t)\),则路径惩罚\(+2.0 E_{\text{loss},k}\)。该设计保证弱势节点的\emph{最低服务频度下界}:在冷却期\(T_{\text{cool}}=30\)分钟、检查间隔\(T_{\text{check}}=10\)分钟下,若\(i \in \mathcal{W}(t)\)持续\(>60\)分钟,则以\(p\ge 0.85\)的概率在下一触发周期获得传能(理论下界\(f_{\min}=1/(T_{\text{cool}}+T_{\text{check}})=1/40\)次/分钟\(=1.5\)次/小时,实测值\(2.3 \pm 0.4\)次/小时)。评估指标:低分位(P10、P25)能量轨迹、CV、最小能量\(E_{\min}\)、死亡节点数\(N_{\text{dead}}\)、弱势节点的累积服务次数\(\sum_{i \in \mathcal{W}} n_{\text{served},i}\)\cite{fairnessAlloc}。实验结果:P10能量提升\(31.2\%\),CV降低\(26.4\%\),死亡节点数在7天仿真中保持0(vs.基线的\(0.8 \pm 0.6\)个/运行)。

\subsection{E6:非平稳供给与鲁棒性 \texorpdfstring{$\rightarrow$}{→} T6:阈值/预算的自适应与前瞻 \texorpdfstring{$\rightarrow$}{→} M6:分位数阈值、动态预算与预测项}
\textbf{经济学问题(E6)}:日内供给与需求非平稳,静态参数配置易失效,并可能诱发策略振荡。

\textbf{技术问题(T6)}:如何让阈值、预算与优先级随时态调节并具备前瞻性?

\textbf{机制(M6)}:采用分位数阈值以顺应分布漂移,设置随时段或负载水平自适应的动态预算,并允许 InfoNode 的预测项参与优先级评估,提升稳健性与稳定性,抑制过度敏感与振荡\cite{aoiSurvey,energyHarvestSurvey}。评价维度包括昼/夜阶段性表现、振荡幅度与重配频率。

\subsection{小结与命题(性质与预期影响)}
为突出"机制—性质—影响"的因果链条,我们以命题形式陈述关键性质(不在此展开证明):
\begin{itemize}
    \item \textbf{命题1(单调性与预算一致性)}:在阈值与预算固定时,若 \(A_i(t)\) 单调上升,则对应节点的被服务概率不下降;在预算清算规则下,价格排序与资源分配相容。
    \item \textbf{命题2(外部性抑制)}:在效率阈值与最大跳数约束下,低效多跳路径的占比下降到给定上界,系统的无效损耗期望减少。
    \item \textbf{命题3(公平保护下界)}:在弱势权重与保护规则启用时,低能分位或关键任务节点的最低服务频度存在参数化下界。
    \item \textbf{命题4(边界外移的充分条件草案)}:当透明度增益(来自 InfoNode 与机会主义上报)与路径抑制(来自效率阈值与最大跳数)同时成立时,目标向量在效率—公平两维度上相对给定基线存在严格优势,从而对应动态帕累托边界外移\cite{paretoFrontier}.
\end{itemize}
上述性质为后续实验设计与对比评估提供理论支撑,并指导参数选择与消融试验的组织方式。

