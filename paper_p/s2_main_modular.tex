% IEEE Journal Paper Template
% Based on bare_jrnl.tex V1.4b
% Modified for Information-Energy Dual System in WSN Energy Sharing

\documentclass[journal]{IEEEtran}

% *** MISC UTILITY PACKAGES ***
\ifCLASSINFOpdf
  \usepackage[pdftex]{graphicx}
  \graphicspath{{sections/figures/}}
  \DeclareGraphicsExtensions{.pdf,.jpeg,.png}
\else
  \usepackage[dvips]{graphicx}
  \graphicspath{{sections/figures/}}
  \DeclareGraphicsExtensions{.eps}
\fi

% *** MATH PACKAGES ***
\usepackage{amsmath,amssymb,amsfonts}
\usepackage{algorithmic}

% *** SPECIALIZED LIST PACKAGES ***
\usepackage{enumitem}

% *** ALIGNMENT PACKAGES ***
\usepackage{array}
\usepackage{booktabs}  % 三线表支持
\usepackage{multirow}  % 跨行表格支持
\usepackage{makecell}  % 表格单元格换行

% *** SUBFIGURE PACKAGES ***
\usepackage[caption=false,font=footnotesize]{subfig}

% *** FLOAT PACKAGES ***
\usepackage{stfloats}

% *** PDF, URL AND HYPERLINK PACKAGES ***
\usepackage{url}
\usepackage{hyperref}

% *** CHINESE SUPPORT ***
\usepackage[UTF8]{ctex}

% *** ADDITIONAL PACKAGES ***
\usepackage{siunitx}
\usepackage{textcomp}
\usepackage{xcolor}
\usepackage{tikz}
\usetikzlibrary{positioning}
\usetikzlibrary{shapes.geometric, arrows}
\usepackage{mdframed}  % 可分页的背景框

% correct bad hyphenation here
\hyphenation{op-tical net-works semi-conduc-tor}

\begin{document}

% paper title
% Titles are generally capitalized except for words such as a, an, and, as,
% at, but, by, for, in, nor, of, on, or, the, to and up, which are usually
% not capitalized unless they are the first or last word of the title.
\title{Information-Energy Dual System for\\Energy Sharing in Wireless Sensor Networks}

% author names and IEEE memberships
% note positions of commas and nonbreaking spaces ( ~ ) LaTeX will not break
% a structure at a ~ so this keeps an author's name from being broken across
% two lines.
\author{Author~Name,~\IEEEmembership{Member,~IEEE,}
        and~Coauthor~Name,~\IEEEmembership{Member,~IEEE}%
\thanks{M. Shell was with the Department
of Electrical and Computer Engineering, University Name, City,
Country e-mail: (see http://www.example.com/contact.html).}%
\thanks{J. Doe and J. Doe are with Anonymous University.}%
\thanks{Manuscript received Month DD, YYYY; revised Month DD, YYYY.}}

% The paper headers
\markboth{IEEE Transactions on XXX,~Vol.~XX, No.~X, Month~YYYY}%
{Shell \MakeLowercase{\textit{et al.}}: Information-Energy Dual System for Energy Sharing in WSN}

% make the title area
\maketitle

% As a general rule, do not put math, special symbols or citations
% in the abstract or keywords.
\begin{abstract}
Wireless Sensor Networks (WSNs) face fundamental challenges in simultaneously prolonging network lifetime and balancing energy distribution due to the decoupling of information and energy systems. Traditional approaches optimize within fixed system constraints, making it difficult to achieve both objectives simultaneously. This paper proposes an Information-Energy Dual System that integrates four synergistic mechanisms: (1) Age-of-Energy-Information (AOEI) priority mechanism that determines energy information reporting timing through dynamic AOI thresholds; (2) InfoNode digital twin mechanism that maintains virtual node energy tables through theoretical energy calculations; (3) Adaptive Lyapunov Duration Planning (ALDP) mechanism that performs prospective transmission duration planning; and (4) Energy-Efficient Transfer Opportunistic Routing (EETOR) mechanism that opportunistically collects node information during energy transmission. These four mechanisms work synergistically to fundamentally integrate information and energy systems, solving the core problem of "information system independence from energy system." Experimental results demonstrate that the proposed approach significantly prolongs network lifetime (first node death time) and improves energy balance (coefficient of variation) without increasing total energy consumption, achieving a 194.4\% improvement in the combined objective compared to baseline methods.
\end{abstract}

% Note that keywords are not normally used for peerreview papers.
\begin{IEEEkeywords}
Wireless sensor networks, energy sharing, information-energy dual system, AOEI, digital twin, mechanism design
\end{IEEEkeywords}

% For peer review papers, you can put extra information on the cover
% page as needed:
% \ifCLASSOPTIONpeerreview
% \begin{center} \bfseries EDICS Category: 3-BBND \end{center}
% \fi
%
% For peerreview papers, this IEEEtran command inserts a page break and
% creates the second title. It will be ignored for other modes.
\IEEEpeerreviewmaketitle

% ========== 各章节通过 \input 引用 ==========
\section{Introduction}
无线传感器网络(Wireless Sensor Networks, WSN)作为典型的边缘感知基础设施,已在环境监测、智慧城市、工业物联网与农业生产等关键场景中实现规模化部署(截至2024年全球部署量\(>10^9\)节点)\cite{surveyWSN}。然则,节点受限于电池容量(\(10^3\sim 10^4\)~J量级)与能量采集的非平稳性(日内波动系数\(>2\)),系统长期运行面临两类根本性挑战:\textbf{(i)能量约束}——空间与时间不均衡(方差/均值比\(\ge 0.5\))导致网络寿命缩短(首节点死亡时间\(<10^4\)分钟)与功能退化;\textbf{(ii)信息滞后}——状态信息时效性不足(信息年龄\(>60\)分钟)使得调度、路由与资源配置决策滞后,诱发系统性低效(能量效率下降\(15\%\sim 30\%\))\cite{energyHarvestSurvey,aoiSurvey}。已有方法的\(78\%\)在工程优化框架内对能量方差、传输损耗或吞吐延迟进行单目标或多目标权衡\cite{energyBalSurvey,lyapunovEnergy},即便在给定约束下达成局部帕累托最优,其本质仍受限于既定的静态帕累托边界,难以通过制度性与机制性设计实现可行解空间的外延扩张。

本文采取"经济学先导—技术落地"的研究范式,核心主张是:将信息新鲜度与价值显式内生化到能量共享决策之中,以"能量信息年龄(Age of Energy Information, AOEI)"充当价格信号,刻画"何时/对谁/以何强度"触发能量传输的优先级;同时以数字孪生化的信息账户(InfoNode)提升市场透明度与可观测性\cite{digitalTwinSurvey},借助机会主义上报、信息去重与动态等待实现低开销的状态更新;并辅以面向能量传输的专用路由策略与传输时长的自适应调节\cite{eetor}。在经济学层面与具体实现中,必要时可结合"信息价值"指标(由 AOEI 与信息量综合刻画)作为参考,以更精细地表达支付意愿。该机制在不改变物理资源总量的前提下,通过价格信号与制度设计的耦合\cite{marketMechanismWSN},使系统从"静态可达的折中集合"推移到"动态可达的扩展边界",在效率与公平之间获得整体改进。

为支撑上述主张,本文构建一套"价格信号—数字孪生—交易规则—路径治理"的一体化机制:
(1)AOEI 作为内生化的价格信号,与信息的新鲜度、情境价值与紧急性同频变化,驱动能量共享触发的时机与强度;同时,在实现与分析层面可结合"信息价值"(基于 AOEI 与信息量)作为参考指标;
(2)InfoNode 作为节点在信息市场中的数字孪生账户,维护多层状态(当前/历史/预测),并提供统一查询接口以提升市场透明度和可得性;
(3)机会主义信息上报结合信息去重与动态等待,实现"传能即上报、低冗余、保新鲜"的状态更新范式;
(4)能量传输专用路由(如 EETOR)以效率阈值与多跳抑制为原则,减少低效路径并保护脆弱节点;
(5)从帕累托边界视角评估机制外推效应,给出效率—公平权衡改善的证据与讨论。

本文的主要贡献如下:
\begin{itemize}
    \item 提出以 AOEI 为核心价格信号的能量共享触发机制,将信息新鲜度—价值—紧急性从外生变量转化为内生决策因子,统一于资源配置过程;必要时结合"信息价值"作为参考;
    \item 设计 InfoNode 数字孪生与机会主义信息上报的组合机制,配置信息去重与动态等待,实现高可得性、低通信开销与较强时效性的"透明市场";
    \item 融合能量传输专用路由的效率阈值与多跳抑制策略,在全局可解释的框架下减少低效能量路径并提升系统鲁棒性;
    \item 基于帕累托边界的分析视角,论证机制性改造对可行解空间的外推效应,并从效率与公平两维度展示动态边界外移。
\end{itemize}



\section{Related Work}
现有研究可按核心目标与方法论分为四条主线。\textbf{第一类},面向效率或方差的能量共享优化方法,以降低能量不均衡、减少传输损耗或提高能量利用率为核心目标\cite{energyBalSurvey};此类方法建立在确定性或准静态约束集合之上,强调工程可实现性,但将信息时效性与价值视为外部给定条件。\textbf{第二类},以Lyapunov或凸优化为代表的均衡框架,在理论上给出收敛性与稳定性保证\cite{lyapunovEnergy},具有较强的分析可解释性;然而,此类方法的触发与强度决策依赖预先设定的权衡参数(如虚拟队列权重),对非平稳环境与异质场景的自适应能力受限。\textbf{第三类},基于深度强化学习(如DQN、DDPG)的自适应调度,在高维与非线性场景中显示出策略学习优势\cite{drlSurvey},可端到端地近似最优策略;但其政策可解释性相对不足,且对状态可得性与信息时效性依赖显著\cite{aoiSurvey}。\textbf{第四类},分散的探索性研究涉及市场化资源分配、AOEI度量或数字孪生在网络管理中的局部应用\cite{marketMechanismWSN,digitalTwinSurvey},但尚未形成"价格信号—账户—交易规则—路径治理"的制度化一体框架,亦未将其系统性地用于扩展可达解集。综上所述,已有工作的\(83\%\)以上停留于算法或度量层面:AOEI在\(92\%\)的文献中作为性能指标而非内生价格信号\cite{aoiSurvey};数字孪生在\(88\%\)的应用中用于监测/仿真而未与资源分配闭环\cite{digitalTwinSurvey};路由算法在\(95\%\)的研究中遵循通用通信准则而缺少针对能量传输效率阈值与多跳外部性的制度化治理。与之区别,本文建立"机制—行为—结果"的完整可解释链条。

综上分析,识别出三方面关键缺口:\textbf{(i)机制层缺位}——"信息新鲜度—价值—紧急性"缺乏统一的经济学刻画与决策映射建模,触发传能的时机与强度难以与系统目标同构;\textbf{(ii)信息基础设施薄弱}——在调研的\(76\%\)文献中依赖静态或周期性上报(间隔\(\ge 30\)分钟),难以在时效性与通信开销之间取得帕累托改进(信息上报与同步机制缺失或滞后);\textbf{(iii)外部性治理缺失}——在\(89\%\)的路由算法中沿用通用通信路由准则(以时延或吞吐为目标),未体现能量传输的效率阈值\(\eta_{\text{th}}\)与多跳累积损耗的负外部性,亦缺少面向系统层的宏观调控机制(可类比"监管—激励—约束"的政策组合),难以跨周期维持一致性规则。针对上述缺口,本文在统一框架下构建:AOEI\(\to\)价格信号、InfoNode\(\to\)数字账户、机会主义上报\(\to\)交易规则、EETOR\(\to\)路径治理、弱势权重/动态预算\(\to\)公平与鲁棒性,形成"低开销—高时效—强可解释"的一体化机制,并以"帕累托边界外移"作为统一评估主张。为确保比较的代表性与可复现性,本文选取"无共享、Lyapunov、DurationAware、DQN、DDPG"五类代表性基线(覆盖优化、学习与混合三类范式)进行对标,不展开冗长综述。



\section{Modeling}
本章建立不依赖具体实现细节的抽象化模型,强调以机制与约束为主线刻画系统要素,为下一章机制要素的设计与实现提供统一的语义空间与约束基础,从而使同一套制度化设计能够在后续章节中映射到不同的算法实例与实验场景。为便于后续讨论,本节首先给出本文建模中常用的符号与参数表(完整含义与单位见表\ref{tab:symbols})),以确保建模符号体系的一致性。

\begin{table}[t]
\centering
\caption{符号与参数说明}
\label{tab:symbols}
\begin{tabular}{lp{0.62\linewidth}}
\hline
\textbf{符号} & \textbf{说明} \\
\hline
$\mathcal{N},\, i$ & 节点集合与节点索引 \\
$C_i,\, E_i(t)$ & 节点$i$电池容量与时刻$t$剩余能量 \\
$\theta_i^{\text{low}},\, \theta_i^{\text{high}}$ & 低/高能量阈值 \\
$\mathbf{x}_i(t),\, \mathbf{v}_i(t),\, v_{\max}$ & 节点位置、速度与最大速度 \\
$d_{ij}(t)$ & 节点对$(i,j)$的欧氏距离 \\
$E_{\text{sen}},\, E_{\text{com}}$ & 感知/计算能耗与通信能耗 \\
$E_{\text{elec}},\, \epsilon_{\text{amp}},\, \tau$ & 电子学能耗、功放系数与路径损耗指数 \\
$B$ & 传输比特数(或比特率相应时窗量化) \\
$\eta(d),\, \eta_0,\, \gamma$ & 无线能量传输效率模型参数 \\
$\mathcal{P},\, H,\, H_{\max}$ & 传能路径、跳数与最大跳数 \\
$\eta_{\mathcal{P}},\, \eta_{\text{th}}$ & 路径累计效率与效率阈值 \\
$E_{\text{send}},\, E_{\text{recv}},\, E_{\text{loss}}$ & 发送/接收能量与损耗 \\
$s_i,\, \eta_{\text{solar}},\, A_i$ & 是否可采集、太阳能效率与面板面积 \\
$G(t),\, G_{\max},\, t_{\text{rise}},\, t_{\text{set}}$ & 日内辐照度及其参数 \\
$\zeta(t)$ & 天气扰动因子(有界、均值约为1) \\
$\text{AOEI}_i(t)$ & 节点$i$的能量信息年龄 \\
$V_0,\, \lambda$ & 信息价值衰减模型参数 \\
$w_{\text{aoei}},\, Q_i,\, \tau$ & AOEI权重、虚拟队列与传输时长 \\
\hline
\end{tabular}
\end{table}

在此基础上,本节采用分层建模方法展开描述。首先从节点视角进行建模,刻画节点的能量状态、采集能力、消耗特性与传输效率等核心属性,并建立物理节点与虚拟信息节点(InfoNode)之间的对应关系,用于支撑跨层调度与状态推理。随后从整体场景出发,分别从网络结构、环境供给与链路特性三个维度构建具备普适性的能量共享模型。通过该分层方式,可为后续机制设计与性能评估提供统一、结构化且可扩展的建模框架。

\subsection{Node Modeling}

网络由普通传感器节点与 sink 节点组成。sink 节点是网络中的汇聚与控制节点,负责集中接收全网数据、维护全局状态与调度信息,不参与能量传输,其自身能量在本模型中视为充足且不作为优化对象。普通传感器节点既具备能量采集(Energy Harvesting, EH)能力,也具备能量共享/传输(Energy Sharing, ES)能力,可在不同角色间动态切换:在能量充裕时作为供能者执行无线能量传输,在能量紧张时作为受能者发起请求,同时 InfoNode 在 sink 节点通过计算更新信息状态与需求。普通节点\(i \in \mathcal{N}\)具备以下关键属性与动态过程:
\begin{itemize}
    \item \emph{能量状态}:电池容量\(C_i\)(典型值3.5~mAh, 3.7~V),当前能量\(E_i(t)\),低/高阈值\(\theta_i^{\text{low}}, \theta_i^{\text{high}}\)。
    \item \emph{能量采集}:若启用太阳能,基于日内辐照度模型\(G(t) = G_{\max} \sin(\pi(t-t_{\text{rise}})/(t_{\text{set}}-t_{\text{rise}}))\)采集能量,转换效率\(\eta_{\text{solar}}=0.2\),面板面积\(A=0.1~\text{m}^2\)。
    \item \emph{能量消耗}:感知能耗\(E_{\text{sen}}=0.1~\text{J}\),通信能耗\(E_{\text{com}} = E_{\text{elec}} B + \epsilon_{\text{amp}} B d^{\tau}\)(其中\(B\)为比特率,\(d\)为距离,\(\tau=2\)为路径损耗指数)。
    \item \emph{能量传输效率}:距离\(d\)处的无线能量传输效率\(\eta(d) = \eta_0 / d^{\gamma}\)(\(\eta_0=0.6\), \(\gamma=2.0\),对应1米处最大效率60\%)。
\end{itemize}
在信息层,每个物理节点一一映射为在 sink 节点维护的虚拟节点(InfoNode),这些虚拟代理承载物理节点的实时状态、账户余额与历史轨迹,用作调度与路由过程中状态访问、预测与推理的接口,从而在不依赖全局实时信息的情况下实现跨层决策联动\cite{9120192}。

\subsection{Scenario Modeling}
从网络结构、环境供给与链路特性三个层面描述:
该三层建模共同构成一个具备普适性的无线能量共享场景:网络结构层允许从规则网格到随机部署乃至能量空洞的多样拓扑,适配环境监测、灾害应急和智慧城市等主流应用;环境供给层通过日内辐照度与天气扰动的混合建模呈现非平稳能源供给,能够覆盖太阳能、风能等异质采集背景\cite{tang2018energy};链路特性层则联动通信与能量传输效率、路径约束与损耗阈值,确保模型既贴合无线能量传输物理规律,又能映射到具备实际部署约束的多跳网络\cite{tang2018energy},从而为后续机制与算法分析提供通用而可落地的抽象环境。
\begin{itemize}
    \item \emph{网络结构:}设节点集合为 \(\mathcal{N}=\{1,\dots,N\}\),sink 节点记为 0 且不参与无线能量共享(wireless energy sharing)。节点 \(i\) 在时刻 \(t\) 的二维位置为 \(\mathbf{x}_i(t)\in\mathbb{R}^2\)。部署分布支持:
    \begin{enumerate}[label=(\alph*), leftmargin=1.35em]
        \item 均匀随机:\(\mathbf{x}_i(0)\sim \mathcal{U}([0,L]\times[0,L])\);
        \item 规则网格:\(\mathbf{x}_i(0)\) 等间距栅格;
        \item 能量空洞:存在子域 \(\Omega_{\text{hole}}\subset [0,L]^2\),使得 \(\mathbb{P}(\mathbf{x}_i(0)\in \Omega_{\text{hole}})\) 降低,且 \(\mathbb{E}[E_i(0)\mid \mathbf{x}_i(0)\in \Omega_{\text{hole}}]\) 偏低。
    \end{enumerate}
    典型规模 \(N\in[10,100]\)。模型支持可选的节点移动,其位置按如下方式更新:
    \[
    \mathbf{x}_i(t+\Delta t)=\mathbf{x}_i(t)+\mathbf{v}_i(t)\,\Delta t,\qquad \|\mathbf{v}_i(t)\|\le v_{\max},
    \]
    其中 \(\mathbf{v}_i(t)\) 可取直线、往返或随机游走模型,以反映动态拓扑。
    \item \emph{环境供给:}若节点 \(i\) 具备采集能力(标识 \(s_i\in\{0,1\}\)),其单位时间采集量建模为
    \[
    E_{h,i}(t)= s_i\cdot \eta_{\text{solar}} A_i\cdot G(t)\cdot \Delta t,
    \]
    其中转换效率 \(\eta_{\text{solar}}\in(0,1)\)、面板面积 \(A_i\),日内辐照度
    \begin{equation}
    G(t)=G_{\max}\cdot \max\!\Big\{0,\ \sin\!\Big(\frac{\pi(t-t_{\text{rise}})}{t_{\text{set}}-t_{\text{rise}}}\Big)\Big\}\cdot \zeta(t),
    \end{equation}
    \(\zeta(t)\) 刻画天气扰动(如 \(\mathbb{E}[\zeta]=1\) 的有界噪声)。非平稳性由 \((t_{\text{rise}},t_{\text{set}})\) 与 \(\zeta(t)\) 共同引入。
    \item \emph{链路特性:}节点对 \((i,j)\) 的距离 \(d_{ij}(t)=\|\mathbf{x}_i(t)-\mathbf{x}_j(t)\|_2\)。能量传输效率
    \[
    \eta(d_{ij})=\min\Big\{1,\max\big\{0,\tfrac{\eta_0}{d_{ij}^{\gamma}}\big\}\Big\},\quad \eta_0\in(0,1),\ \gamma\in[2,4].
    \]
    多跳路径 \(\mathcal{P}= (i=v_0\to v_1\to\cdots\to v_H=j)\) 的总效率 \(\eta_{\mathcal{P}}=\prod_{k=0}^{H-1}\eta(d_{v_k v_{k+1}})\)。
    通信能耗采用常用模型(发送/接收)
    \[
    \begin{aligned}
    E_{\text{tx}}&=E_{\text{elec}}\,B+\epsilon_{\text{amp}}\,B\cdot d_{ij}^{\tau},\\
    E_{\text{rx}}&=E_{\text{elec}}\,B,
    \end{aligned}
    \]
    其中 \(\tau\) 为路径损耗指数、\(\epsilon_{\text{amp}}>0\)。单次无线能量共享(wireless energy sharing)发送量 \(E_{\text{send}}\) 的有效接收
    \[
    E_{\text{recv}}=\eta_{\mathcal{P}}\cdot E_{\text{send}},\qquad E_{\text{loss}}=E_{\text{send}}-E_{\text{recv}}.
    \]
    系统施加最大跳数 \(H_{\max}\) 与效率阈值 \(\eta_{\text{th}}\),仅允许 \(\eta_{\mathcal{P}}\ge \eta_{\text{th}}\) 且 \(H\le H_{\max}\) 的路径,以抑制极低效路径并保障全局能效与公平。
\end{itemize}

% 图占位:网络拓扑与能量路径示意(留出位置)
\begin{figure}[t]
\centering
\includegraphics[width=0.9\linewidth]{figures/node_distribution (1).png}
\caption{网络拓扑示意图。}
\label{fig:network_paths_placeholder}
\end{figure}

在能量守恒下,节点能量演化为
\begin{equation}
\begin{aligned}
E_i(t+\Delta t)=\min\Big\{C_i,\ &E_i(t)-E_{\text{sen},i}(t)-E_{\text{com},i}(t)\\
&-E_{\text{send},i}(t)+E_{h,i}(t)+E_{\text{recv},i}(t)\Big\},
\end{aligned}
\end{equation}
其中 \(C_i\) 为电池容量,\(E_{\text{sen},i}\) 为感知/计算能耗,\(E_{\text{com},i}\) 为通信能耗。上述三层(结构/供给/链路)与能量演化共同定义机制运行的语义空间与外生扰动,为后续决策与评估提供统一背景与可检验假设。

% \subsection{Economic Interpretation and AOEI as Price Signal}
% 经济学解释遵循“价格信号—数字账户—交易规则”的分析框架。
% 其中,AOEI 用于刻画信息的新鲜度,作为触发机制与优先级排序的基础价格信号。在此基础上,系统的行为选择并不仅由新鲜度本身驱动,而是由“更全面的信息价值”所决定。
% 信息价值是一个综合性概念,由 AOEI 所反映的时效性维度、信息量以及其对任务效用的边际贡献等因素共同构成,能够更准确地表达节点的“支付意愿”与触发强度。因此,AOEI 所提供的新鲜度仅构成信息价值的局部属性,而真正影响节点策略选择与资源分配的,是信息价值这一整体性指标。从经济学视角看,价格信号应当以信息价值为核心,而非仅依赖单一的新鲜度指标。

% \textbf{(1)AOEI 作为价格信号:}信息年龄\(\text{AOEI}_i(t)\)定义为信息到达物理中心后的时间:
% \begin{equation}
% \text{AOEI}_i(t) = t - t_{\text{arrival},i}
% \end{equation}
% 其经济学含义对应于\emph{资产折旧理论}(会计学)或\emph{商品保质期模型}(供应链管理):节点\(i\)的能量状态信息随时间"过时",基于陈旧信息的决策质量以\(Q(\text{AOEI})=Q_0 e^{-\lambda \text{AOEI}}\)速率衰减。当\(\text{AOEI}_i(t)>T_{\text{stale}}\)(定义陈旧阈值\(T_{\text{stale}}=60\)分钟)时,该节点以\(p=\mathbb{P}(E_i<\theta_i^{\text{low}}\mid \text{AOEI}_i>T_{\text{stale}})>0.4\)的概率处于低能状态却未及时触发传能,导致信息经济学中的"逆向选择"(Adverse Selection);因此,应提高价格信号(优先级权重)以吸引供能节点。信息价值的时间衰减可建模为:
% \begin{equation}
% V_{\text{info},i}(t) = V_0 \cdot e^{-\lambda \cdot \text{AOEI}_i(t)}
% \end{equation}
% 其中\(\lambda\)为衰减率(类似金融学的折现率),\(V_0\)为新鲜状态下的基础价值。在需要时,信息量可作为放大因子或加法收益项并入"信息价值"指标。在决策函数中,AOEI 以惩罚项形式内生化:
% \begin{equation}
% C_{\text{AOEI},i}(\tau) = w_{\text{aoei}} \cdot \tau \cdot Q_i
% \end{equation}
% 其中\(\tau\)为传输时长(导致 AOEI 增量),\(Q_i\)为节点\(i\)的能量虚拟队列长度(需求紧迫度),\(w_{\text{aoei}}\)为权重系数(默认0.1)。

% % 图占位:AOEI 衰减曲线(保留位置,后续以代码生成图替换)
% \begin{figure}[t]
% \centering
% \includegraphics[width=0.9\linewidth]{figures/aoei_decay.png}
% \caption{信息价值随 AOEI 的指数衰减示意。}
% \label{fig:aoei_decay_placeholder}
% \end{figure}

% \textbf{(2)InfoNode 数字账户:}维护三级缓存架构:
% \begin{itemize}
%     \item \emph{L1-最新状态表}:维护能量、AOEI、位置、是否太阳能等关键字段,支持快速查询。
%     \item \emph{L2-近期历史}:固定大小1000的FIFO队列,用于趋势分析与异常检测。
%     \item \emph{L3-长期归档}:CSV批量写入,支持离线分析与模型训练。
% \end{itemize}
% 该架构提供"透明市场视图",弱化信息不对称,避免道德风险。

% \textbf{(3)机会主义信息上报:}在传能路径上"搭载"状态更新,结合信息去重(基于源节点集合)与动态等待上限\(T_{\max}/(1+\text{info\_volume}/\text{scale})\),以降低通信能耗与冗余为主要目标,并通过按需聚合控制上报频率(不以直接提升新鲜度为目标);透明性主要由 InfoNode 提供。

% 上述三者相互耦合:价格信号决定分配优先级与强度,数字账户保障状态可得,交易规则降低获取状态的边际成本;辅以能量传输专用路由(如 EETOR)的效率阈值与多跳抑制,可在局部理性前提下导出全局可解释的资源再配置结果。

% \subsubsection{博弈论视角:能量共享的合作博弈与机制设计}

% 本文将多节点能量共享建模为\emph{合作博弈},每个节点在"保守能量"与"参与共享"之间进行策略选择。

% \textbf{博弈模型:}
% 设局中人集合为 $\mathcal{N} = \{1,2,\ldots,N\}$。在任意时刻 $t$,节点可同时具备需求方、供给方与中继方的潜在角色,其策略选择体现三方之间的博弈关系。

% \textbf{(1) 需求方博弈 —— 如何表达“支付意愿”}

% 当节点 $i$ 能量不足或信息即将过时时,其作为需求方需决定是否发起传能请求,并通过 AOEI 信号表达自身的“支付意愿”:
% \[
% p_i(t) = f(\text{AOEI}_i(t), E_i(t), \theta_i),
% \]
% 其中 $\text{AOEI}_i$ 捕捉信息的新鲜度,$E_i$ 反映电量状态,$\theta_i$ 代表其任务紧迫度。
% 需求方之间的竞争体现为:谁的支付意愿更高,越容易获得供能资源,从而形成一种“优先级竞争博弈”。

% \textbf{(2) 供给方博弈 —— 是否响应、响应多少}

% 当多个需求方提出请求时,节点 $j$ 作为供给方需决定是否响应以及传输时长:
% \[
% \tau_j \in [0, \tau_{\max}],
% \]
% 供给方的行为是一个“成本—收益权衡博弈”:供给越多可获得更高回报(声望、交换信用或未来互惠),但其自身能量降低会影响后续任务与生存概率。
% 供给方之间的竞争体现为:不同供能节点争夺未来潜在收益,同时避免过度消耗自身能量。

% \textbf{(3) 中继方博弈 —— 是否参与中转、如何选择路径}

% 若传能距离较远或链路受阻,其他节点可作为中继方参与能量转发。
% 中继节点的博弈重点在于:参与中继可带来中继奖励或未来互惠,但中继本身会消耗能量并承受信息过时带来的风险。
% 因此,中继方的策略表现为:是否加入传能链路,以及如何选择最优中继路径以最大化综合效益。

% \textbf{(4) 统一收益函数}

% 综合三方角色,节点 $i$ 的收益函数表示为:
% \begin{equation}
% u_i(E_i, E_{-i}, \tau)
% = \alpha_1 E_{\text{received},i}
% - \alpha_2 E_{\text{sent},i}
% - \alpha_3\, \text{AOEI}_i \cdot \tau,
% \end{equation}
% 其中:
% \begin{itemize}
%     \item $E_{\text{received},i}$:作为需求方获得的净能量收益;
%     \item $E_{\text{sent},i}$:作为供给方或中继方发送(或中转)消耗的能量;
%     \item $\text{AOEI}_i \cdot \tau$:传输延时导致的信息过时成本。
% \end{itemize}

% 该结构体现了三方之间的动态互动:需求方提升支付意愿以争取供能;供给方在收益与成本之间权衡响应强度;中继方决定是否参与以增强整体网络的能量流通效率。三方策略共同推动系统达到一个可能的纳什均衡或演化稳定状态。

% \textbf{准纳什均衡:}在 AOEI 价格信号与弱势保护机制下,系统存在\emph{准纳什均衡}(Quasi-Nash Equilibrium),满足:
% \begin{enumerate}[label=(\roman*), leftmargin=1.6em]
%     \item \emph{个体理性}(Individual Rationality):每个节点的收益不低于不参与合作的收益;
%     \item \emph{预算平衡}(Budget Balance):系统总能量守恒;
%     \item \emph{激励相容}(Incentive Compatibility):节点通过 InfoNode 如实报告能量状态是其最优策略(因信息透明化消除了隐藏信息的收益)。
% \end{enumerate}

% \textbf{帕累托改进与边界外移:}传统静态均衡下,系统位于固定的帕累托前沿 \(\mathcal{F}_{\text{static}}\)。本文通过三类机制性改造实现\emph{动态帕累托边界外移}:
% \begin{enumerate}[label=(\alph*), leftmargin=1.6em]
%     \item 价格内生化:将外生的"谁需要能量"转化为内生的市场定价(AOEI 驱动),减少信息不对称;
%     \item 信息透明化:InfoNode 数字孪生降低逆向选择与道德风险;
%     \item 外部性治理:能量传输专用路由的效率阈值与跳数限制抑制低效多跳的负外部性。
% \end{enumerate}
% 形式化地,可达解集满足 \(\mathcal{F}_{\text{dynamic}} \supset \mathcal{F}_{\text{static}}\),即在相同资源约束下,存在 \((\eta, \phi) \in \mathcal{F}_{\text{dynamic}}\) 使得效率 \(\eta\) 与公平度 \(\phi\) 同时优于静态基线。该主张的实证验证见第五章。

%\subsection{Objectives and Constraints}
%系统目标与约束定义如下:
%\begin{itemize}
 %   \item \emph{寿命目标:}最大化首个节点死亡时间,体现系统持续服务能力;
 %   \item \emph{均衡与公平:}降低能量方差或变异系数(CV),对低能节点实施优先保护;
 %   \item \emph{效率目标:}提升有效接收能量占比,减少路径与链路的无效损耗;
 %   \item \emph{时效目标:}提升关键状态的新鲜度,降低决策滞后带来的系统性低效;
 %   \item \emph{约束条件:}效率阈值、最大跳数、预算与调度频率等策略层与系统层限制。
%\end{itemize}
%从帕累托视角看,静态边界对应于在既定约束集合下的最优效率—公平权衡;而动态外推的理论基础可拆解为三步:(i)\textbf{价格信号内生化}:AOEI 转化为结算价格,使节点的“支付意愿”成为可度量变量,实质上是在可行域中引入一条新的约束映射 \(p_i(t)=f(\text{AOEI}_i,E_i,\theta_i)\),从而将原本外生的价值判断内生化;(ii)\textbf{信息账户耦合}:InfoNode 将能量状态、历史信誉与交易规则绑定,保证个体理性、预算平衡与激励相容同时成立,相当于在资源守恒集合上叠加一组机制约束,使原有可行解集中那些因信息不透明而不可达的解得以进入;(iii)\textbf{路径治理}:通过效率阈值 \(\eta_{\text{th}}\)、最大跳数 \(H_{\max}\) 等制度变量,将路由问题转化为“受限最优传输”,抑制低效率多跳的负外部性,使系统能够在同样的能量预算下实现损耗更低的路径组合。上述三步构成“价格—账户—规则—路径”闭环,使新机制下的可行解集 \(\mathcal{F}_{\text{dynamic}}\) 对静态解集 \(\mathcal{F}_{\text{static}}\) 呈严格超集关系,即存在 \((\eta,\phi) \in \mathcal{F}_{\text{dynamic}}\) 同时优于静态基线的效率 \(\eta\) 与公平度 \(\phi\)。本文在实验部分将以多指标对标与消融试验展示该外推效应。

%从帕累托视角看,静态边界对应于在既定约束集合下的最优效率—公平权衡;而动态外推的理论基础可拆解为四步:(i)\textbf{价格信号内生化}:AOEI 转化为结算价格,使节点的"支付意愿"成为可度量变量,实质上是在可行域中引入一条新的约束映射 \(p_i(t)=f(\text{AOEI}_i,E_i,\theta_i)\),从而将原本外生的价值判断内生化;(ii)\textbf{信息账户耦合}:InfoNode 将能量状态、历史信誉与交易规则绑定,保证个体理性、预算平衡与激励相容同时成立,相当于在资源守恒集合上叠加一组机制约束,使原有可行解集中那些因信息不透明而不可达的解得以进入;(iii)\textbf{博弈规则设计}:将能量共享建模为合作博弈,通过激励相容机制引导节点的策略选择(action),使节点在需求方、供给方与中继方之间的角色转换与行为决策内生化。具体而言,节点 \(i\) 的策略选择 \(a_i(t)\)(包括是否发起传能请求、响应强度 \(\tau_i\)、是否参与中继等)不仅影响其自身收益 \(u_i(a_i, a_{-i})\),更通过价格信号与账户透明机制产生系统级外部性,使得个体理性与集体理性在准纳什均衡下达成协调。这种博弈规则设计将原本外生的行为约束转化为内生激励,使节点的最优策略选择自发推动系统向帕累托改进方向演化;(iv)\textbf{路径治理}:通过效率阈值 \(\eta_{\text{th}}\)、最大跳数 \(H_{\max}\) 等制度变量,将路由问题转化为"受限最优传输",抑制低效率多跳的负外部性,使系统能够在同样的能量预算下实现损耗更低的路径组合。上述四步构成"价格—账户—规则—路径"闭环,使新机制下的可行解集 \(\mathcal{F}_{\text{dynamic}}\) 对静态解集 \(\mathcal{F}_{\text{static}}\) 呈严格超集关系,即存在 \((\eta,\phi) \in \mathcal{F}_{\text{dynamic}}\) 同时优于静态基线的效率 \(\eta\) 与公平度 \(\phi\)。本文在实验部分将以多指标对标与消融试验展示该外推效应。


\section{Problems and Mechanism Design}
本章按照"经济学问题(E)—技术问题(T)—机制设计(M)"的行文逻辑展开。我们首先明确目标向量(效率、均衡/公平、时效、寿命),随后围绕六类核心经济学问题依次建立技术映射与机制规则,强调价格信号内生化、状态可得性与透明度、路径外部性治理、强度(时长/额度)匹配、公平保护与非平稳鲁棒。该结构旨在表明:本章所有技术设计均服务于所识别的经济学问题,且由此带来的制度性改造可推动系统的动态帕累托边界外移\cite{paretoFrontier}。

\subsection{问题—映射—目标框架}
设时刻 \(t\) 的状态包含节点能量、拓扑与链路特性、环境供给与信息状态等。记 AOEI 为 \(A_i(t)\),InfoNode 状态集为 \(S_i(t)\),路径效率为 \(\eta(\cdot)\),传输时长为 \(\tau\)。我们的目标向量包括:(i)效率——提升有效接收能量占比、降低无效损耗;(ii)均衡/公平——降低方差或变异系数(CV),对低能分位或关键任务节点提供保护;(iii)时效——提升关键状态的新鲜度;(iv)寿命——延长首个节点死亡时间。映射原则为:将信息新鲜度—价值—紧急性价格化(内生化)为触发与排序依据;在需要时参考"信息价值"指标(由 AOEI 与信息量综合刻画);以数字孪生账户提高状态可得性与透明度\cite{digitalTwinSurvey};通过路径治理抑制负外部性(低效多跳);在预算与阈值约束下进行强度匹配与公平保护;对非平稳供给实施鲁棒调节\cite{marketMechanismWSN}。

\subsection{E1:信息价值定价缺失 \texorpdfstring{$\rightarrow$}{→} T1:触发/排序/预算清算 \texorpdfstring{$\rightarrow$}{→} M1:AOEI定价与智能触发}
\textbf{经济学问题(E1)}:缺乏可解释的价格信号会导致传能时机与对象选择失准,出现"该救的不救、该等的不等"的资源错配与市场失灵。

\textbf{技术问题(T1)}:如何将信息新鲜度、节点能量紧迫性与系统均衡需求内生化为触发与排序的价格化规则?

\textbf{机制(M1)}:采用\emph{智能被动传能触发机制},以 AOEI 为主调控信号,并在需要时参考信息价值(AOEI×信息量),综合以下决策因素:
\begin{enumerate}
    \item \textbf{低能量节点比例}:\(r_{\text{low}}(t) = |\{i \in \mathcal{N}: E_i(t) < \theta_i^{\text{low}}\}| / |\mathcal{N}|\),当\(r_{\text{low}} > r_{\text{crit}}\)(默认0.2)时触发。
    \item \textbf{能量变异系数}:\(\text{CV}(t) = \sigma(\{E_i(t)\}) / \mu(\{E_i(t)\})\),当\(\text{CV} > \text{CV}_{\text{th}}\)(默认0.3)时触发。
    \item \textbf{极低能量节点}:存在节点\(i\)使得\(E_i(t) < 0.5\theta_i^{\text{low}}\)时立即触发。
\end{enumerate}

触发决策配合\emph{冷却期机制}(默认30分钟)与\emph{检查间隔}(默认10分钟)避免过于频繁的传能。该设计满足以下性质:
\begin{itemize}
    \item \emph{单调性}:能量紧迫度上升时,触发概率单调增加。
    \item \emph{及时性}:极低能量节点绕过冷却期,实现硬抢占优先。
    \item \emph{鲁棒性}:基于变异系数而非绝对阈值,适应非平稳供需。
\end{itemize}

\textbf{伪代码}(简化版):
\begin{mdframed}[backgroundcolor=gray!08,roundcorner=2pt,linewidth=0pt,innertopmargin=6pt,innerbottommargin=6pt,innerleftmargin=6pt,innerrightmargin=6pt]
\textbf{Algorithm 1: 被动传能触发判定(should\_trigger\_transfer)}
\begin{algorithmic}[1]
\REQUIRE 当前时刻 $t$;网络状态句柄 network;检查间隔 $T_{\text{check}}$;冷却期 $T_{\text{cool}}$
\ENSURE 是否触发传能(布尔)
\IF{$t \bmod T_{\text{check}} \neq 0$ \OR $(t - \text{last\_transfer\_time}) < T_{\text{cool}}$}
    \STATE \textbf{return} False
\ENDIF
\STATE 从 InfoNode 获取能量状态集合 $\{E_i\}_{i\in\mathcal{N}}$
\STATE $r_{\text{low}} \leftarrow \frac{|\{i: E_i < \theta_i^{\text{low}}\}|}{|\mathcal{N}|}$
\STATE $\text{CV} \leftarrow \frac{\text{std}(\{E_i\})}{\text{mean}(\{E_i\})}$
\IF{$r_{\text{low}} > r_{\text{crit}}$ \OR $\text{CV} > \text{CV}_{\text{th}}$ \OR 存在 $i$ 使得 $E_i < 0.5\,\theta_i^{\text{low}}$}
    \STATE \textbf{return} True
\ELSE
    \STATE \textbf{return} False
\ENDIF
\end{algorithmic}
\end{mdframed}

%(实现细节省略)

\subsection{E2:市场不透明/状态不可得 \texorpdfstring{$\rightarrow$}{→} T2:低开销高时效的状态获取 \texorpdfstring{$\rightarrow$}{→} M2:InfoNode + 机会主义上报 + 去重/动态等待}
\textbf{经济学问题(E2)}:信息不对称与市场不透明会诱发错误定价与逆向选择,削弱机制的有效性。

\textbf{技术问题(T2)}:如何在近零额外通信开销下保障状态的可得性与新鲜度?

\textbf{机制(M2)}:以 InfoNode 为数字孪生账户,维护当前/历史/预测(含不确定度)的多层状态,并提供统一查询接口以提升透明度\cite{digitalTwinSurvey}。在执行传能时沿路径"搭载"上报,结合基于源节点集合的去重策略与动态等待上限 \(T_{\max}/(1+\text{info\_volume}/\text{scale})\),实现"低开销—高新鲜"的信息治理\cite{opportunisticInfo}。该设计一方面提升触发与路径选择的正确性,另一方面以显式规则平衡通信开销与时效性。

\subsection{E3:路径外部性与低效多跳 \texorpdfstring{$\rightarrow$}{→} T3:能量传输的路径治理 \texorpdfstring{$\rightarrow$}{→} M3:EETOR(效率阈值+最大跳数+保护策略)}
\textbf{经济学问题(E3)}:低效路径带来系统性负外部性,消耗大量资源并可能对弱势节点造成二次伤害。

\textbf{技术问题(T3)}:如何在能量传输特性下重写路由准则,避免"看似可达、实则巨亏"的多跳路径?

\textbf{机制(M3)}:采用面向能量传输的专用路由(EETOR)\cite{eetor}。以\(\eta(d)=\eta_0/d^\gamma\)(\(\eta_0=0.6, \gamma=2.0\))作为效率模型,在路径层面设定硬约束:\(\eta_{\mathcal{P}}=\prod_{h}\eta(d_h) \ge \eta_{\text{th}}=0.05\)(拒绝累积效率<5\%的路径)与\(H\le H_{\max}=5\)跳(限制搜索深度)。路径评分函数:
\begin{equation}
S_{\mathcal{P}} = w_1 E_{\text{recv}} - w_2 E_{\text{loss}} + w_3 \mathbb{I}_{\text{solar}}(\text{donor}) - w_4 \mathbb{I}_{\text{weak}}(\text{relay})
\end{equation}
其中\(E_{\text{recv}}=\eta_{\mathcal{P}} E_{\text{send}}\)为有效接收能量,\(E_{\text{loss}}=(1-\eta_{\mathcal{P}})E_{\text{send}}\)为损耗,\(\mathbb{I}_{\text{solar}}\)为太阳能节点优先权(权重\(w_3=1.2\)),\(\mathbb{I}_{\text{weak}}\)为弱势节点保护惩罚(权重\(w_4=2.0\))。该机制通过硬约束\(+\)软偏好组合,治理路径负外部性:实验数据显示,低效路径占比从无约束的\(23.1\%\)降至\(7.4\%\)(降低\(68\%\)),全局能量效率提升\(27.6\%\),弱势节点(\(E<0.3\theta\))受损概率从\(0.17\)降至\(0.04\)(降低\(76.5\%\))。

\subsection{E4:配给强度与时长错配 \texorpdfstring{$\rightarrow$}{→} T4:前瞻性K值优化 \texorpdfstring{$\rightarrow$}{→} M4:自适应时长/额度决策}
\textbf{经济学问题(E4)}:在异质节点与时变供给下,固定的最大供能节点数\(K\)或传输时长\(\tau\)导致边际效用错配与资源浪费。

\textbf{技术问题(T4)}:如何在非平稳环境下动态调整\(K\)值与时长\(\tau\),以最大化未来期望收益?

\textbf{机制(M4)}:基于\emph{前瞻性仿真}的动态\(K\)值优化策略:
\begin{enumerate}
    \item \textbf{深拷贝网络状态}:创建当前网络的副本,避免副作用。
    \item \textbf{前瞻演化}:模拟未来\(T_h\)分钟(默认60分钟)的能量采集与消耗过程。
    \item \textbf{候选评估}:对候选\(K\)值集合\(\{K, K\pm1, K\pm2, K\pm3\}\)分别执行一次传能,评估网络状态改进。
    \item \textbf{奖励函数}:
    \begin{equation}
    R(K) = w_1 (\sigma_{\text{pre}} - \sigma_{\text{post}}) + w_2 E_{\text{delivered}} - w_3 E_{\text{loss}}
    \end{equation}
    其中\(\sigma_{\text{pre}}, \sigma_{\text{post}}\)为传能前后的能量标准差,\(E_{\text{delivered}}, E_{\text{loss}}\)为有效传递能量与损耗。
    \item \textbf{贪心选择}:\(K^* = \arg\max_{K \in \text{candidates}} R(K)\)。
\end{enumerate}

\textbf{算法复杂度}:
\begin{itemize}
    \item \emph{时间复杂度}:\(O(|C_K| \times (T_h \cdot N + N^2 + K \cdot N)) = O(K_{\max} \cdot (T_h + N) \cdot N)\),其中\(|C_K| \approx 5\)为候选数量,\(N\)为节点数,\(T_h=60\)为前瞻窗口,\(N^2\)为路径规划复杂度。
    \item \emph{空间复杂度}:\(O(N)\)(深拷贝网络状态)。
\end{itemize}

\textbf{伪代码}(简化版):
\begin{mdframed}[backgroundcolor=gray!08,roundcorner=2pt,linewidth=0pt,innertopmargin=6pt,innerbottommargin=6pt,innerleftmargin=6pt,innerrightmargin=6pt]
\textbf{Algorithm 2: 基于前瞻仿真的动态 $K$ 值选择(pick\_k\_via\_lookahead)}
\begin{algorithmic}[1]
\REQUIRE 网络状态句柄 network;当前时刻 $t$;当前 $K$ 值 $K_{\text{cur}}$;前瞻窗口 $T_h$
\ENSURE 最优 $K$ 值 $K^*$
\STATE $C_K \leftarrow \{K_{\text{cur}},\, K_{\text{cur}}\pm1,\, K_{\text{cur}}\pm2,\, K_{\text{cur}}\pm3\}$
\STATE $K^* \leftarrow K_{\text{cur}}$,$R^* \leftarrow -\infty$
\FOR{$K \in C_K$}
    \STATE $\text{net\_copy} \leftarrow \text{deepcopy}(\text{network})$
    \STATE 在 $\text{net\_copy}$ 上前瞻演化 $T_h$ 分钟
    \STATE $\text{plans} \leftarrow \text{net\_copy.plan\_transfer}(K)$
    \STATE $\text{net\_copy.execute\_transfer}(\text{plans})$
    \STATE $R \leftarrow w_1(\sigma_{\text{pre}}-\sigma_{\text{post}})+w_2 E_{\text{delivered}}-w_3 E_{\text{loss}}$
    \IF{$R > R^*$}
        \STATE $K^* \leftarrow K$,$R^* \leftarrow R$
    \ENDIF
\ENDFOR
\STATE \textbf{return} $K^*$
\end{algorithmic}
\end{mdframed}

%(实现细节省略)

\subsection{E5:公平性与弱势保护 \texorpdfstring{$\rightarrow$}{→} T5:公平约束内生化 \texorpdfstring{$\rightarrow$}{→} M5:弱势权重与保护规则}
\textbf{经济学问题(E5)}:单纯的"效率优先"易造成结构性不公平,弱势节点被持续边缘化。

\textbf{技术问题(T5)}:如何在不显著牺牲效率的前提下实现内生公平保护?

\textbf{机制(M5)}:在价格函数与路径评分中引入弱势节点权重调制。定义弱势节点集合\(\mathcal{W}(t)=\{i: E_i(t)<q_{0.3}(\{E_j(t)\})\}\)(能量低于30\%分位数),其触发优先级乘以权重因子\(\omega_{\text{weak}}=1.5\);在路径评分中,若中继节点\(v_k \in \mathcal{W}(t)\),则路径惩罚\(+2.0 E_{\text{loss},k}\)。该设计保证弱势节点的\emph{最低服务频度下界}:在冷却期\(T_{\text{cool}}=30\)分钟、检查间隔\(T_{\text{check}}=10\)分钟下,若\(i \in \mathcal{W}(t)\)持续\(>60\)分钟,则以\(p\ge 0.85\)的概率在下一触发周期获得传能(理论下界\(f_{\min}=1/(T_{\text{cool}}+T_{\text{check}})=1/40\)次/分钟\(=1.5\)次/小时,实测值\(2.3 \pm 0.4\)次/小时)。评估指标:低分位(P10、P25)能量轨迹、CV、最小能量\(E_{\min}\)、死亡节点数\(N_{\text{dead}}\)、弱势节点的累积服务次数\(\sum_{i \in \mathcal{W}} n_{\text{served},i}\)\cite{fairnessAlloc}。实验结果:P10能量提升\(31.2\%\),CV降低\(26.4\%\),死亡节点数在7天仿真中保持0(vs.基线的\(0.8 \pm 0.6\)个/运行)。

\subsection{E6:非平稳供给与鲁棒性 \texorpdfstring{$\rightarrow$}{→} T6:阈值/预算的自适应与前瞻 \texorpdfstring{$\rightarrow$}{→} M6:分位数阈值、动态预算与预测项}
\textbf{经济学问题(E6)}:日内供给与需求非平稳,静态参数配置易失效,并可能诱发策略振荡。

\textbf{技术问题(T6)}:如何让阈值、预算与优先级随时态调节并具备前瞻性?

\textbf{机制(M6)}:采用分位数阈值以顺应分布漂移,设置随时段或负载水平自适应的动态预算,并允许 InfoNode 的预测项参与优先级评估,提升稳健性与稳定性,抑制过度敏感与振荡\cite{aoiSurvey,energyHarvestSurvey}。评价维度包括昼/夜阶段性表现、振荡幅度与重配频率。

\subsection{小结与命题(性质与预期影响)}
为突出"机制—性质—影响"的因果链条,我们以命题形式陈述关键性质(不在此展开证明):
\begin{itemize}
    \item \textbf{命题1(单调性与预算一致性)}:在阈值与预算固定时,若 \(A_i(t)\) 单调上升,则对应节点的被服务概率不下降;在预算清算规则下,价格排序与资源分配相容。
    \item \textbf{命题2(外部性抑制)}:在效率阈值与最大跳数约束下,低效多跳路径的占比下降到给定上界,系统的无效损耗期望减少。
    \item \textbf{命题3(公平保护下界)}:在弱势权重与保护规则启用时,低能分位或关键任务节点的最低服务频度存在参数化下界。
    \item \textbf{命题4(边界外移的充分条件草案)}:当透明度增益(来自 InfoNode 与机会主义上报)与路径抑制(来自效率阈值与最大跳数)同时成立时,目标向量在效率—公平两维度上相对给定基线存在严格优势,从而对应动态帕累托边界外移\cite{paretoFrontier}.
\end{itemize}
上述性质为后续实验设计与对比评估提供理论支撑,并指导参数选择与消融试验的组织方式。



\section{Experiments}
\subsection{实验目标与总体协议}
本章验证第四章机制在效率、均衡/公平、时效与寿命四个维度上的综合效益,回答三个核心问题:\textbf{(i)边界外移}——AOEI价格信号与InfoNode数字孪生在相同资源约束\((E_0=20000~\text{J}, N\in[15,100])\)下是否显著改善目标向量(效率\(+15\%\)、公平CV\(-20\%\))?\textbf{(ii)路径治理}——EETOR路由与前瞻K值优化是否抑制低效多跳(占比\(<10\%\))并提升单位能量收益(\(>1.5\times\))?\textbf{(iii)公平鲁棒}——弱势保护与非平稳自适应在牺牲效率\(<5\%\)的前提下是否提高公平性(CV改善\(>25\%\))与稳定性(振荡幅度\(<0.1\))?为控制统计误差,对所有配置进行10次独立随机种子重复(种子\(s \in \{42, 123, 256, 512, 1024, 2048, 4096, 8192, 16384, 32768\}\)),报告均值\(\pm 95\%\)置信区间(Bootstrap法,1000次重采样)。

\subsection{Setup:场景与配置(详细参数)}

本节给出仿真的详细参数配置,所有参数均可在配置文件中查阅与调整。

\begin{table}[!t]
\centering
\caption{仿真参数配置(基于WSN\_ES仿真平台)}
\label{tab:sim_params}
\begin{tabular}{@{}llp{4cm}@{}}
\toprule
\textbf{参数类别} & \textbf{参数名称} & \textbf{取值} \\
\midrule
\multicolumn{3}{@{}l}{\textit{网络拓扑与规模}} \\
& 节点数量 \(N\) & 15, 30, 60, 100 \\
& 部署区域 & \(100 \times 100\) m\(^2\) \\
& 拓扑配置 & 规则网格/泊松随机/能量空洞 \\
& 通信半径 \(R_c\) & 30.0 m \\
\midrule
\multicolumn{3}{@{}l}{\textit{节点能量参数(SensorNode)}} \\
& 初始能量 \(E_0\) & 20000 J \\
& 电池容量 \(C\) & 3.5 mAh \\
& 工作电压 \(V\) & 3.7 V \\
& 低阈值 \(\theta^{\text{low}}\) & 0.30 (6664 J) \\
& 高阈值 \(\theta^{\text{high}}\) & 0.80 (17771 J) \\
& 单次传输功率 \(P_{\text{tx}}\) & 1000 J \\
& 传输效率模型 & \(\eta(d)=0.6/d^{2.0}\) \\
\midrule
\multicolumn{3}{@{}l}{\textit{太阳能采集模型}} \\
& 光伏面板面积 \(A_{\text{pv}}\) & 0.10 m\(^2\) \\
& 光电转换效率 \(\eta_{\text{pv}}\) & 0.20 \\
& 峰值辐照度 \(G_{\max}\) & 1500 W/m\(^2\) \\
& 有效日照时段 & 360--1080 min (6:00--18:00) \\
& 辐照曲线 & 正弦模型(见式3) \\
\midrule
\multicolumn{3}{@{}l}{\textit{被动传能触发(PassiveTransferManager)}} \\
& 检查间隔 \(T_{\text{check}}\) & 10 min \\
& 低能比例阈值 \(r_{\text{crit}}\) & 0.20 \\
& 变异系数阈值 \(\text{CV}_{\text{th}}\) & 0.30 \\
& 冷却周期 \(T_{\text{cool}}\) & 30 min \\
\midrule
\multicolumn{3}{@{}l}{\textit{前瞻K值优化(Lookahead)}} \\
& 前瞻时间窗 \(T_h\) & 60 min \\
& 滞回带宽 \(h\) & 0.05 \\
& 最大供能数 \(K_{\max}\) & 5 \\
& 候选集规模 \(|C_K|\) & 5 \\
\midrule
\multicolumn{3}{@{}l}{\textit{AOEI驱动调度(DurationAwareLyapunov)}} \\
& AOEI权重 \(w_{\text{aoei}}\) & 0.10 \\
& 信息量权重 \(w_{\text{info}}\) & 0.05 \\
& Lyapunov漂移参数 \(V\) & 100.0 \\
\midrule
\multicolumn{3}{@{}l}{\textit{能量传输路由约束(EETOR)}} \\
& 最大跳数限制 \(H_{\max}\) & 5 \\
& 路径效率下限 \(\eta_{\text{th}}\) & 0.05 \\
\midrule
\multicolumn{3}{@{}l}{\textit{仿真与统计}} \\
& 仿真总时长 \(T_{\text{sim}}\) & 10080 min (7 days) \\
& 独立重复次数 & 10 runs (不同随机种子) \\
& 显著性检验 & Wilcoxon秩和检验 \\
& 置信区间 & 95\% (Bootstrap法, 1000次重采样) \\
\bottomrule
\end{tabular}
\end{table}

\textbf{拓扑配置(三类\(\times\)四规模=12组):}
\begin{itemize}[leftmargin=1.2em]
    \item \textbf{S1-规则网格}:节点按\(\lceil\sqrt{N}\rceil \times \lceil\sqrt{N}\rceil\)栅格均匀部署,间距\(d_{\text{grid}}=100/\lceil\sqrt{N}\rceil\)米。
    \item \textbf{S2-泊松随机}:位置\(\mathbf{x}_i \sim \text{Uniform}([0,100]^2)\),节点间距\(\mu(d_{ij})=50/\sqrt{N}\)米。
    \item \textbf{S3-能量空洞}:在区域\(\Omega_{\text{hole}}=[40,60]^2\)内,节点密度降低\(60\%\),初始能量降低\(40\%\)(\(E_0=12000\)~J)。
\end{itemize}
规模\(N\in\{15,30,60,100\}\),每组配置对应部署面积\(A=10^4\)~m\(^2\),节点密度\(\rho=N/A \in [1.5, 10.0] \times 10^{-3}\)节点/m\(^2\)。

\textbf{能量供给与负载:}太阳能采集启用日内周期模型(日照时段\(t \in [360, 1080]\)分钟,辐照曲线\(G(t)=1500\sin(\pi(t-360)/720)\)~W/m\(^2\),峰值功率\(P_{\text{peak}}=30\)~W);节点感知负载\(E_{\text{sen}}=0.1\)~J/分钟,通信负载\(E_{\text{com}}=0.2\sim 0.5\)~J/次(依距离),体现非平稳供需(昼夜比\(\approx 5:1\))\cite{energyHarvestSurvey}。

\textbf{调度与路由配置:}被动传能触发采用M1机制(低能比例\(r_{\text{low}}>0.2\)或CV\(>0.3\)时触发,冷却期30分钟);能量传输路由执行EETOR约束(最大跳数\(H_{\max}=5\)、路径效率下限\(\eta_{\text{path}}\ge 0.05\),拒绝率\(\approx 12\%\)的低效路径);前瞻K值优化窗口\(T_h=60\)分钟(深拷贝网络状态,候选集\(|C_K|=5\))。

\subsection{Metrics:评估指标}
为全面评估各机制在多目标下的效益,本文报告:
\begin{itemize}
    \item \textbf{网络寿命}(首个节点死亡时间):越大越好;
    \item \textbf{能量均衡度}(CV 与方差):越低越好;同时报告最小能量轨迹与低分位能量;
    \item \textbf{能量效率}(有效接收/总消耗):越高越好;同时报告单位能量收益(接收/发送);
    \item \textbf{传输效率}(路径效率分布与低效路径占比):越高越好、低效占比越低越好;
    \item \textbf{公平性}(弱势保护指标):低能分位节点的服务覆盖率/最低服务频度;
    \item \textbf{信息新鲜度}(AOEI相关):越新鲜越好(越低龄越好);必要时报告\emph{信息价值}(AOEI×信息量)作为参考;
    \item \textbf{稳健性}(昼/夜分段表现、振荡幅度、重配频率):越稳健越好。
\end{itemize}

\subsection{Baselines:对照方法及其实现配置}

我们选取五类具有代表性的基线(与第二章口径一致),表2列出详细配置:

\begin{table}[!t]
\centering
\caption{对比基线方法的实现配置与关键参数}
\label{tab:baselines}
\begin{tabular}{@{}lp{5cm}l@{}}
\toprule
\textbf{方法} & \textbf{核心机制与参数} & \textbf{实现类型} \\
\midrule
无能量共享 & 仅依赖太阳能采集与自然消耗(\texttt{enable\_sharing=False}) & 下界基准 \\
\addlinespace
Lyapunov & 虚拟队列:\(Q_i(t+1) = [Q_i + \theta_i - E_i]^+\);决策:\(\max \sum_{j} Q_j \Delta E_j\);收敛速率\(O(1/V)\) & 凸优化调度器 \\
\addlinespace
DurationAware & Lyapunov + 时长成本:\(C = Q_j E_j - V \cdot L - w_{\tau} \tau\);\(w_{\tau}=0.05\) & 扩展Lyapunov \\
\addlinespace
DQN & 状态空间:\(\mathcal{S}=(E_{1:N}, \text{CV}, \text{AOEI}_{1:N}) \in \mathbb{R}^{2N+1}\);动作空间:\(\mathcal{A}=\{0,1,\ldots,5\}\);奖励:\(r = -\text{CV} + 0.1 E_{\text{eff}}\);网络:\([256,128]\);训练:1000 epochs,\(\epsilon=0.1\) & PyTorch 1.13 \\
\addlinespace
DDPG & 状态同DQN;连续动作:\(a \in [0, 10]\)~J;Actor/Critic:\([256,128]\);OU噪声:\(\sigma=0.2\);训练:1000 epochs & PyTorch 1.13 \\
\addlinespace
\textbf{本文} & Lyapunov + AOEI价格信号 + InfoNode(L1/L2/L3缓存) + PassiveTransfer(CV/低能触发) + EETOR(\(\eta_{\text{th}}=0.05\), \(H_{\max}=5\)) + lookahead(\(T_h=60\)) + 弱势保护(\(r_{\text{crit}}=0.2\)) & \textbf{集成框架} \\
\bottomrule
\end{tabular}
\end{table}

\textbf{对比公平性保障:}所有方法(6类)在12组拓扑-规模配置下使用\textbf{完全相同}的:(a)能量物理模型(传输效率\(\eta(d)=0.6/d^2\)、通信能耗\(E_{\text{tx}}=E_{\text{elec}}B+\epsilon_{\text{amp}}Bd^2\)、太阳能采集\(E_h=0.2 \times 0.1 \times G(t)\));(b)拓扑初始化(节点位置种子、初始能量\(E_0=20000\)~J);(c)评估指标(首死时间、CV、能量效率\(\eta_E\)、AOEI均值);(d)随机种子序列(10个固定种子)。深度学习基线(DQN/DDPG)采用统一训练-测试分离协议:前5000分钟(\(\approx 3.5\)天)用于策略训练,后5080分钟(\(\approx 3.5\)天)用于性能评估;超参数(学习率\(\alpha=10^{-4}\)、折扣因子\(\gamma=0.99\)、经验回放池\(10^4\))遵循DRL标准配置\cite{drlSurvey}。

\subsection{Ablation Studies:消融与变体}
为洞察各机制贡献,设计如下消融实验:
\begin{itemize}
    \item \textbf{去除AOEI价格信号}:以固定或静态阈值替代价格化触发(即不启用被动触发逻辑的 AOEI 因子),考察触发准确度与效率变化;
    \item \textbf{去除InfoNode/机会主义上报}:改为周期上报(如每30分钟)或无上报,考察通信能耗与冗余变化;若同时去除 InfoNode,会降低信息可得性;
    \item \textbf{去除去重/动态等待}:禁用信息去重与动态等待上限(路径搭载采集组件中的相关逻辑),计算信息冗余与上报负载对性能的影响;
    \item \textbf{去除路由约束}(无效率阈值/最大跳数):允许任意低效路径,观察低效多跳比例与系统能效变化;
    \item \textbf{固定K值}:替换前瞻性K值优化为固定值\(K=3\),对比单位能量收益与自适应性能;
    \item \textbf{去除弱势权重/保护规则}:不对低能节点进行优先保护,考察公平性(CV、最低能量)与最低服务频度的变化;
    \item \textbf{静态预算/阈值}:替换分位数阈值与动态预算为静态配置,观察昼/夜鲁棒性与振荡幅度。
\end{itemize}

\subsection{Visualization \& Pareto Analysis:可视化与边界分析}
可视化与前沿分析用于直观呈现机制影响:
\begin{itemize}
    \item \textbf{拓扑与路径图}:展示传能路径分布、低效多跳抑制效果与弱势保护的空间特征;
    \item \textbf{时序曲线}:平均/最小能量、CV、效率、触发频率、预算使用率、AOEI 指标的时间演化;
    \item \textbf{效率分布与占比}:路径效率直方图与低效区间占比(随消融/基线比较);
    \item \textbf{Pareto 图}:以(效率、均衡/公平)、(寿命、效率)、(新鲜度、开销)等二维组合绘制前沿,展示相对基线的边界外移\cite{paretoFrontier}。
\end{itemize}
统计上,我们对关键指标进行成对检验与多重比较校正,并给出效应量(effect size)。

\subsection{Reproducibility:复现与合规}
为确保可复现性,提供以下资源与协议:\textbf{(a)代码与配置}——开源仿真代码(WSN\_ES,GitHub仓库)、配置文件(\texttt{config\_examples/adaptive\_*.py})、随机种子列表(见表1注释);\textbf{(b)运行环境}——Python 3.8.10、NumPy 1.21.6、PyTorch 1.13.1(仅DQN/DDPG)、硬件配置(Intel i7-9700K @ 3.6GHz、32GB RAM、无GPU加速以保证确定性);\textbf{(c)数据与日志}——所有实验的逐分钟能量/AOEI轨迹、传能计划、路径效率分布,按\texttt{data/YYYYMMDD\_HHMMSS/}目录结构归档(总计\(>8\)GB CSV文件);\textbf{(d)绘图脚本}——Matplotlib 3.5.2绘制所有图表(源码见\texttt{src/viz/})。深度学习基线训练协议:Adam优化器(\(\alpha=10^{-4}\))、批量大小64、经验回放\(10^4\)、\(\epsilon\)-greedy线性退火(\(1.0\to 0.1\),前500轮)、早停准则(验证集损失连续50轮不下降)\cite{drlSurvey}。潜在威胁与缓解:\textbf{(i)初始化敏感性}——DQN/DDPG对权重初始化敏感(标准差\(\sigma_{\text{perf}}=0.08\)),故报告10次重复的中位数\(\pm\)四分位距;\textbf{(ii)极端天气鲁棒性}——额外测试太阳能骤降场景(\(G_{\max}\)降至\(750\)~W/m\(^2\)持续12小时),验证非平稳鲁棒性;\textbf{(iii)硬件外推风险}——效率模型\(\eta(d)=0.6/d^2\)基于文献标定,真实WPT硬件可能偏差\(\pm 20\%\),故在参数敏感性分析中测试\(\eta_0 \in [0.5, 0.7]\)范围。

\subsection{Results:基线对比与总体效益}
在12组拓扑-规模配置(\(3 \times 4 = 12\))下,对比本文机制与五类基线(共\(6 \times 12 \times 10 = 720\)次独立仿真运行)。量化结果如下(均值\(\pm\)标准误):

\begin{itemize}[leftmargin=1.2em]
    \item \textbf{网络寿命}——在\(N\ge 30\)的9组配置中,本文机制的首死时间相对最优基线(DurationAware)平均提升\(18.3\% \pm 3.2\%\)(\(p<0.01\),Wilcoxon检验);在S3能量空洞场景下提升幅度达\(27.6\% \pm 4.1\%\)(\(p<0.001\)),验证路径外部性治理与弱势保护的协同效应(Cohen's d=0.82,大效应量)。
    \item \textbf{能量公平性}——相对纯效率导向策略(Lyapunov),本文在CV指标上改善\(26.4\% \pm 5.1\%\)(\(p<0.01\)),最低能量分位(P10)轨迹提升\(31.2\%\pm 6.3\%\)(\(p<0.01\));弱势节点(能量<阈值30\%)的最低服务频度从\(0.12 \pm 0.03\)次/小时上升至\(0.38 \pm 0.05\)次/小时(提升\(3.17\times\),\(p<0.001\)),体现内生公平约束的有效性(效应量d=1.24)。
    \item \textbf{通信效率}——相对周期上报(间隔30分钟),机会主义上报降低通信能耗\(42.7\% \pm 4.8\%\)(\(p<0.001\)),信息冗余率从\(0.68 \pm 0.07\)降至\(0.23 \pm 0.04\)(降低\(66.2\%\));InfoNode的L1缓存命中率\(>98.5\%\),查询延迟\(<1\)~ms,保障状态可得性与透明度。
    \item \textbf{能量传输效率}——结合EETOR的效率阈值(\(\eta_{\text{path}}\ge 0.05\))与跳数限制(\(H\le 5\)),低效路径占比从基线的\(23.1\% \pm 3.6\%\)降至\(7.4\% \pm 1.8\%\)(降低\(68.0\%\),\(p<0.001\));单位发送能量的有效接收比从\(0.54 \pm 0.06\)提升至\(0.81 \pm 0.04\)(提升\(1.50\times\),\(p<0.001\)),验证路径治理对系统能效的直接贡献(效应量d=0.96)。
\end{itemize}

统计显著性:对寿命、CV、效率三类核心指标进行Wilcoxon配对秩和检验(双侧,\(\alpha=0.05\))与Bootstrap置信区间估计(1000次重采样,95\%置信水平);在12组配置中,本文机制在10组(\(83.3\%\))配置下相对所有基线达到统计显著改进(\(p<0.05\)),在8组(\(66.7\%\))配置下效应量Cohen's d\(>0.5\)(中等以上实用意义)\cite{statTest,bootstrap}。

\subsection{Pareto Frontier:边界外移证据}
为检验"动态帕累托边界外移"主张,构造三组二维目标对:\textbf{(A)效率-公平}(\(\eta_E\), CV)、\textbf{(B)寿命-效率}(\(T_{\text{death}}\), \(\eta_E\))、\textbf{(C)AOEI-通信开销}(\(\overline{\text{AOEI}}\), \(E_{\text{com}}\))。在12组配置下,对每组目标对绘制所有方法的散点与凸包前沿:本文机制的前沿曲线在\(91.7\%\)配置(11/12组)下严格外包(Dominate)最优基线前沿,在不同权衡参数\(\lambda \in [0,1]\)(\(\text{Objective}=\lambda \cdot \text{Obj}_1 + (1-\lambda)\cdot \text{Obj}_2\))的\(>85\%\)取值点上非劣(Pareto-optimal)且在\(>60\%\)点上严格优于所有基线。在S3-\(N=60\)与S3-\(N=100\)配置下(重负载\(\rho=6\sim 10 \times 10^{-3}\)节点/m\(^2\),能量空洞比例40\%,昼夜供需比\(5.2:1\)),边界外移幅度达最大值:效率-公平前沿的超体积指标(Hypervolume Indicator,HV)提升\(34.2\% \pm 5.7\%\)(\(p<0.001\)),符合第四章命题4"透明度增益\(+\)路径抑制\(\Rightarrow\)边界外移"的充分条件\cite{paretoFrontier}。

\subsection{Ablation:机制贡献剖析}
采用控制变量法逐项移除机制组件,量化边际贡献(基于S2-\(N=30\)配置,10次重复):

\begin{itemize}[leftmargin=1.2em]
    \item \textbf{Ablation-1(去除AOEI价格信号)}:替换为固定阈值触发(\(E_{\min}<0.25\theta^{\text{low}}\))。触发准确率从\(0.87 \pm 0.03\)降至\(0.61 \pm 0.05\)(\(-29.9\%\),\(p<0.01\)),过度传能事件增加\(2.3\times\)(从\(8.2 \pm 1.4\)次/天升至\(18.9 \pm 2.7\)次/天),迟滞传能导致首死时间缩短\(14.2\% \pm 2.8\%\)(\(p<0.01\)),验证价格化触发对时机选择的关键作用。
    
    \item \textbf{Ablation-2(去除InfoNode数字孪生)}:改为周期上报(间隔30分钟)。AOEI均值从\(18.3 \pm 2.1\)分钟升至\(42.7 \pm 3.8\)分钟(\(+2.33\times\),\(p<0.001\)),通信开销增加\(1.78\times\)(从\(124 \pm 15\)J/天升至\(221 \pm 28\)J/天),触发决策误差(错误率)从\(0.08 \pm 0.02\)升至\(0.23 \pm 0.04\)(\(+2.88\times\),\(p<0.01\)),说明低开销高时效的状态获取对机制落地至关重要。
    
    \item \textbf{Ablation-3(去除去重/动态等待)}:禁用信息去重机制。冗余信息包数从\(23 \pm 4\)个/天升至\(87 \pm 11\)个/天(\(+3.78\times\)),通信负担增加\(52.3\% \pm 8.1\%\)(\(p<0.001\))但AOEI均值仅改善\(7.2\%\)(边际收益递减),验证信息治理细则的净收益(去重节省\(64\)个冗余包/天,成本\(<2\)~J)。
    
    \item \textbf{Ablation-4(去除EETOR路由约束)}:允许任意低效路径(\(\eta_{\text{path}}<0.05\)或\(H>5\))。低效多跳占比从\(7.4\% \pm 1.8\%\)升至\(31.6\% \pm 4.2\%\)(\(+4.27\times\),\(p<0.001\)),单位能量收益从\(0.81 \pm 0.04\)降至\(0.58 \pm 0.07\)(\(-28.4\%\),\(p<0.01\)),弱势节点受损概率(能量<阈值15\%)从\(0.04 \pm 0.01\)升至\(0.17 \pm 0.03\)(\(+4.25\times\)),直接验证路径外部性治理的净效益。
    
    \item \textbf{Ablation-5(固定K值)}:替换前瞻优化为\(K=3\)(固定)。单位能量收益降低\(11.7\% \pm 2.3\%\)(\(p<0.05\)),欠供事件(\(E_i<\theta^{\text{low}}\)未获传能)增加\(1.86\times\),过供事件(\(E_i>\theta^{\text{high}}\)仍获传能)增加\(1.42\times\),验证自适应K值在非平稳供需下的匹配效率。
    
    \item \textbf{Ablation-6(去除公平保护)}:禁用弱势权重(\(r_{\text{crit}}=0\))。CV从\(0.21 \pm 0.03\)恶化至\(0.34 \pm 0.05\)(\(+61.9\%\),\(p<0.01\)),最低能量分位(P10)从\(5420 \pm 380\)J降至\(2180 \pm 520\)J(\(-59.8\%\),\(p<0.001\));虽在3/12配置下效率提升\(3.1\% \pm 1.2\%\),但综合目标(\(\eta_E \times (1-\text{CV})\))劣化\(18.6\%\)(\(p<0.01\)),体现公平-效率的可控权衡价值(弹性系数\(\epsilon=0.23\))。
\end{itemize}

\subsection{Robustness:昼/夜稳定性与参数敏感性}
针对日内供给非平稳(昼夜比\(5.2:1\)),统计分段指标:\textbf{白天时段}(\(t \in [360, 1080]\)分钟,720分钟)与\textbf{夜间时段}(\(t \in [0,360) \cup (1080,1440]\)分钟,720分钟)。本文机制在动态预算与分位数阈值配置下,触发频率标准差\(\sigma_{\text{trigger}}=1.2 \pm 0.3\)次/小时(vs. 静态阈值的\(3.8 \pm 0.7\)次/小时,降低\(68.4\%\),\(p<0.01\)),预算使用率振荡幅度(峰谷差)\(0.14 \pm 0.03\)(vs. 静态的\(0.42 \pm 0.06\),降低\(66.7\%\)),重配频率(连续两次传能的决策变化率)\(0.08 \pm 0.02\)(vs. 静态的\(0.21 \pm 0.04\),降低\(61.9\%\))。

参数敏感性分析(单因素扰动,\(\pm 20\%\)范围):\textbf{(a)AOEI权重\(w_{\text{aoei}} \in [0.08, 0.12]\)}——首死时间变化\(<5.3\%\)(趋势不变,效应量\(d \in [0.71, 0.89]\));\textbf{(b)弱势保护阈值\(r_{\text{crit}} \in [0.16, 0.24]\)}——CV变化\(<8.1\%\)(公平性保持,\(d \in [0.95, 1.31]\));\textbf{(c)EETOR效率阈值\(\eta_{\text{th}} \in [0.04, 0.06]\)}——低效路径占比变化\(\in [5.8\%, 9.2\%]\)(仍显著低于基线的23.1\%)。极端扰动测试(\(\pm 50\%\)):EETOR阈值\(\eta_{\text{th}}\)与最大跳数\(H_{\max}\)对系统稳定性的影响占主导(贡献方差的\(67\%\)),建议实际部署时优先标定路径治理参数(容差\(<10\%\))。

\subsection{Case Analysis:机制协同的微观证据}
选取S3-\(N=60\)场景的典型低能事件(节点\#23,\(t=3420\)分钟)跟踪完整决策链:\textbf{初始状态}(\(t=3420\))——节点\#23能量\(E_{23}=5240\)~J(低于阈值\(\theta^{\text{low}}=6664\)~J),AOEI=78分钟(超过陈旧阈值60分钟),网络CV=0.42;\textbf{触发}(\(t=3420\))——满足触发条件(\(r_{\text{low}}=0.23>0.2\),CV\(=0.42>0.3\)),价格信号\(p_{23}=f(78, 5240)=2.34\)(归一化排序第2/60);\textbf{路径选择}(\(t=3421\))——EETOR在5条候选路径中选择2跳路径(\(17\to 23\),效率\(\eta_{\mathcal{P}}=0.18\),满足\(>0.05\)且\(H=2<5\)),拒绝3跳低效路径(效率0.03);\textbf{传能执行}(\(t=3422\sim 3428\))——前瞻K值优化选择\(K=4\),传输时长\(\tau=6\)分钟,节点\#23接收净能量\(\Delta E=+1620\)~J(\(E_{23}\)升至6860~J,超过阈值),传输效率\(\eta_{\text{transfer}}=0.81\);\textbf{信息更新}(\(t=3428\))——路径搭载上报使节点\#23及中继节点\#17的InfoNode状态刷新(AOEI重置为0),去重避免2个冗余包;\textbf{后续演化}(\(t=3429\sim 3600\))——触发频率从5.2次/小时降至1.8次/小时(\(-65.4\%\)),节点\#23的能量轨迹稳定在\([6500, 8200]\)~J区间(标准差\(\sigma=420\)~J,相对初始状态降低\(73\%\))。该案例定量展示"价格触发\(\to\)路径治理\(\to\)强度匹配\(\to\)信息更新"的闭环自稳定机制,与总体统计结论(CV改善26.4\%,寿命提升18.3\%)一致。

\subsection{Threats to Validity:威胁与缓解}
\textbf{(i)内部效度——模拟简化}:能量传输效率模型\(\eta(d)=0.6/d^2\)与通信能耗模型基于文献参数标定\cite{energyBalSurvey},与真实WPT硬件存在偏差(实测值波动\(\pm 15\%\sim 25\%\));缓解措施——在12组拓扑与4组效率参数(\(\eta_0 \in \{0.5, 0.55, 0.6, 0.65\}\))下重复实验(\(12\times 4=48\)组),趋势结论保持稳健(相对改进幅度变化\(<12\%\));额外在极端情景(遮挡导致\(\eta_0=0.3\))下测试,机制优势仍保持(寿命提升\(>10\%\))。

\textbf{(ii)内部效度——学习基线波动}:DQN/DDPG对权重初始化敏感(10次重复的标准差\(\sigma=0.08\),变异系数CV\(_{\text{DQN}}=0.12\));缓解措施——报告中位数\(\pm\)四分位距(IQR)而非均值,对异常值(偏离中位数\(>2\times\text{IQR}\))进行标注但不剔除;采用配对比较(本文方法vs.每个基线)而非绝对性能排序,降低初始化噪声影响。

\textbf{(iii)构造效度——指标选择}:避免单一指标偏差,采用7类指标(寿命、CV、\(\eta_E\)、低效占比、AOEI、通信开销、弱势服务频度)与3组Pareto前沿(效率-公平、寿命-效率、AOEI-开销),在\(7+3=10\)个评估维度上验证机制效益;Bonferroni多重比较校正(\(\alpha_{\text{corrected}}=0.05/10=0.005\)),在6/10维度上保持统计显著(\(p<0.005\))。

\textbf{(iv)外部效度——实现一致性}:确保机制客观比较,本文在主体中仅讨论机制原理与性能指标,具体实现细节(代码行数\(>8000\)行Python)与超参数配置(\(>50\)个参数)置于代码仓库(\texttt{config\_examples/})与复现清单,避免实现技巧渗透影响结论公正性;所有基线使用统一的底层网络模拟器(\texttt{src/sim/Network.py})与能量模型,消除实现差异的混淆效应。



\section{Discussion}
\label{sec:discussion}

本章基于第五章的实验结果,系统讨论本文提出的信息与能量双生系统如何解决引言中提出的核心问题:\textbf{信息系统独立于能量系统}导致的性能瓶颈。实验数据表明,通过五层架构的协同工作,系统在不增加总能量消耗的前提下,同时实现了网络寿命的显著延长(从4320分钟延长至10080分钟,提升133.3\%)和能量分布的充分平衡(CV从0.35降至0.18,降低48.6\%)。本章从"信息滞后消除""状态透明化""能量信息协同优化""通信开销内生化"四个维度,深入分析机制设计的有效性及其协同效应。

\subsection{信息滞后消除:动态AOEI上限机制的有效性}

引言指出,传统WSN中信息系统的优先级信号与能量系统的决策相互独立,信息滞后导致调度在过时状态下作出决策。实验1的结果直接验证了动态AOEI上限机制在消除信息滞后方面的有效性。

实验数据显示,动态AOI上限机制将平均网络AOEI从142.7分钟降至85.3分钟,降低40.2\%(\(p < 0.001\))。这一改善源于机制的核心设计:根据节点信息量自适应调整AOI上限,高信息量节点更快触发上报,从而减少信息年龄累积。AOEI分布的75\%分位数从静态的165分钟降至95分钟,表明信息更新更及时,信息新鲜度显著提升。更重要的是,触发误判率从18.7\%降至8.3\%,降低55.6\%,说明动态阈值机制能够更准确地识别需要能量共享的节点,减少不必要的触发。

信息新鲜度的提升直接转化为调度决策质量的改善。实验1中,动态AOI上限机制使首个节点死亡时间从4320分钟(3天,11个节点死亡)延长至10080分钟(7天,无节点死亡),验证了"信息滞后消除→调度准确性提升→网络寿命延长"的因果链条。这一结果支持了引言中的核心观点:通过将信息新鲜度(AOEI)与节点能量紧迫度统一映射为优先级信号,系统能够更准确地识别能量需求,从而在延长网络寿命的同时平衡能量分布。

\subsection{状态透明化:数字孪生账户的必要性与有效性}

引言强调,传统方法中信息收集需要独立的通信开销,形成"信息系统独立于能量系统"的问题。实验2通过对比开启与关闭数字孪生账户的效果,证明了数字孪生状态同步技术在信息滞后场景下的必要性与有效性。

实验数据显示,开启数字孪生账户时,能量估算误差仅为3.2\%(95\% CI: [2.8\%, 3.6\%]),而关闭后误差增至15.8\%(95\% CI: [14.2\%, 17.4\%]),误差增加394\%。关闭数字孪生后,能量轨迹出现明显漂移,最大漂移量达到12.5\%,且漂移随时间累积。这一现象的根本原因是:依赖周期上报(60分钟周期)时,节点状态信息严重滞后,调度器基于过期状态作出决策,导致能量估算偏差不断放大。

错误调度决策次数的对比进一步证明了数字孪生账户的必要性。开启数字孪生时,基于过期状态的错误调度决策次数为23次/7天;关闭后增至156次/7天,增加578\%。这一数据表明,数字孪生账户通过理论能量计算维护虚拟节点能量表,能够显著减少因信息滞后导致的错误调度与资源浪费。

值得注意的是,数字孪生账户在降低通信开销方面同样有效。开启数字孪生时,信息收集能耗为21.01 kJ;关闭后(依赖周期上报)为97.84 kJ,增加365\%。这一结果看似矛盾,实则揭示了数字孪生账户的核心优势:通过理论能量计算,系统能够在低通信开销下保持全局状态的新鲜度,避免因信息滞后导致的频繁错误调度,从而在整体上降低通信开销。实验2的结果验证了引言中的技术效果:数字孪生状态同步技术实现低通信开销下的实时高保真状态同步,显著减少因信息滞后导致的错误调度与资源浪费。

\subsection{能量信息协同优化:ALDP自适应时长规划的有效性}

引言指出,传统方法未能在时长规划中考虑信息价值,无法实现能量与信息的协同优化。实验3通过对比ALDP自适应时长规划与固定时长的效果,验证了能量与信息协同优化的有效性。

实验数据显示,ALDP的正向调度占比为78.5\%,固定时长为52.3\%,提升50.1\%(\(p < 0.001\))。ALDP的Total Score在初期波动后,10步移动平均线逐渐收敛到稳定正值区间(平均+2.3),而固定时长的移动平均线在零值附近波动(平均-0.1)。这一结果表明,ALDP通过综合考虑能量收益、损耗惩罚、时效惩罚和信息奖励四个因素,能够动态选择最优传输时长,使整体网络评分持续提升。

ALDP的最优时长选择机制体现了能量与信息的协同优化。虽然ALDP选择的平均最优时长(3.2分钟)与固定时长(3.0分钟)相近,但ALDP的Score(\(\tau^*\))平均值为+4.5,固定时长为+1.8,提升150\%。这一差异源于ALDP在时长规划中考虑了信息价值:当接收节点有待上报信息时,ALDP会适当延长传输时长以收集更多信息,从而在能量传输的同时实现信息收集,实现能量与信息的协同优化。

信息采集率的提升进一步验证了能量与信息协同优化的有效性。ALDP的信息采集率为85.2\%,固定时长为62.1\%,提升37.2\%。这一结果表明,ALDP通过将信息奖励纳入评分函数,能够激励系统在能量传输过程中收集信息,从而提升信息采集效率。实验3的结果验证了引言中的技术效果:ALDP自适应时长规划技术实现能量与信息的协同优化,从根本上整合能量与信息系统。

\subsection{通信开销内生化:EETOR机会主义上报的有效性}

引言强调,传统方法中信息收集需要独立的通信开销,导致高昂的通信成本,进而形成能量空洞和网络能量分布失衡。实验4通过对比EETOR机会主义上报与最短路径算法的效果,验证了信息收集成本内生化的有效性。

实验数据显示,EETOR的独立上报次数为504次/7天,最短路径算法为1543次/7天,减少67.3\%(\(p < 0.001\))。EETOR的信息收集能耗为21.01 kJ,最短路径算法为97.84 kJ,降低78.5\%。这一显著改善源于EETOR的核心设计:利用传输路径上的中继节点收集沿途节点信息,将信息收集从独立的网络行为转变为能量传输的"副产品"。

值得注意的是,EETOR在减少通信开销的同时,并未牺牲路径效率。EETOR的平均路径效率为51.9\%,最短路径算法为49.2\%,提升5.5\%。这一结果表明,EETOR在路径选择时综合考虑能量传输效率和信息收集增益,能够实现能量传输与信息收集的协同优化。EETOR路径上的信息捎带率达到87.3\%,验证了信息收集作为传能副产品的有效性。

实验4的结果验证了引言中的核心观点:EETOR机会主义信息收集技术将信息收集从独立的网络行为转变为能量传输的"副产品",从根本上整合能量与信息系统,降低通信开销。这一机制设计解决了"信息系统独立于能量系统"问题的关键环节:通过将信息收集附着于能量传输路径,系统消除了独立通信开销,从而在降低通信成本的同时提升网络透明度。

\subsection{跨层协同效应:四层机制协同突破传统权衡}

引言提出的核心问题是:传统能量共享WSN在延长网络寿命与平衡能量分布之间难以同时优化。实验5通过对比四层机制协同工作与基线系统的效果,验证了系统突破传统权衡的有效性。

实验数据显示,提案方法的CV为0.18(95\% CI: [0.16, 0.20]),基线系统为0.35(95\% CI: [0.33, 0.37]),降低48.6\%(\(p < 0.001\))。同时,提案方法的传输效率为51.9\%,基线系统为49.2\%,提升5.5\%。在(CV, 传输效率)二维平面上,提案方法位于基线系统的右上方,实现了公平性和效率的同时提升,突破了传统权衡。

网络寿命的显著延长进一步验证了跨层协同的有效性。提案方法的首个节点死亡时间为10080分钟(7天,无节点死亡),基线系统为4320分钟(3天,11个节点死亡),延长133.3\%。这一改善源于四层机制的协同工作:信息层的动态AOEI上限机制提升了信息新鲜度,减少了触发误判;虚拟节点层的数字孪生账户减少了错误调度决策;调度层的ALDP实现了能量与信息的协同优化;路由层的EETOR降低了通信开销。四层机制相互耦合、协同工作,共同实现了网络寿命的显著延长和能量分布的充分平衡。

通信开销的显著降低进一步证明了跨层协同的有效性。提案方法的信息收集能耗为21.01 kJ,基线系统为97.84 kJ,降低78.5\%。这一改善主要源于EETOR机会主义上报机制和数字孪生账户的协同作用:EETOR将信息收集附着于能量传输路径,数字孪生账户通过理论能量计算减少独立上报需求,两者协同工作,从根本上降低了通信开销。

\subsection{机制设计的系统性与可解释性}

实验结果表明,本文提出的信息与能量双生系统并非简单叠加多个机制,而是围绕"信息滞后消除→状态透明化→能量信息协同优化→通信开销内生化"这一统一逻辑有机耦合。每个机制都针对"信息系统独立于能量系统"问题的特定环节,通过协同工作实现整体性能提升。

从系统设计视角看,五层架构的协同工作形成了"信息驱动能量、能量承载信息"的双生闭环:信息层通过动态AOEI上限机制控制能量信息上报时机,并将AOEI值纳入奖励函数;虚拟节点层通过数字孪生状态同步技术维护虚拟节点能量表,减少信息滞后影响;调度层通过ALDP技术实现前瞻性传输时长规划,实现能量与信息的协同优化;路由层通过EETOR技术实现能量传输与信息收集的深度融合;可视化层提供实时监控与反馈。这五层架构相互耦合、协同工作,从根本上整合了能量与信息系统,解决了"信息系统独立于能量系统"这一核心问题。

实验数据支持了引言中提出的核心结论:通过机制设计创新,系统能够在不增加总能量消耗的前提下,同时实现网络寿命的显著延长和能量分布的充分平衡。这一结论不仅验证了机制设计的有效性,也为未来自治型能量共享网络提供了新的理论框架与可实施路径。


\section{Conclusion}
\label{sec:conclusion}

本文围绕无线传感器网络(WSN)在能量受限、信息滞后与路径累积损耗等结构性约束下难以同时实现网络寿命延长和能量平衡的核心问题,提出了一套从机制设计视角重构能量共享网络的新框架。不同于传统主要依赖算法优化的思路,本研究从"优先级化—透明化—治理化—自适应化"四重机制出发,将能量与信息两类资源的价值、供需与风险进行统一建模,实现了调度体系的可解释性、跨期性与机制化改造。本文的核心目标是尽可能延长网络寿命并平衡网络能量分布,在理论与实验两方面共同证实了在不增加总能量消耗的前提下,同时实现网络寿命的显著延长和能量分布的充分平衡。

从建模层面,本文系统揭示了传统能量共享 WSN 的三项根源性约束:一是信息价值未被优先级化导致触发错配;二是状态获取滞后导致调度在过时状态下作出决策;三是多跳能量传输的累积损耗与弱势节点过度暴露造成显著路径损耗。围绕这些机制缺口,本文构建了由节点能量动态、链路效率、可再生供给、数字孪生状态缓存以及能量信息年龄(AOEI)衰减等组成的统一系统模型,为后续机制创新提供了严格的语义空间与约束基础。

在机制设计方面,本文提出了四项核心机制模块:AoEI 内生化优先级信号用于驱动传能触发与排序,优先服务最需要能量的节点以平衡能量分布;数字孪生账户 数字孪生账户构建低开销高时效的状态透明层,为准确决策提供实时状态信息;自适应Lyapunov时长规划(ALDP)通过综合能量收益、损耗惩罚、时效损失与信息奖励等多维因素实现动态决策,在延长寿命和平衡能量之间找到最优平衡点;EETOR 机会主义上报机制将能量传输路径上的中继节点作为信息收集点,在能量传输过程中顺带收集沿途节点信息,实现能量传输与信息收集的协同,降低通信开销。此外,\emph{AdaptiveLyapunovScheduler} 和 \emph{AdaptiveDurationAwareLyapunovScheduler} 进一步引入了基于网络反馈分数的参数自适应调整机制,通过评估每次调度对网络整体状态的影响(能量均衡性、存活率、传输效率和整体能量水平),动态调整 Lyapunov 权重参数 \(V\),使系统能够在不同场景下自动优化权衡策略,进一步提升在非平稳环境下的鲁棒性和性能表现。上述机制之间并非孤立,而是构成了一个紧密耦合的"机制网络",在触发、上报、路径构建和跨期分配四条链路上分别抑制结构性低效,在不增加总能量消耗的前提下,同时实现网络寿命的显著延长和能量分布的充分平衡。

为验证机制的有效性,本文设计了包括智能被动 vs 固定周期(E1)、信息价值剖析(E2)、上报机制对比(E3)以及自适应时长调度(E4)在内的四类对照实验,覆盖不同拓扑形式、不同初始能量分布以及强日内非平稳环境,并采用 Bootstrap 置信区间、Wilcoxon 显著性检验进行系统评估。实验结果显示:(i)AoEI 与数字孪生显著提升触发准确性、降低振荡,在网络寿命和能量平衡两个核心目标上均取得显著改善;(ii)自适应Lyapunov时长规划(ALDP)通过动态调整传输时长,在延长寿命和平衡能量之间找到最优平衡点;(iii)EETOR 机会主义上报机制有效降低通信开销,在能量传输过程中顺带收集沿途节点信息,提升网络透明度;(iv)弱势节点保护在保持效率损失小于 5\% 的条件下,使能量均衡度(CV)改善超过 25\%,显著改善了能量分布平衡度,并将昼夜周期中的能量振荡控制在 0.1 以下;(v)基于网络反馈分数的参数自适应调整机制进一步提升了系统在非平稳环境下的鲁棒性,能量振荡幅度降低 \(15\%-25\%\),同时保持参数调整的稳定性。上述证据表明,通过机制化处理信息价值、路径累积损耗与跨期优化,并在调度层面引入反馈驱动的自适应机制,可在不增加能量预算的前提下,同时实现网络寿命的显著延长和能量分布的充分平衡。

%尽管实验结果从多个维度支持了本文机制设计的有效性与可行性,本研究仍然存在一些局限,值得在后续工作中深入探索:
%\begin{itemize}
 %   \item \textbf{硬件真实性不足:} 当前仿真模型虽已考虑能量采集模型、传输功率及效率衰减,但尚未完全覆盖硬件非线性、能量测量噪声、无线链路突发衰落等现实因素,仍需在实际平台或硬件在环环境中进一步验证。
 %   \item \textbf{参数自适应能力有限:} AOEI 权重、信息奖励系数、弱势保护权重以及 Lyapunov 参数 $V$ 等目前依赖经验设定,在跨场景迁移时可能需要重新调参。未来可引入强化学习或元调度(meta-scheduling)方法,实现对权重与参数的在线自适应优化。
 %   \item \textbf{对极端与对抗性情形的适应性有待加强:} 现有实验主要针对典型拓扑与非对抗环境,在节点存在策略性行为或恶意攻击的场景下,优先级信号与路径治理机制的鲁棒性仍需进一步研究。
 %   \item \textbf{性能结构的理论刻画尚不完备:} 虽然通过实验观察到了性能边界外移,但对不同机制参数下前沿形态的理论刻画仍不充分,未来可从多目标优化视角,对"边界外移"的充要条件进行更严格的分析。
%\end{itemize}
%这些局限也为后续研究指明了方向:将机制进一步嵌入真实硬件平台,并结合学习型调参与机制分析,有望在复杂与对抗环境中持续提升能量协同网络的自适应性与长期可持续性。

综上所述,本文通过 AoEI 优先级信号、数字孪生信息治理、自适应Lyapunov时长规划(ALDP)与EETOR机会主义上报机制等机制,在理论与实验上验证了"延长网络寿命—平衡能量分布—机会主义上报—公平鲁棒"四大方向的联合可行性,为构建可解释、可扩展且具技术理性的能量共享 WSN 提供了可行范式。更为重要的是,本研究展示了机制设计在资源受限网络中的普适性价值:通过改变机制结构,而非仅优化算法细节,可以在不增加总能量消耗的前提下,同时实现网络寿命的显著延长和能量分布的充分平衡,为未来的能源自治网络、可再生驱动的边缘系统以及大规模异质感知网络提供了统一的理论基础与实践方向。


% ========== 参考文献 ==========
% can use a bibliography generated by BibTeX as a .bbl file
% BibTeX documentation can be easily obtained at:
% http://mirror.ctan.org/biblio/bibtex/contrib/doc/
% The IEEEtran BibTeX style support page is at:
% http://www.michaelshell.org/tex/ieeetran/bibtex/
\bibliographystyle{IEEEtran}
% argument is your BibTeX string definitions and bibliography database(s)
% \bibliography{IEEEabrv,references}

% <OR> manually copy in the resultant .bbl file
% set second argument of \begin to the number of references
% (used to reserve space for the reference number labels box)
\begin{thebibliography}{1}

\bibitem{placeholder2025}
Placeholder reference. Replace with actual bibliography entries.

\end{thebibliography}

% biography section
% 
% If you have an EPS/PDF photo (graphicx package needed) extra braces are
% needed around the contents of the optional argument to biography to prevent
% the LaTeX parser from getting confused when it sees the complicated
% \includegraphics command within an optional argument. (You could create
% your own custom macro containing the \includegraphics command to make things
% simpler here.)
\begin{IEEEbiographynophoto}{Author Name}
Biography text here.
\end{IEEEbiographynophoto}

% insert where needed to balance the two columns on the last page with
% biographies
%\newpage

\begin{IEEEbiographynophoto}{Coauthor Name}
Biography text here.
\end{IEEEbiographynophoto}

% that's all folks
\end{document}
