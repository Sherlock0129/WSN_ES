\section{Discussion}
\label{sec:discussion}

本章基于第五章的实验结果,系统讨论本文提出的信息与能量双生系统如何解决引言中提出的核心问题:\textbf{信息系统独立于能量系统}导致的性能瓶颈。实验数据表明,通过五层架构的协同工作,系统在不增加总能量消耗的前提下,同时实现了网络寿命的显著延长(从4320分钟延长至10080分钟,提升133.3\%)和能量分布的充分平衡(CV从0.35降至0.18,降低48.6\%)。本章从"信息滞后消除""状态透明化""能量信息协同优化""通信开销内生化"四个维度,深入分析机制设计的有效性及其协同效应。

\subsection{信息滞后消除:动态AOEI上限机制的有效性}

引言指出,传统WSN中信息系统的优先级信号与能量系统的决策相互独立,信息滞后导致调度在过时状态下作出决策。实验1的结果直接验证了动态AOEI上限机制在消除信息滞后方面的有效性。

实验数据显示,动态AOI上限机制将平均网络AOEI从142.7分钟降至85.3分钟,降低40.2\%(\(p < 0.001\))。这一改善源于机制的核心设计:根据节点信息量自适应调整AOI上限,高信息量节点更快触发上报,从而减少信息年龄累积。AOEI分布的75\%分位数从静态的165分钟降至95分钟,表明信息更新更及时,信息新鲜度显著提升。更重要的是,触发误判率从18.7\%降至8.3\%,降低55.6\%,说明动态阈值机制能够更准确地识别需要能量共享的节点,减少不必要的触发。

信息新鲜度的提升直接转化为调度决策质量的改善。实验1中,动态AOI上限机制使首个节点死亡时间从4320分钟(3天,11个节点死亡)延长至10080分钟(7天,无节点死亡),验证了"信息滞后消除→调度准确性提升→网络寿命延长"的因果链条。这一结果支持了引言中的核心观点:通过将信息新鲜度(AOEI)与节点能量紧迫度统一映射为优先级信号,系统能够更准确地识别能量需求,从而在延长网络寿命的同时平衡能量分布。

\subsection{状态透明化:数字孪生账户的必要性与有效性}

引言强调,传统方法中信息收集需要独立的通信开销,形成"信息系统独立于能量系统"的问题。实验2通过对比开启与关闭数字孪生账户的效果,证明了数字孪生状态同步技术在信息滞后场景下的必要性与有效性。

实验数据显示,开启数字孪生账户时,能量估算误差仅为3.2\%(95\% CI: [2.8\%, 3.6\%]),而关闭后误差增至15.8\%(95\% CI: [14.2\%, 17.4\%]),误差增加394\%。关闭数字孪生后,能量轨迹出现明显漂移,最大漂移量达到12.5\%,且漂移随时间累积。这一现象的根本原因是:依赖周期上报(60分钟周期)时,节点状态信息严重滞后,调度器基于过期状态作出决策,导致能量估算偏差不断放大。

错误调度决策次数的对比进一步证明了数字孪生账户的必要性。开启数字孪生时,基于过期状态的错误调度决策次数为23次/7天;关闭后增至156次/7天,增加578\%。这一数据表明,数字孪生账户通过理论能量计算维护虚拟节点能量表,能够显著减少因信息滞后导致的错误调度与资源浪费。

值得注意的是,数字孪生账户在降低通信开销方面同样有效。开启数字孪生时,信息收集能耗为21.01 kJ;关闭后(依赖周期上报)为97.84 kJ,增加365\%。这一结果看似矛盾,实则揭示了数字孪生账户的核心优势:通过理论能量计算,系统能够在低通信开销下保持全局状态的新鲜度,避免因信息滞后导致的频繁错误调度,从而在整体上降低通信开销。实验2的结果验证了引言中的技术效果:数字孪生状态同步技术实现低通信开销下的实时高保真状态同步,显著减少因信息滞后导致的错误调度与资源浪费。

\subsection{能量信息协同优化:ALDP自适应时长规划的有效性}

引言指出,传统方法未能在时长规划中考虑信息价值,无法实现能量与信息的协同优化。实验3通过对比ALDP自适应时长规划与固定时长的效果,验证了能量与信息协同优化的有效性。

实验数据显示,ALDP的正向调度占比为78.5\%,固定时长为52.3\%,提升50.1\%(\(p < 0.001\))。ALDP的Total Score在初期波动后,10步移动平均线逐渐收敛到稳定正值区间(平均+2.3),而固定时长的移动平均线在零值附近波动(平均-0.1)。这一结果表明,ALDP通过综合考虑能量收益、损耗惩罚、时效惩罚和信息奖励四个因素,能够动态选择最优传输时长,使整体网络评分持续提升。

ALDP的最优时长选择机制体现了能量与信息的协同优化。虽然ALDP选择的平均最优时长(3.2分钟)与固定时长(3.0分钟)相近,但ALDP的Score(\(\tau^*\))平均值为+4.5,固定时长为+1.8,提升150\%。这一差异源于ALDP在时长规划中考虑了信息价值:当接收节点有待上报信息时,ALDP会适当延长传输时长以收集更多信息,从而在能量传输的同时实现信息收集,实现能量与信息的协同优化。

信息采集率的提升进一步验证了能量与信息协同优化的有效性。ALDP的信息采集率为85.2\%,固定时长为62.1\%,提升37.2\%。这一结果表明,ALDP通过将信息奖励纳入评分函数,能够激励系统在能量传输过程中收集信息,从而提升信息采集效率。实验3的结果验证了引言中的技术效果:ALDP自适应时长规划技术实现能量与信息的协同优化,从根本上整合能量与信息系统。

\subsection{通信开销内生化:EETOR机会主义上报的有效性}

引言强调,传统方法中信息收集需要独立的通信开销,导致高昂的通信成本,进而形成能量空洞和网络能量分布失衡。实验4通过对比EETOR机会主义上报与最短路径算法的效果,验证了信息收集成本内生化的有效性。

实验数据显示,EETOR的独立上报次数为504次/7天,最短路径算法为1543次/7天,减少67.3\%(\(p < 0.001\))。EETOR的信息收集能耗为21.01 kJ,最短路径算法为97.84 kJ,降低78.5\%。这一显著改善源于EETOR的核心设计:利用传输路径上的中继节点收集沿途节点信息,将信息收集从独立的网络行为转变为能量传输的"副产品"。

值得注意的是,EETOR在减少通信开销的同时,并未牺牲路径效率。EETOR的平均路径效率为51.9\%,最短路径算法为49.2\%,提升5.5\%。这一结果表明,EETOR在路径选择时综合考虑能量传输效率和信息收集增益,能够实现能量传输与信息收集的协同优化。EETOR路径上的信息捎带率达到87.3\%,验证了信息收集作为传能副产品的有效性。

实验4的结果验证了引言中的核心观点:EETOR机会主义信息收集技术将信息收集从独立的网络行为转变为能量传输的"副产品",从根本上整合能量与信息系统,降低通信开销。这一机制设计解决了"信息系统独立于能量系统"问题的关键环节:通过将信息收集附着于能量传输路径,系统消除了独立通信开销,从而在降低通信成本的同时提升网络透明度。

\subsection{跨层协同效应:四层机制协同突破传统权衡}

引言提出的核心问题是:传统能量共享WSN在延长网络寿命与平衡能量分布之间难以同时优化。实验5通过对比四层机制协同工作与基线系统的效果,验证了系统突破传统权衡的有效性。

实验数据显示,提案方法的CV为0.18(95\% CI: [0.16, 0.20]),基线系统为0.35(95\% CI: [0.33, 0.37]),降低48.6\%(\(p < 0.001\))。同时,提案方法的传输效率为51.9\%,基线系统为49.2\%,提升5.5\%。在(CV, 传输效率)二维平面上,提案方法位于基线系统的右上方,实现了公平性和效率的同时提升,突破了传统权衡。

网络寿命的显著延长进一步验证了跨层协同的有效性。提案方法的首个节点死亡时间为10080分钟(7天,无节点死亡),基线系统为4320分钟(3天,11个节点死亡),延长133.3\%。这一改善源于四层机制的协同工作:信息层的动态AOEI上限机制提升了信息新鲜度,减少了触发误判;虚拟节点层的数字孪生账户减少了错误调度决策;调度层的ALDP实现了能量与信息的协同优化;路由层的EETOR降低了通信开销。四层机制相互耦合、协同工作,共同实现了网络寿命的显著延长和能量分布的充分平衡。

通信开销的显著降低进一步证明了跨层协同的有效性。提案方法的信息收集能耗为21.01 kJ,基线系统为97.84 kJ,降低78.5\%。这一改善主要源于EETOR机会主义上报机制和数字孪生账户的协同作用:EETOR将信息收集附着于能量传输路径,数字孪生账户通过理论能量计算减少独立上报需求,两者协同工作,从根本上降低了通信开销。

\subsection{机制设计的系统性与可解释性}

实验结果表明,本文提出的信息与能量双生系统并非简单叠加多个机制,而是围绕"信息滞后消除→状态透明化→能量信息协同优化→通信开销内生化"这一统一逻辑有机耦合。每个机制都针对"信息系统独立于能量系统"问题的特定环节,通过协同工作实现整体性能提升。

从系统设计视角看,五层架构的协同工作形成了"信息驱动能量、能量承载信息"的双生闭环:信息层通过动态AOEI上限机制控制能量信息上报时机,并将AOEI值纳入奖励函数;虚拟节点层通过数字孪生状态同步技术维护虚拟节点能量表,减少信息滞后影响;调度层通过ALDP技术实现前瞻性传输时长规划,实现能量与信息的协同优化;路由层通过EETOR技术实现能量传输与信息收集的深度融合;可视化层提供实时监控与反馈。这五层架构相互耦合、协同工作,从根本上整合了能量与信息系统,解决了"信息系统独立于能量系统"这一核心问题。

实验数据支持了引言中提出的核心结论:通过机制设计创新,系统能够在不增加总能量消耗的前提下,同时实现网络寿命的显著延长和能量分布的充分平衡。这一结论不仅验证了机制设计的有效性,也为未来自治型能量共享网络提供了新的理论框架与可实施路径。
