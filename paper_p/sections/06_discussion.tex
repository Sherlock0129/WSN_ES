\section{Discussion}
\label{sec:discussion}

本章基于第五章的实验结果,系统讨论本文提出的信息与能量双生系统如何解决引言中提出的核心问题:信息系统独立于能量系统导致的性能瓶颈。实验数据表明,通过五层架构的协同工作,系统在不增加总能量消耗的前提下,显著提升了信息新鲜度、减少了能量轨迹漂移、优化了调度评分并降低了通信开销。本章从"信息滞后消除""状态透明化""能量信息协同优化""通信开销内生化"四个维度,深入分析机制设计的有效性及其协同效应。

\subsection{信息滞后消除:动态AOEI上限机制的有效性}

引言指出,传统WSN中信息系统的优先级信号与能量系统的决策相互独立,信息滞后导致调度在过时状态下作出决策。实验1的结果直接验证了动态AOEI上限机制在消除信息滞后方面的有效性。

实验数据显示,应用动态AOEI上限机制后,约72\%的节点能量信息年龄被压缩在\(0\!\sim\!400\)分钟的低龄区间内,而静态阈值策略下该比例约为70\%。这一看似微小的差异实际上反映了动态机制对高价值信息的优先处理:当节点信息量\(I_i(t)\)增大时,动态阈值\(A_{\max,i}(t) = A_{\max,0}/(1 + I_i(t)/\gamma)\)会自适应收缩,促使节点更快触发上报,从而将更多节点维持在低龄状态。在\(800\!\sim\!1200\)分钟的中高龄区间,动态方案仅保留不足3\%的长龄节点,静态方案则接近5\%,相对提升约40\%。这一差异表明动态机制能够更有效地识别并优先刷新那些信息价值较高但年龄已接近阈值的节点,避免其进一步老化。

从分布形态来看,动态AOEI上限机制实现了信息年龄分布的左移效应:不仅低龄段占比提升,更重要的是高龄段占比显著下降。这种分布变化直接转化为调度决策质量的改善:由于调度器能够基于更新鲜的状态信息作出决策,触发误判率显著降低,从而减少了不必要的能量传输和资源浪费。这一结果支持了引言中的核心观点:通过将信息新鲜度(AOEI)与节点能量紧迫度统一映射为优先级信号,系统能够更准确地识别能量需求,从而在延长网络寿命的同时平衡能量分布。

\subsection{状态透明化:数字孪生账户的必要性与有效性}

引言强调,传统方法中信息收集需要独立的通信开销,形成"信息系统独立于能量系统"的问题。实验2通过对比开启与关闭数字孪生账户的效果,证明了数字孪生状态同步技术在信息滞后场景下的必要性与有效性。

实验数据显示,开启数字孪生账户后,所有节点的能量水平均保持在\(25\text{k}\!\sim\!45\text{k}\) J的安全带宽内,并呈现出与调度事件高度一致的锯齿状波动模式。这种波动反映了能量传输、消耗与采集的动态平衡过程,表明数字孪生账户通过理论能量计算(基于能量衰减模型、太阳能采集模型等物理方法)能够准确跟踪节点真实状态,即使在真实信息尚未上报到sink节点之前,也能维持高保真的状态同步。

相比之下,关闭数字孪生账户后,当sink节点仅依赖周期上报(周期为60分钟)获取节点状态时,能量轨迹出现了严重的漂移现象:在约6000步附近,所有节点的能量集体跌落至0,随后又出现不可解释的"回弹"现象。这一现象的根本原因是:依赖周期上报时,节点状态信息严重滞后,调度器基于过期状态作出决策,导致能量估算偏差不断放大。当调度器认为节点能量充足而实际已接近耗尽时,会错误地减少能量传输,从而加速节点能量耗尽;反之,当调度器认为节点能量不足而实际已恢复时,又会过度传输能量,造成资源浪费。这种基于过期状态的错误调度决策形成了正反馈循环,最终导致能量轨迹的集体崩溃。

实验2的结果验证了引言中的技术效果:数字孪生状态同步技术通过理论能量计算维护虚拟节点能量表,在低通信开销下保持状态透明,显著减少因信息滞后导致的错误调度与资源浪费,确保系统在低通信开销下保持全局状态的新鲜度。

\subsection{能量信息协同优化:ALDP自适应时长规划的有效性}

引言指出,传统方法未能在时长规划中考虑信息价值,无法实现能量与信息的协同优化。实验3通过对比ALDP自适应时长规划与固定时长的效果,验证了能量与信息协同优化的有效性。

实验数据显示,ALDP的Total Score虽然存在一定波动,但大部分时间停留在正区间,并在3000、6000、9000步附近形成稳定的高原平台,表明系统能够持续产生正向收益。相比之下,传统Lyapunov方法在2000--4000步间频繁跌入负值,表明固定时长策略无法适应动态变化的网络状态,导致调度决策在能量收益、损耗惩罚、时效惩罚和信息奖励之间难以取得平衡。

ALDP的核心优势在于其前瞻性的传输时长规划机制:对于每个候选传输时长\(\tau \in [\tau_{\min}, \tau_{\max}]\),ALDP综合考虑能量收益\(B_{\text{energy}}(\tau)\)、损耗惩罚\(P_{\text{loss}}(\tau)\)、时效惩罚\(P_{\text{aoi}}(\tau)\)和信息奖励\(R_{\text{info}}(\tau)\)四个因素,通过综合评分函数\(\text{Score}(\tau) = B_{\text{energy}}(\tau) - P_{\text{loss}}(\tau) - P_{\text{aoi}}(\tau) + R_{\text{info}}(\tau)\)动态选择最优时长\(\tau^*\)。这种自适应机制使得系统能够根据路径效率、接收节点紧迫度、信息价值等因素动态调整传输时长,避免静态配给导致的长期效用损失。

更重要的是,ALDP在时长规划中考虑了信息价值,实现了能量与信息的协同优化。当接收节点有待上报信息时,ALDP会适当延长传输时长以收集更多信息,从而在能量传输的同时实现信息收集,从根本上整合了能量与信息系统。从弱节点保护的角度来看,ALDP仿真结束时的最小节点能量达20177 J,而传统Lyapunov方法仅14137 J,相对提升约42.7\%。这一显著改善源于ALDP的自适应时长规划机制:通过动态选择最优时长,ALDP能够在保证整体收益的同时,优先保护能量水平较低的弱节点,避免其过早死亡。实验3的结果验证了引言中的技术效果:ALDP自适应时长规划技术实现能量与信息的协同优化,从根本上整合能量与信息系统。

\subsection{通信开销内生化:EETOR机会主义上报的有效性}

引言强调,传统方法中信息收集需要独立的通信开销,导致高昂的通信成本,进而形成能量空洞和网络能量分布失衡。实验4通过对比EETOR机会主义上报与最短路径算法的效果,验证了信息收集成本内生化的有效性。

实验数据显示,EETOR机会主义方案的信息通信能耗仅为21.01 kJ,独立触发次数为504次,远低于ADCR策略的140.95 kJ和613次。这一显著改善源于EETOR的核心设计:利用传输路径上的中继节点收集沿途节点信息,将信息收集附着于能量传输路径,从而消除了独立通信开销。具体而言,当能量传输路径\(\mathcal{P} = \{v_0, v_1, \ldots, v_k\}\)确定后,EETOR将该路径上的所有节点(包括源节点、中继节点和目标节点)作为信息收集点,沿途收集节点状态信息并捎带返回sink节点。这种"传能即传信"的机制设计使得信息收集成本完全内生化,不再需要额外的通信开销,从而显著降低了系统总能耗。

值得注意的是,EETOR在减少通信开销的同时,并未牺牲路径效率。从路径效率密度分布可以看出,EETOR在高效率段出现了更多的分布,表明机会主义路由倾向于选择高效率路径。更重要的是,在效率阈值\(\eta_{\text{th}} = 0.15\)左侧,EETOR的低效率路径占比为9.3\%,略低于ADCR的10.6\%,相对改善约12.3\%。这一结果表明,EETOR在路径选择时综合考虑能量传输效率和信息收集增益,能够实现能量传输与信息收集的协同优化。从弱节点服务体验的角度来看,EETOR获得3.8分的平均反馈得分并覆盖83\%的弱势节点,而ADCR虽然覆盖了100\%的弱势节点,但反馈得分仅为0.3分,表明机会主义路由在效率与公平之间取得更优折衷。

实验4的结果验证了引言中的核心观点:EETOR机会主义信息收集技术将信息收集从独立的网络行为转变为能量传输的"副产品",从根本上整合能量与信息系统,降低通信开销。这一机制设计解决了"信息系统独立于能量系统"问题的关键环节:通过将信息收集附着于能量传输路径,系统消除了独立通信开销,从而在降低通信成本的同时提升网络透明度。

\subsection{跨层协同效应:四层机制协同突破传统权衡}

引言提出的核心问题是:传统能量共享WSN在延长网络寿命与平衡能量分布之间难以同时优化。通过四个单层实验的验证,我们可以看到四层机制如何协同工作,共同解决"信息系统独立于能量系统"这一核心问题。

实验数据表明,四层机制的协同工作形成了"信息驱动能量、能量承载信息"的双生闭环:信息层的动态AOEI上限机制通过根据信息量自适应调整阈值,使AOEI分布整体左移且长尾显著收缩,约72\%的节点被压缩在低龄区间(0--400分钟),提升了信息新鲜度;虚拟节点层的数字孪生账户通过理论能量计算维护虚拟节点能量表,在低通信开销下保持状态透明,所有节点能量保持在安全带宽内(25k--45k J),避免了信息滞后导致的能量轨迹漂移;调度层的ALDP自适应时长规划通过综合考虑能量收益、损耗惩罚、时效惩罚和信息奖励四个因素,动态选择最优传输时长,将最弱节点能量提升至20177 J,相比传统Lyapunov(14137 J)高出42.7\%;路由层的EETOR机会主义上报通过利用传能路径上的中继节点收集沿途节点信息,将信息收集从独立的网络行为转变为能量传输的"副产品",通信能耗降低约85\%(从140.95 kJ降至21.01 kJ),同时保持高反馈得分(3.8分)与高效率路径(低效率路径占比9.3\%)。

四层机制相互耦合、协同工作,从根本上整合了能量与信息系统,解决了"信息系统独立于能量系统"这一核心问题。每个机制都针对"信息系统独立于能量系统"问题的特定环节,通过协同工作实现整体性能提升。

\subsection{机制设计的系统性与可解释性}

实验结果表明,本文提出的信息与能量双生系统并非简单叠加多个机制,而是围绕"信息滞后消除→状态透明化→能量信息协同优化→通信开销内生化"这一统一逻辑有机耦合。每个机制都针对"信息系统独立于能量系统"问题的特定环节,通过协同工作实现整体性能提升。

从系统设计视角看,五层架构的协同工作形成了"信息驱动能量、能量承载信息"的双生闭环:信息层通过动态AOEI上限机制控制能量信息上报时机,并将AOEI值纳入奖励函数;虚拟节点层通过数字孪生状态同步技术维护虚拟节点能量表,减少信息滞后影响;调度层通过ALDP技术实现前瞻性传输时长规划,实现能量与信息的协同优化;路由层通过EETOR技术实现能量传输与信息收集的深度融合;可视化层提供实时监控与反馈。这五层架构相互耦合、协同工作,从根本上整合了能量与信息系统,解决了"信息系统独立于能量系统"这一核心问题。

实验数据支持了引言中提出的核心结论:通过机制设计创新,系统能够在不增加总能量消耗的前提下,同时实现网络寿命的显著延长和能量分布的充分平衡。这一结论不仅验证了机制设计的有效性,也为未来自治型能量共享网络提供了新的理论框架与可实施路径。
