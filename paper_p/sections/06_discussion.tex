\section{Discussion}
\label{sec:discussion}

本章基于第五章的实验结果与分析,从"网络寿命延长""能量平衡改善""路径治理""公平鲁棒"四条主线,对本文提出的机制体系进行系统讨论。与第四章的理论推导相呼应,本章的讨论严格围绕实验设计中的四类对照(E1--E4)、核心指标体系(网络寿命、能量平衡度、效率、振荡等)展开,旨在给出机制有效性的统一解释框架。

\subsection{AoEI 与数字孪生驱动的网络寿命延长与能量平衡}

实验 E1(智能被动 vs 固定周期)与 E2(信息价值剖析)直接验证了 AoEI 优先级信号以及 数字孪生账户 数字孪生缓存机制的理论效应。结果表明,在统一的资源约束和网络规模下,引入 AoEI 与数字孪生后,系统在网络寿命和能量平衡两个核心目标上均取得了显著改善。

首先,在能量预算相同的前提下,优先级驱动的智能被动触发替代固定周期触发,使能量传输事件与"能量稀缺—信息价值—网络状态"三者紧密耦合。实验结果显示:优先级触发机制能够有效减少无效或过早的传能行为,显著降低调度振荡与重配频率,并在网络寿命、能量效率与能量均衡度(CV 与方差)上同时优于固定周期策略。

其次,数字孪生账户 数字孪生与机会主义上报显著降低了信息获取成本。与周期直发或简单聚类上报相比,数字孪生提供了一种在“有限通信成本”下获得“近似全局态势”的方式,从而提高触发判断的准确性并减少冷却周期内的误触发。实验数据表明,引入 数字孪生账户 后,触发事件在时间上的分布更加平滑,能量分布的振荡幅度能够控制在 0.1 以下。

在网络寿命与能量平衡度(以首个节点死亡时间 $T_{\text{death}}$ 和能量变异系数 CV 作为核心指标)构成的二维空间中,启用 AoEI 与数字孪生的机制配置,其实验点整体位于基线方法的右上区域(更高的寿命、更低的 CV),呈现出明显的协同优化效果。这一现象验证了第四章提出的核心观点:通过提升信息价值感知与降低全局状态获取成本,系统在既定资源约束下同时实现了网络寿命的显著延长和能量分布的充分平衡。

\subsection{EETOR 机会主义上报机制与自适应时长规划}

路径治理能力主要通过实验 E3(上报机制对比)和 E4(自适应时长调度)加以验证,并结合路径效率分布、低效路径占比和重配频率等指标进行定量分析。

一方面,EETOR 机会主义上报机制将信息收集从独立的网络行为转变为能量传输的"副产品"。通过将能量传输路径上的中继节点作为信息收集点,在能量传输过程中顺带收集沿途节点信息,实现了能量传输与信息收集的协同,显著降低了独立的通信开销。与无机会主义上报的基线相比,通信开销降低超过 40\%,信息收集效率提升超过 1.5 倍,同时网络透明度显著提升,表明 EETOR 能够有效降低通信开销并提升网络透明度。

另一方面,前瞻候选规模优化和动态强度控制机制(Dynamic-$k$)在高负载和非平稳供给场景中发挥了关键作用。通过前瞻时间窗对未来 60 分钟的需求进行预测,并结合滞回带宽防止频繁切换,系统能够在确保覆盖能力的前提下限制同时激活的供能路径数量,从而避免“过度竞争”导致的大量能量消耗和路径震荡。实验结果显示,Dynamic-$k$ 有效降低了重配频率与高成本路径的同时激活比例,使路径集合更接近理论上的高效率子集。

此外,E3 中对机会主义上报、ADCR 聚合上报与周期直发上报的对比进一步表明,信息层治理与路径治理是相辅相成的:机会主义上报减少了低价值状态的注入,使路径构建基于更精确的收益评估;数字孪生与缓存机制则缓解了即时感知不足带来的估计偏差。这些结果从实验上支持了"路径治理与信息治理协同抑制路径损耗"的机制设计初衷。

\subsection{弱势保护与非平稳自适应机制的公平鲁棒性}

公平性与鲁棒性主要通过 E1、E2 与 E4 三组实验综合验证,考察了弱势节点保护与非平稳自适应时长调度在多种拓扑与昼夜周期供能条件下的表现。

实验结果表明,引入弱势保护机制(包括低能节点优先的权重设计、服务下限约束等)并未导致预期中的显著效率损失。整体来看,在效率下降小于 5\% 的前提下,能量变异系数和方差均改善超过 25\%,低分位能量节点的服务覆盖率与最低服务频度显著提升。这一结果说明,在当前机制配置下,公平性提升与效率下降之间的权衡是温和且可控的,印证了第四章中关于"弱势保护在一定区间内具有净效用"的理论判断。

在昼夜周期明显的可再生供给模型下,非平稳自适应时长调度展现了显著的稳定性优势。若采用固定参数 Lyapunov 调度,系统容易在白天快速累积能量、夜间快速枯竭,能量分布和触发事件序列均出现明显的高频振荡。加入自适应传输时长后,调度策略能够根据当前能量状态与预期供给变化动态调整传输时长与触发节奏,将能量振荡幅度压制在 0.1 以下,且传能事件在昼夜周期中的分布更加平滑,有效降低了因过度调度导致的能量浪费与路径抖动。

此外,信息奖励与 AoEI 驱动的服务优先级调整,使得高价值信息节点能够在保障弱势节点基本服务的前提下,获得相对稳定的调度机会,从而在公平与时效两个维度上同时取得改善。整体上,弱势保护与非平稳自适应机制共同构成了系统的“公平鲁棒性”基础,使网络在面对可再生供给波动和负载变化时仍能维持较高的服务质量与稳定性。

\subsection{多维指标的机制映射与性能结构}

基于第五章的可视化与性能分析,可以将本文提出的关键机制要素与各项评估指标之间建立较为清晰的映射关系:

\begin{itemize}
    \item 效率提升(有效接收/总消耗)主要来源于 AoEI 优先级信号驱动的智能触发、自适应Lyapunov时长规划(ALDP)与 EETOR 机会主义上报机制;
    \item 公平性改善(CV 降低、弱势覆盖率提升)主要由弱势保护权重设计与自适应调度机制保障;
    \item 网络寿命延长(首个节点死亡时间和存活曲线改善)则是优先级信号稳定、误触发减少和路径生产率提升的综合结果;
    \item 稳定性提升(振荡幅度、重配频率降低)归因于数字孪生提供的平滑态势感知、滞回控制与有限候选集约束;
    \item 通信负担的下降与信息新鲜度的维持,则依赖机会主义上报与缓存预测之间的平衡。
\end{itemize}

这些映射关系表明,本文提出的机制体系并非简单叠加多个机制,而是围绕"优先级信号—信息治理—路径治理—自适应调度"这一统一逻辑有机耦合,以延长网络寿命和平衡能量分布为核心目标。在给定能量预算与拓扑约束下,通过机制创新可以同时实现网络寿命的显著延长和能量分布的充分平衡:系统不仅在这两个核心目标上均取得提升,而且在不增加总能量消耗的前提下实现了协同优化。


