\section{Conclusion}
\label{sec:conclusion}

本文围绕无线传感器网络(WSN)在能量受限、信息滞后与路径累积损耗等结构性约束下难以同时实现网络寿命延长和能量平衡的核心问题,从系统设计视角提出了一套信息与能量双生系统。不同于传统主要依赖算法优化的思路,本研究从"信息滞后消除→状态透明化→能量信息协同优化→通信开销内生化"四重机制出发,将能量与信息两类资源的价值、供需与风险进行统一建模,实现了调度体系的可解释性、跨期性与机制化改造。实验结果表明,系统在不增加总能量消耗的前提下,显著提升了信息新鲜度、减少了能量轨迹漂移、优化了调度评分并降低了通信开销,验证了引言中提出的所有技术效果。

\subsection{核心问题与系统设计}

引言系统揭示了传统能量共享WSN的根本性约束:信息系统独立于能量系统,导致信息滞后、路径损耗与静态资源配给的结构性瓶颈。传统方法中,信息上报与能量传输分离,使得状态获取必须承担额外通信开销,进一步削减节点能量并加剧能量空洞。为了延长网络寿命,调度器必须压缩能量流动频率,却因此无法及时平衡空间能量分布;而为追求能量均衡又不得不频繁共享能量,反向缩短寿命。性能瓶颈因此并非单一算法能力不足,而是系统设计缺乏信息—能量协同所致。

基于此,本文提出信息与能量双生系统,通过五层处理架构实现能量与信息的深度融合与协同优化:信息层通过AOEI能量信息年龄优先级技术持续监测各节点的能量信息年龄,系统维护一个动态AOI上限,当节点的能量信息年龄达到该上限时触发能量共享需求信号,同时将AOEI值作为关键因素纳入多个奖励函数;虚拟节点层通过数字孪生状态同步技术为每个物理节点维护一个数字孪生账户,在节点的真实信息尚未上报到sink节点之前,通过理论能量计算维护虚拟节点能量表;调度层基于AOEI优先级信号和数字孪生账户提供的状态信息,采用ALDP自适应时长规划技术,通过自适应参数的Lyapunov优化进行前瞻性传输能量时长规划;路由层负责路径收集与中继决策,通过EETOR机会主义信息收集技术利用传输路径上的中继节点收集沿途节点的状态信息;可视化层从虚拟节点层获取聚合状态,实时监控网络寿命、能量分布、网络状态与能量变化等关键指标。

\subsection{机制设计的有效性验证}

为验证机制的有效性,本文设计了包括信息层(动态AOEI vs 静态AOEI上限)、虚拟节点层(数字孪生账户开启 vs 关闭)、调度层(ALDP自适应时长 vs 传统Lyapunov)、路由层(EETOR机会主义 vs ADCR)在内的四类对照实验,覆盖不同拓扑形式、不同初始能量分布以及强日内非平稳环境,并采用Bootstrap置信区间、Wilcoxon显著性检验进行系统评估。

实验结果显示:(i)信息层机制有效性:动态AOI上限机制通过根据信息量自适应调整阈值,使AOEI分布整体左移且长尾显著收缩,约72\%的节点被压缩在低龄区间(0--400分钟),在\(800\!\sim\!1200\)分钟的中高龄区间,动态方案仅保留不足3\%的长龄节点,静态方案则接近5\%,相对提升约40\%,验证了"高价值信息先刷新"的设计理念;(ii)虚拟节点层机制有效性:数字孪生账户通过理论能量计算维护虚拟节点能量表,在低通信开销下保持状态透明,所有节点能量保持在安全带宽内(25k--45k J),避免了信息滞后导致的能量轨迹漂移和错误调度决策。关闭数字孪生后,能量在约6000步附近集体跌落至0,验证了该机制在信息滞后场景下的必要性;(iii)调度层机制有效性:ALDP自适应时长规划通过综合考虑能量收益、损耗惩罚、时效惩罚和信息奖励四个因素,动态选择最优传输时长,使Total Score大部分时间保持在正区间,并在多个时间点形成稳定高原。更重要的是,ALDP将最弱节点能量提升至20177 J,相比传统Lyapunov(14137 J)高出42.7\%,验证了自适应时长规划在弱节点保护方面的有效性;(iv)路由层机制有效性:EETOR机会主义上报通过利用传能路径上的中继节点收集沿途节点信息,将信息收集从独立的网络行为转变为能量传输的"副产品",通信能耗降低约85\%(从140.95 kJ降至21.01 kJ),独立触发次数从613次降至504次,同时保持高反馈得分(3.8分)与高效率路径(低效率路径占比9.3\%),验证了"传能即传信"机制在降低通信开销方面的有效性。

上述证据表明,通过机制化处理信息价值、路径累积损耗与跨期优化,系统可在不增加能量预算的前提下,同时实现网络寿命的显著延长和能量分布的充分平衡。实验数据支持了引言中提出的核心结论:信息与能量双生系统通过五层架构的协同工作,从根本上整合了能量与信息系统,解决了"信息系统独立于能量系统"这一核心问题。

\subsection{系统设计的理论贡献与实践价值}

从理论层面,本文构建了"系统要素—网络行为—性能结果"的可解释链条,四大机制(AOEI优先级化、数字孪生状态同步、ALDP自适应时长规划、EETOR机会主义上报)互相耦合、协同运行,形成信息与能量双生系统,为WSN能量共享提供一种高效、公平且可解释的系统化新范式。与既有研究相比,本文并非对传统算法进行局部调参,而是从系统设计层面对能量共享网络的运行逻辑进行整体重构,以延长网络寿命和平衡能量分布为核心目标。

从实践层面,本文提出的信息与能量双生系统为未来自治型能量共享网络提供了新的理论框架与可实施路径。系统通过五层架构的协同工作,实现了"信息驱动能量、能量承载信息"的双生闭环,从根本上整合了能量与信息系统。实验数据表明,系统在不增加总能量消耗的前提下,同时实现了网络寿命的显著延长和能量分布的充分平衡,为未来的能源自治网络、可再生驱动的边缘系统以及大规模异质感知网络提供了统一的理论基础与实践方向。

\subsection{研究局限与未来工作}

尽管实验结果从多个维度支持了本文机制设计的有效性与可行性,本研究仍然存在一些局限,值得在后续工作中深入探索:(i)硬件真实性不足:当前仿真模型虽已考虑能量采集模型、传输功率及效率衰减,但尚未完全覆盖硬件非线性、能量测量噪声、无线链路突发衰落等现实因素,仍需在实际平台或硬件在环环境中进一步验证;(ii)参数自适应能力有限:AOEI权重、信息奖励系数、弱势保护权重以及Lyapunov参数\(V\)等目前依赖经验设定,在跨场景迁移时可能需要重新调参,未来可引入强化学习或元调度(meta-scheduling)方法,实现对权重与参数的在线自适应优化;(iii)对极端与对抗性情形的适应性有待加强:现有实验主要针对典型拓扑与非对抗环境,在节点存在策略性行为或恶意攻击的场景下,优先级信号与路径治理机制的鲁棒性仍需进一步研究;(iv)性能结构的理论刻画尚不完备:虽然通过实验观察到了性能边界外移,但对不同机制参数下前沿形态的理论刻画仍不充分,未来可从多目标优化视角,对"边界外移"的充要条件进行更严格的分析。

这些局限也为后续研究指明了方向:将机制进一步嵌入真实硬件平台,并结合学习型调参与机制分析,有望在复杂与对抗环境中持续提升能量协同网络的自适应性与长期可持续性。

\subsection{总结}

综上所述,本文通过AOEI优先级信号、数字孪生信息治理、自适应Lyapunov时长规划(ALDP)与EETOR机会主义上报机制等机制,在理论与实验上验证了"信息滞后消除→状态透明化→能量信息协同优化→通信开销内生化"四大方向的联合可行性,为构建可解释、可扩展且具技术理性的能量共享WSN提供了可行范式。更为重要的是,本研究展示了机制设计在资源受限网络中的普适性价值:通过改变机制结构,而非仅优化算法细节,可以在不增加总能量消耗的前提下,同时实现网络寿命的显著延长和能量分布的充分平衡,为未来的能源自治网络、可再生驱动的边缘系统以及大规模异质感知网络提供了统一的理论基础与实践方向。
