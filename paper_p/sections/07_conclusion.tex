\section{Conclusion}
\label{sec:conclusion}

本文围绕无线传感器网络(WSN)在能量受限、信息滞后与路径累积损耗等结构性约束下难以同时实现网络寿命延长和能量平衡的核心问题,提出了一套从机制设计视角重构能量共享网络的新框架。不同于传统主要依赖算法优化的思路,本研究从"优先级化—透明化—治理化—自适应化"四重机制出发,将能量与信息两类资源的价值、供需与风险进行统一建模,实现了调度体系的可解释性、跨期性与机制化改造。本文的核心目标是尽可能延长网络寿命并平衡网络能量分布,在理论与实验两方面共同证实了在不增加总能量消耗的前提下,同时实现网络寿命的显著延长和能量分布的充分平衡。

从建模层面,本文系统揭示了传统能量共享 WSN 的三项根源性约束:一是信息价值未被优先级化导致触发错配;二是状态获取滞后导致调度在过时状态下作出决策;三是多跳能量传输的累积损耗与弱势节点过度暴露造成显著路径损耗。围绕这些机制缺口,本文构建了由节点能量动态、链路效率、可再生供给、数字孪生状态缓存以及能量信息年龄(AOEI)衰减等组成的统一系统模型,为后续机制创新提供了严格的语义空间与约束基础。

在机制设计方面,本文提出了四项核心机制模块:AoEI 内生化优先级信号用于驱动传能触发与排序,优先服务最需要能量的节点以平衡能量分布;数字孪生账户 数字孪生账户构建低开销高时效的状态透明层,为准确决策提供实时状态信息;自适应Lyapunov时长规划(ALDP)通过综合能量收益、损耗惩罚、时效损失与信息奖励等多维因素实现动态决策,在延长寿命和平衡能量之间找到最优平衡点;EETOR 机会主义上报机制将能量传输路径上的中继节点作为信息收集点,在能量传输过程中顺带收集沿途节点信息,实现能量传输与信息收集的协同,降低通信开销。此外,\emph{AdaptiveLyapunovScheduler} 和 \emph{AdaptiveDurationAwareLyapunovScheduler} 进一步引入了基于网络反馈分数的参数自适应调整机制,通过评估每次调度对网络整体状态的影响(能量均衡性、存活率、传输效率和整体能量水平),动态调整 Lyapunov 权重参数 \(V\),使系统能够在不同场景下自动优化权衡策略,进一步提升在非平稳环境下的鲁棒性和性能表现。上述机制之间并非孤立,而是构成了一个紧密耦合的"机制网络",在触发、上报、路径构建和跨期分配四条链路上分别抑制结构性低效,在不增加总能量消耗的前提下,同时实现网络寿命的显著延长和能量分布的充分平衡。

为验证机制的有效性,本文设计了包括智能被动 vs 固定周期(E1)、信息价值剖析(E2)、上报机制对比(E3)以及自适应时长调度(E4)在内的四类对照实验,覆盖不同拓扑形式、不同初始能量分布以及强日内非平稳环境,并采用 Bootstrap 置信区间、Wilcoxon 显著性检验进行系统评估。实验结果显示:(i)AoEI 与数字孪生显著提升触发准确性、降低振荡,在网络寿命和能量平衡两个核心目标上均取得显著改善;(ii)自适应Lyapunov时长规划(ALDP)通过动态调整传输时长,在延长寿命和平衡能量之间找到最优平衡点;(iii)EETOR 机会主义上报机制有效降低通信开销,在能量传输过程中顺带收集沿途节点信息,提升网络透明度;(iv)弱势节点保护在保持效率损失小于 5\% 的条件下,使能量均衡度(CV)改善超过 25\%,显著改善了能量分布平衡度,并将昼夜周期中的能量振荡控制在 0.1 以下;(v)基于网络反馈分数的参数自适应调整机制进一步提升了系统在非平稳环境下的鲁棒性,能量振荡幅度降低 \(15\%-25\%\),同时保持参数调整的稳定性。上述证据表明,通过机制化处理信息价值、路径累积损耗与跨期优化,并在调度层面引入反馈驱动的自适应机制,可在不增加能量预算的前提下,同时实现网络寿命的显著延长和能量分布的充分平衡。

%尽管实验结果从多个维度支持了本文机制设计的有效性与可行性,本研究仍然存在一些局限,值得在后续工作中深入探索:
%\begin{itemize}
 %   \item \textbf{硬件真实性不足:} 当前仿真模型虽已考虑能量采集模型、传输功率及效率衰减,但尚未完全覆盖硬件非线性、能量测量噪声、无线链路突发衰落等现实因素,仍需在实际平台或硬件在环环境中进一步验证。
 %   \item \textbf{参数自适应能力有限:} AOEI 权重、信息奖励系数、弱势保护权重以及 Lyapunov 参数 $V$ 等目前依赖经验设定,在跨场景迁移时可能需要重新调参。未来可引入强化学习或元调度(meta-scheduling)方法,实现对权重与参数的在线自适应优化。
 %   \item \textbf{对极端与对抗性情形的适应性有待加强:} 现有实验主要针对典型拓扑与非对抗环境,在节点存在策略性行为或恶意攻击的场景下,优先级信号与路径治理机制的鲁棒性仍需进一步研究。
 %   \item \textbf{性能结构的理论刻画尚不完备:} 虽然通过实验观察到了性能边界外移,但对不同机制参数下前沿形态的理论刻画仍不充分,未来可从多目标优化视角,对"边界外移"的充要条件进行更严格的分析。
%\end{itemize}
%这些局限也为后续研究指明了方向:将机制进一步嵌入真实硬件平台,并结合学习型调参与机制分析,有望在复杂与对抗环境中持续提升能量协同网络的自适应性与长期可持续性。

综上所述,本文通过 AoEI 优先级信号、数字孪生信息治理、自适应Lyapunov时长规划(ALDP)与EETOR机会主义上报机制等机制,在理论与实验上验证了"延长网络寿命—平衡能量分布—机会主义上报—公平鲁棒"四大方向的联合可行性,为构建可解释、可扩展且具技术理性的能量共享 WSN 提供了可行范式。更为重要的是,本研究展示了机制设计在资源受限网络中的普适性价值:通过改变机制结构,而非仅优化算法细节,可以在不增加总能量消耗的前提下,同时实现网络寿命的显著延长和能量分布的充分平衡,为未来的能源自治网络、可再生驱动的边缘系统以及大规模异质感知网络提供了统一的理论基础与实践方向。
