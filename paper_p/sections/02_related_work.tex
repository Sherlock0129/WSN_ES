\section{Related Work}

无线传感器网络(WSN)能量共享研究长期围绕节能调度、链路选择、机会式路由、Lyapunov优化以及多目标调度等工程化策略展开。这些工作普遍在既定系统假设下追求局部性能提升,却难以突破信息滞后、路径损耗与静态资源配给的结构性瓶颈。正如引言所述,传统方案遵循的是"约束内优化"范式:能量预算、通信开销与公平性约束均被视作外生条件,导致网络寿命与能量平衡无法协同跃升。

\textbf{关键掣肘在于信息系统与能量系统相互独立}。信息上报与能量传输分离,使得状态获取必须承担额外通信开销,进一步削减节点能量并加剧能量空洞。为了延长网络寿命,调度器必须压缩能量流动频率,却因此无法及时平衡空间能量分布;而为追求能量均衡又不得不频繁共享能量,反向缩短寿命。性能瓶颈因此并非单一算法能力不足,而是系统设计缺乏信息—能量协同所致。

已有工作大体可分为三类系统设计思路:\textbf{局部优化型、集中调度型与分布式探索型},但都未能建立信息—能量的一体化协同。

\textbf{局部优化型系统}以节能调度、链路选择、机会式路由、传输功率控制等工程手段为代表,通常在静态或准静态假设下调优局部指标。它们侧重减少节点能量不均衡、缩短传输路径或抑制能量损耗,却把状态时效性视为外部输入:信息由固定周期或静态阈值触发,无法根据能量紧迫度实时调整。信息上报的通信成本与能量传输完全分离,导致在提升寿命与维持能量均衡之间始终存在硬性权衡。

\textbf{集中调度型系统}依托Lyapunov优化、凸优化或混合整数规划,从全局视角推导可收敛的调度策略,并通过虚拟队列、先验权重等方式平衡不同目标。此类系统具备较强的理论保证,但高度依赖预设参数与完备信息,缺乏对非平稳环境、异构节点与突发事件的自适应能力。更重要的是,它们通常要求节点独立上报最新状态,调度器再集中决策,信息与能量流仍然沿两条相互独立的通道运行。

\textbf{分布式探索型系统}近年利用分布式资源分配、AOEI度量、数字孪生、深度强化学习或机会式中继等概念,以期降低中心节点压力并提升可扩展性。然而,多数研究只是在局部模块试验某一功能:AOEI被当作性能指标而非调度信号,数字孪生用于离线监测或仿真而未与能量分配闭环,强化学习依赖高频状态采样而难以在信息滞后场景稳定运行。由于缺乏统一的优先级、账户、协调与路径治理体系,信息收集依旧需要独立通信,沿途节点也未被系统性地纳入状态采集流程。

尽管上述各阶段取得一定进展,信息/能量分离依旧导致典型的系统级失效。可以识别出四类关键系统缺口,均源于状态更新与能量调度无法同源:

\textbf{(i)优先级信号缺位导致系统解耦}——信息新鲜度、价值与紧急性缺乏统一量化,信息龄(AOEI)未能内生为调度信号,触发传能的时机和强度无法与系统目标一致。

\textbf{(ii)信息基础设施薄弱导致独立通信开销}——状态感知、时间同步、AOEI度量、节点上报协议等底座仍依赖静态或粗粒度上报(典型间隔\(\ge 30\)分钟),缺少事件驱动与跨层一致性,使调度、路由与信息层状态错位。\textbf{最关键的是,信息上报仍无法附着于传能路径,必须独立通信,进一步侵蚀能量预算并诱发信息不对称。}

\textbf{(iii)传输时长静态化阻碍协同}——多数调度算法预设固定传输时长,忽视路径效率、受助节点紧迫度与信息价值,无法通过自适应Lyapunov规划在寿命、效率与公平之间取得动态平衡。

\textbf{(iv)路径选择未嵌入信息收集}——通用路由准则重吞吐、轻能量效率阈值\(\eta_{\text{th}}\),缺乏跨周期的全局控制策略;多跳损耗难以抑制,且中继节点未被视为实时状态采集点,信息收集仍是独立网络行为。

基于系统设计逻辑,必须同时重构优先级形成、状态透明、时长规划与机会主义上报四大系统要素,才能真正打通信息与能量双生循环。本文据此构建了一个结合"优先级形成 + 信息透明 + 时长规划 + 机会主义上报"的系统框架,四个系统要素(AOEI优先级化、数字孪生状态同步、ALDP自适应时长规划、EETOR机会主义上报)共同实现\textbf{信息与能量双生系统}。针对上述缺口,本文在统一框架下的要素映射见表~\ref{tab:mechanism_mapping},形成"低开销—高时效—强可解释"的系统范式,并以"性能边界外移"作为统一评估主张。

特别地,系统通过四个要素的协同工作,完成信息/能量的深度整合:\textbf{AOEI优先级要素}依据动态AOI上限触发路由终点上报,并将AOEI值纳入多个奖励函数;\textbf{数字孪生状态同步要素}在真实状态尚未抵达sink前由理论能量计算维护虚拟能量表,并沿传能路径同步,避免独立通信;\textbf{ALDP要素}利用自适应Lyapunov优化进行前瞻性的传能时长规划,将信息价值纳入供给决策;\textbf{EETOR要素}优先把其他路由目标节点纳入中继,沿传能路径采集并回传状态,使信息收集成为能量传输的伴生过程。

与既有研究相比,本文构建了"系统要素—网络行为—性能结果"的可解释链条,四大要素互相耦合、协同运行,形成信息与能量双生系统,为WSN能量共享提供一种高效、公平且可解释的系统化新范式。

此外,现有文献鲜少触及引言中所述的五层系统协同:信息层、虚拟节点层、调度层、路由层与可视化层往往被拆分为各自孤立的算法模块。主流设计通常将sink节点视作高能量的集中处理者,却忽视其周边节点因频繁转发或能量共享而形成的能量空洞问题,也缺少将状态透明化与路径效率阈值\(\eta_{\text{th}}\)联动的机制。相比之下,本文提出的双生系统把AOEI驱动的优先级信号、数字孪生账户、ALDP时长规划与EETOR机会主义上报嵌入统一框架,使信息沿能量路径回流、能量以信息反馈为先导,进一步弥补了现有研究在跨层整合与闭环治理方面的缺口。

\begin{table}[t]
\centering
\caption{系统要素与实现载体的对应关系}
\label{tab:mechanism_mapping}
\begin{tabular}{p{0.36\linewidth} p{0.56\linewidth}}
\toprule
系统要素 & 实现载体/功能 \\
\midrule
优先级信号化 & AOEI(动态AOI上限确定上报时机,AOEI值纳入奖励函数) \\
信息透明化 & 数字孪生账户(理论能量计算维护虚拟节点能量表) \\
时长规划优化 & ALDP(自适应Lyapunov优化,能量与信息协同优化) \\
机会主义上报 & EETOR(利用传输路径上的中继节点收集信息,整合能量与信息系统) \\

\bottomrule
\end{tabular}
\end{table}

