\section{Related Work}

在无线传感器网络(WSN)的能量共享场景中,现有研究多采用工程优化思路,包括节能调度、链路选择、机会式路由、Lyapunov优化以及多目标调度等方法。这些方法确实能够在既定系统条件下提升能量利用效率或在特定配置下实现局部最优,然而,它们都存在一个共同的结构性限制:其优化过程本质上受限于既定的系统框架,往往在延长网络寿命与平衡能量分布之间难以同时兼顾。从方法论视角看,此类方法遵循的是"约束内优化"范式,其改进表现为在既定约束条件下寻求更优解,但难以从根本上解决网络寿命与能量平衡的协同优化问题。

\textbf{核心问题在于传统能量共享WSN中信息系统独立于能量系统}。在传统决策规则下,信息收集需要独立的通信开销,形成"信息系统独立于能量系统"的问题,导致高昂的通信开销进而形成能量空洞,网络能量分布失衡。传统方法往往面临以下困境:延长网络寿命需要减少能量传输以降低整体消耗,但这会加剧能量分布不均衡;而平衡能量分布需要频繁的能量共享,但这又会加速能量消耗,缩短网络寿命。在此框架下,无论设计多少新的调度算法或启发式方法,都难以同时实现网络寿命的显著延长和能量分布的充分平衡。这正是传统能量共享WSN难以进一步突破的重要原因:其性能瓶颈不是技术层面的,而是\textbf{机制设计层面的}。

从机制演化的视角审视,节点能量共享的研究进程可类比为一条机制演化的路径,大致经历以下五个阶段。然而,这些阶段的研究都未能从根本上解决"信息系统独立于能量系统"这一核心问题:

\textbf{阶段一:无规则盲目交换阶段}。在这一最初阶段,网络中的节点几乎缺乏任何统一规则来指导能量共享,每个节点按照本地经验或随机策略发出能量请求或传输。能量的流动缺乏任何优先级信号,节点间的交换呈现完全盲目状态,效率往往难以保障。更重要的是,信息收集与能量传输完全解耦,信息上报需要独立的通信开销,形成"信息系统独立于能量系统"的典型表现。

\textbf{阶段二:经验性局部优化阶段}。随后,研究者提出了一系列局部优化算法,试图改善网络性能。此类方法以降低节点能量不均衡、减少传输损耗或提高整体能量利用率为核心目标,通常建立在确定性或准静态约束假设下,通过工程手段优化特定指标。然而,此类方法将信息时效性与价值视为外部给定条件,仅在固定假设下获得局部帕累托最优,难以适应动态环境变化与异质节点需求。更重要的是,信息收集仍然需要独立的通信开销,未能整合能量与信息系统。

\textbf{阶段三:集中式调度阶段}。紧接着出现了基于Lyapunov优化或凸优化的集中式调度框架,这些方法从全局视角出发,在理论上可以保证调度策略的收敛性与稳定性,具有较强的分析可解释性。然而,此类方法的能量触发时机与强度往往依赖预先设置的权衡参数(如虚拟队列权重),对非平稳环境与异质场景的自适应能力受限,难以应对网络环境的突变与异构性。更重要的是,这些方法仍然假设信息收集需要独立的通信开销,未能从根本上整合能量与信息系统。

\textbf{阶段四:黑箱式学习阶段}。近年来,深度强化学习(如DQN、DDPG)等黑箱式方法被引入能量调度领域。这类方法可在高维、非线性场景下学习近似最优的调度策略,可端到端地近似最优策略。然而,其内部决策过程缺乏可解释性,且对节点状态信息的可得性与时效性高度依赖,在信息滞后或状态不可观测的场景下性能显著下降。更重要的是,这些方法仍然将信息收集视为独立的网络行为,未能将信息收集与能量传输深度融合。

\textbf{阶段五:初代分布式机制试验阶段}。最近,一些研究开始探索分散式与分布式的能量分配方式,涉及分布式资源分配、AOEI度量或数字孪生在网络管理中的局部应用。然而,这些探索多停留在局部应用层面,尚未形成"优先级信号—账户—协调规则—路径治理"的机制化一体框架,亦未将其系统性地用于扩展可达解集。结果是,各类探索性方法依然局限于特定场景的局部优化,难以在系统层面实现跨多目标的性能改进。截至目前,已有工作的大多数停留于算法或度量层面:AOEI在多数文献中作为性能指标而非内生优先级信号;数字孪生在相当多的应用中用于监测/仿真而未与资源分配闭环;路由算法在绝大多数研究中遵循通用通信准则而缺少针对能量传输效率阈值\(\eta_{\text{th}}\)与多跳累积损耗的机制化治理。更重要的是,这些方法仍然未能从根本上解决"信息系统独立于能量系统"的问题,信息收集仍然需要独立的通信开销,导致高昂的通信成本,进而形成能量空洞和网络能量分布失衡。

尽管上述各阶段的方法实现了不同程度的性能提升,但它们均未能从根本上解决"信息系统独立于能量系统"这一核心问题,导致典型的"系统失效"现象。具体而言,可识别出四方面关键机制缺口,这些缺口都源于"信息系统独立于能量系统"这一根本性问题:

\textbf{(i)优先级信号缺位导致信息与能量系统解耦}——传统方法中信息系统的优先级信号与能量系统的决策相互独立,无法有效整合。"信息新鲜度—价值—紧急性"缺乏统一的量化刻画与决策映射建模,触发传能的时机与强度难以与系统目标同构。信息价值未被量化为优先级信号,无法通过优先级机制内生节点需求强度,导致资源分配缺乏有效的引导机制。更重要的是,传统方法未考虑到信息年龄(AOEI)这一通信量,未能将信息新鲜度与节点能量紧迫度统一映射为优先级信号,使信息与能量系统耦合。

\textbf{(ii)信息基础设施薄弱导致独立通信开销}——支撑能量共享决策闭环的状态感知、时间同步、AOEI度量与流处理、节点上报协议等基础机制存在系统性缺陷。在我们调研的范围内(覆盖2018–2025年WSN/IoT能量共享相关研究),多数方法仍以静态或固定周期上报为主(典型间隔\(\ge 30\)分钟),导致信息年龄常超过60分钟,决策严重滞后;缺乏事件驱动/差分上报机制,无法在关键状态变化时及时响应;缺少跨层指标对齐与一致性缓存,使得调度层、路由层与信息层之间的状态不一致,诱发错误决策。\textbf{最关键的是,信息上报与能量传输解耦,无法利用传能路径"搭载"状态更新,信息收集需要独立的通信开销,形成"信息系统独立于能量系统"的典型表现},通信开销居高不下。上述薄弱环节共同导致系统在时效性与通信开销之间难以实现性能改进,信息滞后成为制约能量共享效率提升的关键瓶颈。信息不对称进一步带来逆向选择问题——部分节点可能隐瞒真实剩余能量以图私利,损害了资源分配的公平性与效率。

\textbf{(iii)传输时长静态化导致能量与信息协同缺失}——在多数调度算法中,传输时长以静态方式设定,无法根据路径效率、接收节点的能量需求、信息新鲜度以及潜在的信息量动态调整供给。传统方法未能在时长规划中考虑信息价值,无法实现能量与信息的协同优化。过短的传输导致频繁启动和控制开销过大,过长的传输虽然一次可送更多能量并带回更多信息,但容易造成能量过度消耗并拉长信息时滞,因而在效率、公平与寿命之间形成不可突破的trade-off。更重要的是,传统方法未能通过自适应参数的Lyapunov优化进行前瞻性传输能量时长规划,使供能强度与传输时长随供需变化动态调整,导致静态配给造成的长期效用损失。

\textbf{(iv)路径选择未考虑信息收集导致系统解耦}——在多数路由算法中沿用通用通信路由准则(以时延或吞吐为目标),未体现能量传输的效率阈值\(\eta_{\text{th}}\)与多跳累积损耗,亦缺少面向系统层的全局控制机制(可类比"监管—协调—约束"的策略组合),难以跨周期维持一致性规则。多跳传输造成的累积损耗无法被有效抑制,导致网络能量利用效率下降,系统只能停留在既定的静态性能前沿之内。\textbf{更重要的是,传统方法未能将能量传输路径上的中继节点作为信息收集点,在能量传输过程中顺带收集沿途节点的状态信息,未能将信息收集从独立的网络行为转变为能量传输的"副产品"},无法从根本上整合能量与信息系统,实现能量传输与信息收集的协同,降低通信开销。

基于机制设计的分析逻辑,必须通过机制创新来突破这些僵局。\textbf{核心在于从根本上解决"信息系统独立于能量系统"的问题,通过机制创新整合能量与信息系统,降低通信开销}。因此,本文的设计出发点是构建一个结合"优先级形成 + 信息透明 + 时长规划 + 机会主义上报"的一体化机制框架,四个机制(AOEI优先级化、数字孪生状态同步、ALDP自适应时长规划、EETOR机会主义上报)共同实现\textbf{信息与能量双生系统}。针对上述缺口,本文在统一框架下构建的要素映射见表~\ref{tab:mechanism_mapping},形成"低开销—高时效—强可解释"的一体化机制,并以"性能边界外移"作为统一评估主张。

特别地,本文通过四个机制的协同工作,从根本上整合了能量与信息系统:\textbf{AOEI优先级机制}通过动态AOI上限确定路由终点节点等多久进行能量上报,并将AOEI值作为关键因素纳入多个奖励函数,使信息与能量系统耦合;\textbf{数字孪生状态同步机制}在节点的真实信息没有上报到sink节点之前,通过理论能量计算维护一个理论的虚拟节点能量表,利用能量传输路径同步状态,避免独立的通信开销;\textbf{ALDP机制}通过自适应参数的Lyapunov优化进行前瞻性传输能量时长规划,在时长规划中考虑信息价值,实现能量与信息的协同优化;\textbf{EETOR机会主义上报机制}优先选取将其他路由的目标节点作为中继节点来收集节点信息,在能量传输过程中收集沿途节点的状态信息并上报到sink,将信息收集从独立的网络行为转变为能量传输的"副产品",从根本上整合能量与信息系统。

与已有工作不同,本文建立"机制—行为—结果"的完整可解释链条,从机制设计层面系统性地揭示机制创新如何驱动系统行为变化并最终实现性能提升。四个机制相互耦合、协同工作,共同实现了信息与能量双生系统,从根本上整合了能量与信息系统,以期为WSN的能量共享提供一种可解释、高效且公平的新范式。

\begin{table}[t]
\centering
\caption{机制要素与实现载体的对应关系}
\label{tab:mechanism_mapping}
\begin{tabular}{p{0.36\linewidth} p{0.56\linewidth}}
\toprule
机制要素 & 实现载体/功能 \\
\midrule
优先级信号化 & AOEI(动态AOI上限确定上报时机,AOEI值纳入奖励函数) \\
信息透明化 & 数字孪生账户(理论能量计算维护虚拟节点能量表) \\
时长规划优化 & ALDP(自适应Lyapunov优化,能量与信息协同优化) \\
机会主义上报 & EETOR(利用传输路径上的中继节点收集信息,整合能量与信息系统) \\

\bottomrule
\end{tabular}
\end{table}

