\section{Introduction}

无线传感器网络(Wireless Sensor Networks, WSN)作为典型的边缘感知基础设施,已在环境监测、智慧城市、工业物联网与精准农业等场景中实现规模化部署。然而,由于节点受限于有限的电池容量以及能量采集的强非平稳性,当前 WSN 在长期运行中普遍面临两类根本性系统瓶颈:其一是能量的空间与时间分布不均衡导致网络寿命缩短与功能退化;其二是状态信息严重滞后,使得路由、调度与能量共享决策建立在过时状态之上,进而诱发相应的系统能量效率损失。

现有研究多采用工程优化思路,包括节能调度、链路选择、机会式路由、Lyapunov 优化以及多目标调度等方法。这些方法确实能够在既定系统条件下提升能量利用效率或在特定配置下实现局部最优。然而,这些方法都存在一个共同的结构性限制:其优化过程本质上受限于既定的系统框架,往往在延长网络寿命与平衡能量分布之间难以同时兼顾。从方法论视角看,此类方法遵循的是"约束内优化"范式,其改进表现为在既定约束条件下寻求更优解,但难以从根本上解决网络寿命与能量平衡的协同优化问题。

在传统能量共享 WSN 中,能量预算、链路容量、信息时延、通信开销与公平性约束均作为外生且固定的系统前提存在。这些固定约束使得系统在延长网络寿命与平衡能量分布之间难以同时优化。传统方法往往面临以下困境:延长网络寿命需要减少能量传输以降低整体消耗,但这会加剧能量分布不均衡;而平衡能量分布需要频繁的能量共享,但这又会加速能量消耗,缩短网络寿命。在此框架下设计的优化算法,都难以同时实现网络寿命的显著延长和能量分布的充分平衡。在传统WSN系统中,通常存在sink节点负责收集和处理传感器节点上传的数据,并将其传输到云端进行进一步处理。sink节点通常具有较高的能量储备,但sink节点周围的传感器节点容易因频繁参与中继转发或能量共享而被耗尽能量,导致过早死亡并形成能量空洞,如何处理能量空洞也是当前WSN系统中的一个重要问题。
% 其中,弱势节点是指能量水平低于低能阈值或处于能量分布低分位的节点。这些节点由于能量储备不足,在网络运行中容易因频繁参与中继转发或能量共享而被耗尽能量,导致过早死亡并形成能量空洞。弱势节点的存在加剧了网络能量分布的不均衡性,使得能量变异系数(CV)增大,网络寿命缩短。传统调度方案往往忽视弱势节点的特殊需求,或仅通过外生公平性约束进行事后补偿,无法从根本上解决能量分布失衡问题。

当系统框架本身不发生改变时,传统能量共享 WSN 难以进一步突破的一个重要原因是其性能瓶颈不是技术层面的,而是\textbf{系统设计层面的}。传统能量共享 WSN 在其默认的决策规则中隐含了多项机制性限制:未考虑到信息年龄这一通信量、节点状态不可得或滞后,从而局限于\textbf{信息系统独立于能量系统},导致高昂的通信开销进而形成能量空洞,网络能量分布失衡等问题。基于此,本文提出一个\textbf{信息与能量双生系统},从根本上整合能量与信息系统,通过智能识别关键节点、优化传输路径、动态调整传输策略,在不增加总能量预算的前提下,同时实现网络寿命的显著延长和能量分布的充分平衡。该系统通过五层处理架构实现能量与信息的深度融合与协同优化。图~\ref{fig:system-model}给出了本文提出的系统模型架构。

系统的运行机制如下:\textbf{信息层}(Information Layer)通过\textbf{能量信息年龄优先级技术}(Age-of-Energy-Information, AOEI)持续监测各节点的能量信息年龄,系统维护一个动态AOI上限,当节点的能量信息年龄达到该上限时触发能量共享需求信号,同时将AOEI值作为关键因素纳入多个奖励函数,用于优化能量传输决策和路径选择。该技术将"能量信息新鲜度 $\times$ 能量紧迫度"作为非中心节点的内生优先级信号,动态驱动能量共享触发、排序和预算清算,突破传统效率—公平权衡。\textbf{虚拟节点层}(Virtual Node Layer)通过\textbf{数字孪生状态同步技术}为每个物理节点维护一个数字孪生账户,在节点的真实信息尚未上报到sink节点之前,通过理论能量计算(基于能量衰减模型、太阳能采集模型等物理方法)维护虚拟节点能量表,实现低通信开销下的实时高保真状态同步,显著减少因信息滞后导致的错误调度与资源浪费,确保系统在低通信开销下保持全局状态的新鲜度。\textbf{调度层}(Scheduling Layer)基于AOEI优先级信号和数字孪生账户提供的状态信息,采用\textbf{自适应时长规划技术}(Adaptive Lyapunov Duration Planning, ALDP),通过自适应参数的Lyapunov优化进行前瞻性传输能量时长规划,依据能量收益、损耗惩罚、信息时效惩罚和信息增益奖励四个因素对候选传输时长进行打分,动态选择综合收益最大的传输时长,使供能强度与传输时长随供需变化自适应调整,避免静态配给导致的长期效用损失。在此基础上,调度层进行全局调度决策,识别能量紧迫且信息过时的节点作为受助者,选择能量充足且路径高效的节点作为施助者,在延长网络寿命与平衡能量分布间进行协调优化,实现能量与信息的协同优化。\textbf{路由层}(Routing Layer)负责路径收集与中继决策,优先选择将其他路由的目标节点作为中继节点,形成多跳能量传输路径。在能量传输过程中,路由层通过\textbf{机会主义信息收集技术}(Energy-Efficient Transfer Opportunistic Routing, EETOR)利用传输路径上的中继节点收集沿途节点的状态信息,将信息收集从独立的网络行为转变为能量传输的"副产品",实现能量传输与信息收集的深度融合,从根本上整合能量与信息系统,降低通信开销。收集到的信息反馈回信息层,更新数字孪生账户状态,形成闭环反馈。\textbf{可视化层}从虚拟节点层获取聚合状态,实时监控网络寿命、能量分布、网络状态与能量变化等关键指标。

该系统通过上述运行机制实现了信息与能量的深度融合:信息流自下而上汇聚形成虚拟节点状态与调度决策,能量流自上而下沿路由执行传输,同时在传输链路上回传状态信息,形成"信息驱动能量、能量承载信息"的双生闭环。虚线表示节点信息流,实线表示能量调度信息流。通过这种机制设计,系统能够在不增加总能量预算的前提下,智能地识别关键节点、优化传输路径、动态调整传输策略,同时实现网络寿命的显著延长和能量分布的充分平衡。

\begin{figure*}[t]
    \centering
    \includegraphics[width=0.6\linewidth]{sections/figures/s2NEW.png}
    \caption{系统模型示意图。信息自下而上汇聚形成虚拟节点与调度决策,能量自上而下沿路由执行传输,同时在传输链路上回传状态信息,最终在可视化层统一展示。}
    \label{fig:system-model}
\end{figure*}

本文的主要贡献如下:

1) 从系统设计角度系统性揭示了传统能量共享 WSN 在延长网络寿命与平衡能量分布之间难以同时优化的根本原因,说明信息滞后、路径损耗与静态规则共同限制了系统的优化能力;

2) 提出信息与能量双生系统,构建了全新的能量共享系统设计框架。该系统通过五层处理架构整合能量与信息系统,采用能量信息年龄优先级技术(AOEI)、数字孪生状态同步技术、自适应时长规划技术(ALDP)与机会主义信息收集技术(EETOR)等关键技术,各技术相互耦合、协同工作,从根本上整合了能量与信息系统,以延长网络寿命和平衡能量分布为核心目标;

3) 从理论层面证明所提出系统能够在不增加总能量消耗的前提下,同时实现网络寿命的显著延长和能量分布的充分平衡;

4) 通过仿真实验验证了系统的有效性,表明本文方法可在不增加总能量消耗的前提下显著延长首个节点死亡时间,降低能量变异系数(CV),提升网络整体能量均衡度。
